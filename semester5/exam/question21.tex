\begin{col-answer-preambule}
	\begin{plan}
    \item пока пусто
	\end{plan}
\end{col-answer-preambule}

\colquestion{Интерполяционные методы Адамса}

Являются неявными методами и определяются расчётной формулой
\begin{equation}
  y_{n + 1} = y_n + h\sum\limits_{i = -1}^qA_if_{n - i}
\end{equation}

Параметры $A_i$ определяются также, как и в методе последовательного повышения
порядка точности.
\begin{equation}
  \begin{split}
    &\alpha_i = -i, i = \overline{1, q}\\
    &\sum\limits_{i = -1}^qA_i(-i)^j = \dfrac{1}{j + 1}, j = \overline{0, q + 1}
  \end{split}
\end{equation}

Система имеет единственное решение при $q \geqslant -1$.

Параметры $A_i$ можно найти, используя интерполяционный полином Лагранжа, который
строится по значениям функции $f$ в узлах
\begin{align*}
  &x: x_{n + 1}, x_n, \ldots x_{n - q}\\
  &\alpha: 1, 0, \ldots -q
\end{align*}

Получим полином $(q + 1)$-ой степени, где
\begin{align}
  &A_i = \dfrac{(-1)^{i + 1}}{(i + 1)!(q - i)!}\int\limits_0^1
  \dfrac{(\alpha - 1)\alpha(\alpha + 1)\ldots(\alpha + 1)}{(\alpha + i)}d\alpha\\
  &r_{n + 1} = h^{q + 3}u_n^{(q + 3)}\left(\dfrac{1}{(q + 3)!}
  - \dfrac{1}{(q + 2)!}\sum\limits_{i = -1}^q A_i(-i)^{q + 2}\right)
  + O(h^{q + 4}) = O(h^{q + 3})
\end{align}

Т.е. метод имеет порядок точности $(q + 2)$.

\begin{examples}
  \item $q = -1$.
  \begin{equation}
    y_{n + 1} = y_n + h f_{n + 1} \text{ совпадает с неявным методом Эйлера}
  \end{equation}
  \begin{align*}
    r_{n + 1} = -\dfrac{1}{2}h^2u''_n + O(h^3) = O(h^2)
  \end{align*}
  Для случая, когда сетка равномерная, можно подынтегральную функцию заменить
  интерполяционным полиномом Ньютона.
  \item $q = 0$.
  \begin{align}
    &y_{n + 1} = y_n + \dfrac{h}{2}(f_{n - 1} + f_n) \text{ - формула трапеций}\\
    \nonumber
    &r_{n + 1} = -\dfrac{1}{12}h^3u_n^{(3)} + O(h^4) = O(h^3)
  \end{align}
  В общем виде:
  \begin{equation}
    y_{n + 1} = y_n + \phi_{n + 1} - \dfrac{1}{2}\Delta\phi_n - \dfrac{1}{12}
    \Delta^2\phi_{n - 1} - \dfrac{1}{24}\Delta^3\phi_{n - 2} - \ldots
    - C_{q + 1}\Delta^{q + 1}\phi_{n - q}
  \end{equation}
  \begin{align*}
    \phi_i = hf_i, C_{q + 1} = \dfrac{1}{(q + 1)!}\int\limits_0^1\alpha
    (\alpha + 1)\ldots(\alpha + q)d\alpha
  \end{align*}
  \begin{equation}
    r_{n + 1} = h^{q + 3}u_n^{(q + 3)}C_{q + 2} + O(h^{q + 4})
  \end{equation}
\end{examples}

Поскольку отрезок $[x_n, x_{n + 1}]$, на котором аппроксимируется функция $f$
входит в отрезок $[x_{n - q}, x_{n + 1}]$, на котором расположены узлы интерполяции...

Интерполяционные формулы Адамса представляют собой в общем случае неявное
уравнение относительно $y_{n + 1}$. Значение $y_{n + 1}$ находится с помощью
некоторого итерационного метода.

В качестве нулевой итерации обычно берут значение, приближённое значение,
полученной с помощью эксстраполяционного метода Адамса или методом Рунге-Кутты.
При этом часто ограничиваются только одной итерацией. В этом случае вычислительный
процесс относится к типу \textbf{предиктор-корректор}.
