\begin{col-answer-preambule}
	\begin{plan}
    \item пока пусто
	\end{plan}
\end{col-answer-preambule}

\colquestion{Простейшие разностные операторы}

% \subsection{Основные понятия в теории разностных схем}
%
% Сеточные (разностные) методы основаны на переходе от функций непрерывного аргумента к
% функциям дискретного аргумента. Например, если на отрезке $[0, 1]$ ввести точки-узлы $x_i$,
% которые образуют множество
% \begin{equation*}
%   \omega_h = \set{x_i = ih, i = \overline{0, h}, nh = l}, h \text{ - шаг сетки}
% \end{equation*}
% то приближённое решение ищется в узлах сетки $\omega_h$ и обозначается $y_n(x_i),
% i = \overline{0, n}$ ($y_n$ - функция дискретного аргумента).
%
% Для нахождения этой сеточной функции формулируется разностная задача
% \begin{equation}
%   Lu = F
% \end{equation}
% где $u$ - искомая функция, $F$ - вектор правой части, содержащий входные данные задачи.
% Например, рассмотрим задачу Коши
%
% \begin{equation}
%   \begin{cases}
%     u' = f(x, u)\\
%     u(0) = u_0
%   \end{cases}
% \end{equation}
% соответствующая разностная задача записывается в следующем виде
% \begin{equation}
%   Lu = \begin{cases}
%     u' - 1, x > 0\\
%     u(0), x = 0
%   \end{cases}
%   F = \begin{cases}
%     0, x > 0\\
%     u_0, x = 0
%   \end{cases}
% \end{equation}
%
% Одна из аппроксимирующих эту задачу разностных схем имеет вид
% \begin{equation}
%   \begin{cases}
%     \dfrac{y_{i + 1} - y_i}{h} = f_i, i = \overline{0, n - 1}\\
%     y_0 = u_0
%   \end{cases}
% \end{equation}
% В операторной форме
% \begin{align}
%   &\Omega_ny_n = \phi_n \text{ - в общем случае}\\
%   &\Omega_ny_n = \begin{cases}
%     \dfrac{y_{i + 1} - y_i}{h}\\
%     y_0
%   \end{cases}
%   \phi_n = \begin{cases}
%     f_i\\
%     u_0
%   \end{cases}
% \end{align}
%
% Введём норму остаточной функции
% \begin{equation}
%   \abs{\abs{y_n}}_h = \underset{0 \leq}{max}
% \end{equation}

Область решения $\overline{\Omega} = [0, l]$.

Сетка узлов на этой области
\begin{align*}
  \overline{\omega_h} = \set{x_i = ih, i = \overline{0, n}, hn = l}
\end{align*}

Построим аппроксимацию производной
\begin{align*}
  Lu = u'
\end{align*}

Функцию $u$ будем считать достаточно гладкой: $u(x) \in C^k(\Omega), k > 2$. Поставим
в соответствие оператору $Lu$ разностный оператор $\Lambda_h$.

\begin{definition}
  Множество узлов сетки, которое используется для построения оператора $\Lambda_h$
  называется \textbf{шаблоном}.
\end{definition}

\textbf{Погрешность аппроксимации} оператора $Lu$ разностным оператором $\Lambda_h$
в $i$-ом узле определяется как
\begin{align*}
  \psi_i = \Lambda_hu_i - (Lu)_i
\end{align*}

Будем использовать разложение в ряд Тейлора в окрестности точки $x_i$, где
\begin{align*}
  u_{i\pm 1} = u_i \pm hu_i' + \dfrac{1}{2}h^2u_i'' \pm \dfrac{1}{6}h^3u_i''' + O(h^3)
\end{align*}

Используя это разложение можно построить разностную схему оператора левой разностной
производной:
\begin{align}
  \label{23-left}
  u_{\overline{x}} = \dfrac{u_i - u_{i - 1}}{h} = u_i' - \dfrac{h}{2}u_i'' + O(h^2)
\end{align}
Оператор правой разностной производной
\begin{align}
  \label{23-right}
  u_{\overline{x}} = \dfrac{u_{i + 1} - u_i}{h} = u_i' + \dfrac{h}{2}u_i'' + O(h^2)
\end{align}

Минимальный шаблон - 2 узла. $u_{\overline{x}, i + 1} = u_{x, i}$. Если будем
использовать шаблон из трёх узлов, то можно построить центральную разностную
производную
\begin{equation}
  \label{23-center}
  u_{\mathring{x}} = \dfrac{u_{i + 1} - u_{i - 1}}{2h} = \dfrac{1}{2}
  (u_{\overline{x}, i} + u_{x, i}) = u_i' + \dfrac{h^2}{6}u''' + O(h^3)
\end{equation}

Для оператора второй производной можно применить линейную комбинацию левой и правой
производной.
\begin{align}
  \label{23-2nd}
  &\nonumber Lu = u''\\
  &(u_{\overline{x}})_x = \dfrac{1}{h}(u_x - u_{\overline{x}}) = \dfrac{1}{h^2}
  (u_{i + 1} - 2u_i + u_{i - 1})
\end{align}

Погрешность аппроксимации оценивалась в отдельном $i$-ом узле. Для оценки на всей
сетке $\omega_h$ необходимо использовать сеточные нормы
\begin{equation}
  \begin{split}
    &\abs{\abs{\psi}}_{C,h} = \underset{x \in \omega_h}\max\abs{\psi(x)}\\
    &\abs{\abs{\psi}}_{2, h} = \left(\sum\limits_{x \in \omega_h}
    \psi^2(x)h\right)^{\frac{1}{2}}
  \end{split}
\end{equation}

\eqref{23-left} - \eqref{23-2nd} имеют одинаковый порядок аппроксимации. В общем
случае порядок аппроксимации может быть разным в различных сеточных нормах.

В качестве альтернативного подхода можно использовать определение производной как
решение интегралного уравнения и применения некоторой квадратурной формулы
\begin{align*}
  \dfrac{d^ku}{dx^k} = f(x)
\end{align*}
\begin{equation}
  \label{23-le}
  u(x) = \dfrac{1}{(k - 1)!}\int\limits_0^x(x - t)^{k - 1}f(t)dt
\end{equation}

Таким образом, можно определить производную как решение интегрального уравнения
\eqref{23-le} при известной функции $u(x)$.

\begin{example}
  $k = 1$.
  \begin{equation}
    \label{23-ex}
    u_{i + 1} - u_{i - 1} = \int\limits_{x_{i - 1}}^{x_{i + 1}}f(t)dt
  \end{equation}
  Если для вычисления \eqref{23-ex} использовать формулу центральных
  прямоугольников, то получим
  \begin{equation}
    \dfrac{u_{i + 1} - u_{i - 1}}{2h} = f_i + O(h^2)
  \end{equation}
\end{example}

Есть и другие варианты построения. Например, строить интерполяционный полином
и брать производную. Использование квадратур с многими внутренними узлами приводит
к так называемым \textbf{компактным разностным операторам}.

Если в формулы \eqref{23-ex} вычислять интеграл с использованием трёхточечной
формулы Симпсона, то получим
\begin{equation}
  \dfrac{u_{i + 1} - u_{i - 1}}{2h} = \dfrac{1}{6}(f_{i - 1} + 4f_i + f_{i + 1})
  + O(h^4).
\end{equation}

В этом случае без расширения шаблона достигается более высокий порядок аппроксимации,
но при этом вычисление связано с обращением трёхдиагональной матрицы.
