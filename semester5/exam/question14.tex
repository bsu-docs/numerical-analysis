\begin{col-answer-preambule}
	\begin{plan}
    \item пока пусто
	\end{plan}
\end{col-answer-preambule}

\colquestion{Метод наименьших квадратов}

Если функция $f(x)$ задана на конечном множестве узлов $x_j$, другими
словами, $f(x)$ - сеточная функция, то скалярное произведение определяется не
интегралом, а суммой:
\begin{equation}
  \begin{split}
    &(f, g) = \sum\limits_{i = 1}^m\rho_i f_i g_i, f_i = f(x_i),\\
    & \rho_i > 0 - \text{ весовые коэффициенты}.
  \end{split}
\end{equation}

Будем рассматривать полиномиальную аппроксимацию многочлена. Тогда базисные
функции -
\begin{equation}
  \label{super-polinomials}
  g_k(x) = x^k, k = \overline{0, n}
\end{equation}

Если значения $f$ задаются в $(n + 1)$ разных точках, то существует единственный
интерполяционный полином степени не выше $n$.

Во многих случаях значения $f$ находят в результате измерений и содержат ошибки.
При этом число измерений проводят гораздо большее число раз, чем $(n + 1)$,
надеясь при этом в результате измерения уменшить эти ошибки.

Обычно в качестве такого метода усреднения выбирают метод наименьших квадратов.

Для базиса из полиномов \eqref{super-polinomials} система определяет элемент
наилучшего определения.

\begin{equation}
  \label{14-system}
  \sum\limits_{i = 0}^n \alpha_i(g_i, g_j) = (f_i, g_j), \forall j = \overline{0, n}
\end{equation}

Имеет следующий вид: $[(g_0, g_0) = (1, 1) = m]$.
\begin{equation}
  \label{14-normal-equation}
  \begin{bmatrix}
    m & \sum\limits x_i & \sum\limits x_i^2 & \ldots & \sum x_i^n\\
    \sum x_i & \sum x_i^2 & \sum x_i^3 & \ldots & \sum x_i^{n + 1}\\
    \vdots & \vdots & \vdots & \ddots & \vdots\\
    \sum x_i^n & \sum x_i^{n + 1} & \sum x_i^{n + 2} & \ldots & \sum x_i^{2n}\\
  \end{bmatrix}\cdot
  \begin{bmatrix}
    \alpha_0\\
    \alpha_1\\
    \vdots\\
    \alpha_n
  \end{bmatrix}
  = \begin{bmatrix}
    \sum f_i\\
    \sum x_i f_i\\
    \vdots\\
    \sum x_i^nf_i\\
  \end{bmatrix}
\end{equation}

Уравнения в \eqref{14-normal-equation} называются нормальными.
\begin{align*}
  \phi = \alpha_0 + \alpha_1x + \ldots + \alpha_nx^n.
\end{align*}

На практике, когда $n \geqslant 5$ нормальные уравнения обычно становятся плохо
обусловленными. Решить эту проблему можно с помощью ортогональных полиномов.

Будем говорить, что полиномы $g_j$, где $j$ - степень полинома, образуют на
множестве точек $x_1, \ldots x_m$ ортогональную систему, если
\begin{equation}
  (g_k, g_j) = \sum\limits_{i = 1}^m g_k(x_i)g_j(x_i) = 0, \forall k \neq j,
  k, j = \overline{0, n}.
\end{equation}

Тогда система \eqref{14-system} будет иметь вид
\begin{equation}
  \label{14-cons}
  \sum\limits_{i = 1}^mg_k^2(x_i)\alpha_k = \sum\limits_{i = 1}^mg_k(x_i)f_i,
  k = \overline{0, n}.
\end{equation}

Из \eqref{14-cons}
\begin{equation}
  \alpha_k = \dfrac{\sum\limits_{i = 1}^m g_k(x_i)f_i}
                   {\sum\limits_{i = 1}^m g_k^2(x_i)}
\end{equation}

Для полинома Чебышева: $T_p = 1, T_1 = x, \ldots$.
$T_{n + 1} = 2xT_n - T_{n - 1}$ - частный случай ортогональных полиномов с
$\rho = \dfrac{1}{\sqrt{1 - x^2}}$. Элемент наилучшего приближения
\begin{equation}
  \phi(x) = \sum\limits_{k = 0}^n\alpha_kg_k(x)
\end{equation}

Геометрический смысл -проекция.
