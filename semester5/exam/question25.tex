\begin{col-answer-preambule}
	\begin{plan}
    \item пока пусто
	\end{plan}
\end{col-answer-preambule}

\colquestion{Интегро-интерполяционный метод}

Вывод основных дифференциальных уравнений основан на записи соответствующих
законов сохранения для элементарных объёмов и стягивания этих объёмов к нулю.
Разностные схемы, выражающие законы сохранения на сетке, нызываются
\textbf{консервативными}. При этом законы сохранения для всей сетки должны быть
алгебраическим следствием разностных уравнений. Для построения консервативных
разностных схем естественно исходить из законов сохранения для отдельных ячеек
разностной сетки. Такой метод называется методом балланса или интегрально-
интерполяционным методом.

Будем рассматривать этот подход на примере одномерного стационарного уравнения диффузии.

\begin{align}
  \label{25-eq}
  &-(k(x)u'(x))' = f(x), 0 < x < l\\
  \nonumber
  &k(x) \geqslant k_1 > 0, k_1 = \const
\end{align}

В качестве граничных условий на левой границе возьмём условие третьего рода
\begin{equation}
  \label{25-initial}
  -k(0)u'(0) + \sigma u(0) = g_1
\end{equation}
на правой границе - условие Дирихле.
\begin{equation}
  u(l) = g_2
\end{equation}
Введём обозначения
\begin{equation}
  q \equiv -k(x)u'(x)
\end{equation}

Будем строить разностную схему на равномерной сетке $\overline{\omega_h}$.
Проинтегрируем
уравнение \eqref{25-eq} на отрезке $x_{i - \frac{1}{2}} \leqslant x
\leqslant x_{i + \frac{1}{2}}$.
\begin{equation}
  \label{25-int}
  q_{i + \frac{1}{2}} - q_{i - \frac{1}{2}} = \int\limits_{x_{i - \frac{1}{2}}}^
  {x_{i + \frac{1}{2}}}f(t)dt
\end{equation}

\eqref{25-int} отражает закон сохранения количества тепла для отрезка
$[x_{i - \frac{1}{2}}, x_{i + \frac{1}{2}}]$. Величина $q_{i - \frac{1}{2}}$ даёт
количество тепла, вытекшего через сечение $x_{i - \frac{1}{2}}$, а $q_{i +\frac{1}{2}}$
- количество тепла, втекающее через сечение $x_{i + \frac{1}{2}}$. Дисбаланс этих
потоков обусловлен распределением источников тепла.

Из определения потока находим
\begin{equation}
  \label{25-diff}
  u' = - \dfrac{q}{k}
\end{equation}

Интегрируем \eqref{25-diff} на $[x_{i - 1}, x_i]$
\begin{equation}
  \label{25-int2}
  u_i - u_{i - 1} = \int\limits_{x_{i - 1}}^{x_i}\dfrac{q}{k}dx =
  q_{i - \frac{1}{2}}\int\limits_{x_{i - 1}}^{x_i}\dfrac{dx}{k}
\end{equation}

Введём обозначение
\begin{equation}
  \label{25-def}
  a_i \equiv \left(\dfrac{1}{h}\int\limits_{x_{i - 1}}^{x_i}\dfrac{dx}{k}\right)^{-1}
\end{equation}

Из \eqref{25-int2}, \eqref{25-def}
\begin{equation}
  \label{25-q}
  \begin{split}
    &q_{i - \frac{1}{2}} = -a_i\dfrac{u_i - u_{i - 1}}{h} = -a_i\cdot
    u_{\overline{x}, i}\\
    &q_{i + \frac{1}{2}} = -a_{i + 1}u_{x, i}
  \end{split}
\end{equation}

Если применим формулу средних прямоугольников, по получим $a_i = k(x_i - \dfrac{h}{2})$.

Если применим формулу трапеций, то получим $a_i = \dfrac{1}{2}(k_i + k_{i - 1})$.
Погрешность аппроксимации: $O(h^2)$.

Подставляем \eqref{25-q} в \eqref{25-int}, получим
\begin{align}
  &-(au_{\overline{x}})_x = \phi\\
  \phi = \dfrac{1}{h}\int\limits_{x_{i - 1}}^{x_i}fdx\\
\end{align}

На правой границе было условие Дирихле
\begin{equation}
  u(l) = g_2
\end{equation}

Для аппроксимации значения на левой границе используем интегрально-интерполяционный
подход. Интегрируем уравнение \eqref{25-eq} на отрезке $\left[0, \dfrac{h}{2}\right]$.

\begin{align}
  &q_{\frac{1}{2}} - q_0 = \int\limits_0^{x_{\frac{1}{2}}}fdx\\
  &q_{\frac{1}{2}} = -a_1u_{x, 0} \text{ - аппроксимация в дробном узле}\\
  &\eqref{25-initial} \Rightarrow q_0 = g_1 - \sigma u_0
\end{align}

В итоге получаем следующую разностную аппроксимацию граничных условий \eqref{25-initial}
\begin{align}
  &-a_1u_{x, 0} + \sigma u_0 = g_1 + \dfrac{h}{2}\phi_0\\
  &\phi_0 = \dfrac{1}{h}\int\limits_0^{\frac{h}{2}}fdx
\end{align}

Аналогично строится $\phi$ на правой границе (там будут заданы условия второго или
третьего рода).
