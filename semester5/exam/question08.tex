\begin{col-answer-preambule}
	\begin{plan}
    \item пока пусто
	\end{plan}
\end{col-answer-preambule}

\colquestion{Минимизация остатка интерполирования}

\textit{Критерий}.\newline
Чебышев показал, что из всех полиномов степени $n$, $P_n(x)$ со старшим коэффициентом равным $1$, у полинома
\begin{equation*}
  2^{1-n} T_n(x) = \overline{T}(x).
\end{equation*}

точная верхняя грань абсолютных значений на $[-1, 1]$ наименьшая и равна $2^{1-n}$, т.к. $\max\limits_{[-1, 1]} |T_n(x)| = 1$.

$\left|\left| \overline{T_n} - 0 \right| \right|_{\infty, [-1, 1]} = 2^{1-n} \Rightarrow \overline{T_n} (x)$ - полином, наименее отклоняющийся от нуля.

В случае отрезка произвольной длины $x \in [a, b]$ сделаем линейную замену переменных, которая отображает $[a, b]$ на $[-1, 1]$.

\begin{equation}\label{eq:8_2}
  x = \dfrac{b-a}{2} t + \dfrac{a+b}{2} = \psi(t), t \in [-1, 1].
\end{equation}

\begin{equation*}
  P_n(x) = x^n + P_{n-1}(x) = \psi^n (t) + P_{n-1} \Big( \psi(t) \Big) = \left(\dfrac{b-a}{2}\right)^n \overline{P_n}(t).
\end{equation*}

\begin{equation*}
  \left|\left| P_n(x) \right| \right|_{\infty, [a, b]} = \left(\dfrac{b-a}{2}\right)^n \left|\left| \overline{P_n}(t) \right| \right|_{\infty, [-1, 1]} \geqslant 2^{1-n} \left( \dfrac{b-a}{2} \right)^n = (b-a)^n 2^{1-2n}.
\end{equation*}

Равенство в этой формуле достигается при
\begin{equation*}
  \overline{T_n}\overset{[a, b]}{(x)} = (b - a)^n 2^{1-2n} T_n \left( \dfrac{2x - a - b}{b - a} \right).
\end{equation*}

$\overline{T_n}\overset{[a, b]}{(x)}$ называется наименее отклоняющимся от нуля на отрезке $[a, b]$.

В силу замены переменных \eqref{eq:8_2} корни $\overline{T_n}\overset{[a, b]}{(x)}$ находятся по формуле:
\begin{equation*}
  x_m = \dfrac{a+b}{2} + \dfrac{b-a}{2} \cos \Big( \dfrac{\pi (m + \frac{1}{2})}{n} \Big), m = \overline{0, n-1}.
\end{equation*}

Для оценки остатка интерполирования функции $f(x)$ на Чебышевской сетке узов $\{ x_m \}$ будем использовать равномерную норму:
\begin{equation*}
  \left| \left| f(x) \right| \right|_{\infty} = \sup\limits_{[a, b]} |f(x)|.
\end{equation*}

Из представления остатка в общем виде следует
\begin{equation*}
  \left| \left| f(x) - P_{n-1}(x) \right| \right| \leqslant \dfrac{1}{n!} \left| \left| f^{(n)}(x) \right| \right|  \left| \left| w_n(x) \right| \right|.
\end{equation*}

Будем минимизировать правую часть в неравенстве выше.
\begin{equation*}
  w_n = (x - x_1) \ldots (x - x_n), \text{deg }w_n = n, \text{ старший коэффициент равен 1}.
\end{equation*}

Поэтому в качестве узлов интерполирования $x_1, \ldots, x_n$ мы можем взять корни многочлена Чебышева. В этом случае $w_n$ будет иметь вид:

\begin{equation*}
  w_n = \overline{T_n}\overset{[a, b]}{(x)} = (b - a)^n 2^{1-2n} T_n \left( \dfrac{2x - a - b}{b - a} \right).
\end{equation*}

Из равенства выше следует что $|| w_n (x) || = (b - a)^n 2^{1-2n} \Rightarrow$ на Чебышевском наборе узлов оценка погрешности интерполирования имеет вид:
\begin{equation*}
  \left| \left| f(x) - P_{n}(x) \right| \right| \leqslant \dfrac{1}{n!} \left| \left| f^{(n)}(x) \right| \right| (b - a)^n 2^{1-2n}.
\end{equation*}

Мы получили неуменьшаемую оценку погрешности интерполяции.
