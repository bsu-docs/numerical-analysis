\begin{col-answer-preambule}
	\begin{plan}
    \item пока пусто
	\end{plan}
\end{col-answer-preambule}

\colquestion{Основные понятия теории разностных схем}

Сеточные (разностные) методы основаны на переходе от функций непрерывного аргумента к
функциям дискретного аргумента. Например, если на отрезке $[0, 1]$ ввести точки-узлы
$x_i$, которые образуют множество
\begin{equation*}
  \omega_h = \set{x_i = ih, i = \overline{0, h}, nh = l}, h \text{ - шаг сетки}
\end{equation*}
то приближённое решение ищется в узлах сетки $\omega_h$ и обозначается $y_h(x_i),
i = \overline{0, n}$ ($y_n$ - функция дискретного аргумента).

Для нахождения этой сеточной функции формулируется разностная задача. Запишем её в
операторном виде
\begin{equation}
  \label{24-op}
  Lu = F
\end{equation}
где $u$ - искомая функция, $F$ - вектор правой части, содержащий входные данные задачи.

Например, рассмотрим задачу Коши
\begin{equation}
  \begin{cases}
    u' = f(x, u)\\
    u(0) = u_0
  \end{cases}
\end{equation}
соответствующая разностная задача записывается в следующем виде
\begin{equation}
  Lu = \begin{cases}
    u' - f, x > 0\\
    u(0), x = 0
  \end{cases}
  F = \begin{cases}
    0, x > 0\\
    u_0, x = 0
  \end{cases}
\end{equation}

Одна из аппроксимирующих эту задачу разностных схем имеет вид
\begin{equation}
  \begin{cases}
    \dfrac{y_{i + 1} - y_i}{h} = f_i, i = \overline{0, n - 1}\\
    y_0 = u_0
  \end{cases}
\end{equation}
Операторная форма записи
\begin{align}
  \label{24-diff}
  &\Lambda_hy_n = \phi_h \text{ - в общем случае}\\
  &\Lambda_hy_n = \begin{cases}
    \dfrac{y_{i + 1} - y_i}{h}\\
    y_0
  \end{cases}
  \phi_h = \begin{cases}
    f_i\\
    u_0
  \end{cases}
\end{align}

Введём норму сеточной функции
\begin{equation}
  \abs{\abs{y_n}}_h = \underset{0 \leqslant i \leqslant n}{\max}\abs{y_i}
\end{equation}

\begin{definition}
  Решение разностной задачи \eqref{24-diff} сводится при $h \to 0$ к решению исходной
  задачи \eqref{24-op}, если
  \begin{equation}
    \norm{(u)_h - y_h}_h \to 0
  \end{equation}
\end{definition}
$(u)_h$ означает некоторую проекцию точного решения $u$ на сетку $\omega_h$. Самая
простая проекция: $u_i$. Если
\begin{equation}
  \norm{(u)_h - y_h}_h \leqslant ch^p,
\end{equation}
где $c = \const$, не зависит от $h$, то имеет место сходимость порядка $p$.

\begin{definition}
  Задача \eqref{24-diff} аппроксимирует исходную задачу \eqref{24-op} на её решение,
  если невязка
  \begin{equation}
    \norm{\psi_h}_h \xrightarrow[h \to 0]{} 0
  \end{equation}
\end{definition}
\begin{align*}
  \psi_h = \Lambda_h(u)_h - \phi_h
\end{align*}

Если при этом имеет место оценка
\begin{equation}
  \norm{\psi_h}_h \leqslant C_1h^p,
\end{equation}
где $C_1 = \const$ не зависящая от $h$, то схема \eqref{24-diff} имеет порядок
аппроксимации $p$.

\begin{definition}
  Разностная схема \eqref{24-diff} называется устойчивой, если для любых достаточно
  малых $h, z_h$ возмущённая разностная схема
  \begin{equation}
    \Lambda_hv_h = \phi_h + z_h
  \end{equation}
  однозначно разрешима и $\exists \const c_2 > 0$ не зависящая от $h, \norm{z_h}_h$,
  такая, что выполняется оценка
  \begin{equation}
    \norm{v_h - y_h}_h \leqslant c_2\norm{z_h}_h
  \end{equation}
  Другими словами, малые возмущения правой части разностной схемы приводят к равномерно
  малому по $h$ изменению решения.
\end{definition}

Поскольку
\begin{align*}
  \eqref{24-diff} \Rightarrow y_h = \Lambda_h^{-1}(\phi_h), v_h =
  \Lambda_h^{-1}(\phi_h + z_h),
\end{align*}
то условие устойчивости можно записать следующим образом
\begin{equation}
  \label{24-stab}
  \norm{\Lambda_h^{-1}(\phi_h + z_h) - \Lambda_h^{-1}(\phi_h)}_h \leqslant c_2
  \norm{z_h}_h
\end{equation}

\eqref{24-stab} означает непрерывность обратного оператора $\Lambda_h^{-1}$ в точке
$\phi_h$.

\begin{theorem}[Лакса]
  Любая устойчивая разностная схема $p$-го порядка аппроксимации на решении является
  схемой $p$-го порядка сходимости.
\end{theorem}
