\begin{col-answer-preambule}
	\begin{plan}
    \item пока пусто
	\end{plan}
\end{col-answer-preambule}

\colquestion{Разделённые разности и их свойства}

Разделенные разности нулевого порядка совпадают со значениями функциями $f(x_i) = f_i$. Разделенная разность первого порядка записывается как
\begin{equation*}
  f(x_i, x_j) = \dfrac{f_j - f_i}{x_j - x_i}.
\end{equation*}

Разделенная разность второго порядка:
\begin{equation*}
  f(x_i, x_j, x_k) = \dfrac{f(x_j, x_k) - f(x_i, x_j)}{x_k - x_i}.
\end{equation*}

Разделенная разность $k$-ого порядка:
\begin{equation*}
  f(x_1, x_2, \ldots, x_{k+1}) = \dfrac{f(x_2, \ldots, x_{k+1}) - f(x_1, \ldots, x_k)}{x_{k+1} - x_1}.
\end{equation*}

\begin{lemma}
  Разделенную разность можно представить в виде суммы
  \begin{equation*}
    f(x_1, \ldots, x_n) = \sum\limits_{j=1}^n \dfrac{f_j}{w^{'}_n (x_j)},
  \end{equation*}
  где $w(x) = (x - x_1) \ldots (x - x_n)$.
\end{lemma}
\begin{proof}
ММИ.
  \begin{enumerate}
    \item $n = 1$. $f_1 = f_1$.
    \item $n = 2$. $f(x_1, x_2) = \dfrac{f_2 - f_1}{x_2 - x_1} = \dfrac{f_1}{x_1 - x_2} + \dfrac{f_2}{x_2 - x_1}$.
    \item $n = k$. $k-1$ порядка:
    \begin{equation*}
      f(x_1, \ldots, x_k) = \sum\limits_{j=1}^k \dfrac{f_j}{w^{'}_k(x_j)}
    \end{equation*}
    \begin{equation*}
      f(x_2, \ldots, x_{k+1}) = \sum\limits_{j=2}^{k+1} \dfrac{f_j}{w^{'}_k(x_j)}
    \end{equation*}
    
    \item докажем для $n = k + 1$. $k$-ый порядок:
    \begin{equation*}
      f(x_1, \ldots, x_{k+1}) = \dfrac{f(x_2, \ldots, x_{k+1}) - f(x_1, \ldots, x_k)}{x_{k+1} - x_k} = \dfrac{1}{x_1 - x_{k + 1}} \sum\limits_{j=1}^k \dfrac{f_j}{w^{'}(x_j)} + \dfrac{1}{x_{k + 1} - x_1} \sum\limits_{j=2}^{k+1} \dfrac{f_j}{w^{'}(x_j)} =  \sum\limits_{j=1}^{k+1} \dfrac{f_j}{w^{'}_{k+1}(x_j)}.
    \end{equation*}
    Найдем коэффициент при $f_j$:
    \begin{equation*}
      \dfrac{1}{x_{k+1} - x_1} \left( \dfrac{1}{\prod\limits_{i = 2, i \ne j}^{k+1} (x_j - x_i)} - \dfrac{1}{\prod\limits_{i = 1, i \ne j}^{k} (x_j - x_i)} \right) = \dfrac{1}{x_{k+1} - x_1} \cdot \dfrac{1}{\prod\limits_{i = 1, i \ne j}^{k+1} (x_j - x_i)} \left( (x_j - x_i) - (x_j - x_{k+1}) \right) = 
    \end{equation*}
    \begin{equation*}
      = \dfrac{1}{\prod\limits_{i = 1, i \ne j}^{k+1} (x_j - x_i)} = \dfrac{1}{w^{'}_{k+1}(x_j)}
    \end{equation*}
  \end{enumerate}
\end{proof}
\begin{consequences}
  \item Разделенная разность - линейный оператор функции $f$:
  \begin{equation*}
    \left( \alpha_1 f_1(x) + \alpha_2 f_2 (x) \right) (x_1, x_2, \ldots, x_n) = \alpha_1 f_1(x_1, \ldots, x_n) + \alpha_2 f_2(x_1, \ldots, x_n).
  \end{equation*}
  \item Разделенная разность симметрична относительно своих аргументов, т.е. не меняется от перестановки $\forall \; x_i, x_j$.
  \item Разделенные разности удобно записывать в виде таблицы:
  \begin{center}
    \begin{tabular}{ | c | c | c | c |}
      \hline
      $x_1$ & $f_1$ &              &                \\ \hline
            &       & $f(x_1, x_2)$ &                  \\ \hline
      $x_2$ & $f_2$ &               & $f(x_1, x_2, x_3)$ \\ \hline
            &       & $f(x_2, x_3)$ &                   \\ \hline
      $x_3$ & $f_3$ &               & $f(x_2, x_3, x_4)$\\ \hline
            &       & $f(x_3, x_4)$ &                   \\ \hline
      $x_4$ & $f_4$ &               &                 \\
        \hline
    \end{tabular}
  \end{center}
\end{consequences}
