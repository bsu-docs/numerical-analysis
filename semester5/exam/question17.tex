 \begin{col-answer-preambule}
	\begin{plan}
    \item пока пусто
	\end{plan}
\end{col-answer-preambule}

\colquestion{Сходимость явного метода Эйлера}

При иcпользовании приближённых методом основным является вопрос о сходимости.
Сформулируем понятие сходимости, когда $h \to 0$. Зафиксируем некоторую точку
$x$ и будем строить последовательность сеток $\omega_h$ таких, что $h \to 0,
x_n = x_0 + nh = x$.

\begin{definition}
  Говорят, что методя сходится в точке $x$, если разностное решение
  $\abs{y_n(x) - u(x)} \xrightarrow[h \to 0]{} 0$.
\end{definition}

\begin{definition}
  Метод сходится на отрезке $[x_0, x]$, если он сходится в каждой точке этого
  отрезка.
\end{definition}

\begin{definition}
  Говорят, что метод имеет порядок точности $p > 0$, если $\abs{y_n(x) - u(x)}
  = O(h^p), h > 0$.
\end{definition}

Исследуем сходимость явного метода Эйлера
\begin{align}
  &y_{n + 1} = y_n + hf_n\\
  \label{17-taylor}
  &u(x_{n + 1}) = u(x_n) + hu'(x_n) + \dfrac{h^2}{2}u''(\xi_n),\\
  \nonumber
  &x_n < \xi_n < x_{n + 1}, u(x) \in C^2[x_0, x]\\
  &u' = f(x, u)\\
  &\eqref{17-taylor} \Leftrightarrow u(x_{n + 1}) = u(x_n) + hf(x_n, u(x_n)) +
  \dfrac{h^2}{2}u''(\xi_n)\\
  \label{17-some-cons}
  &u(x_{n + 1}) - y_{n + 1} = u(x_n) - y_n + h(f(x_n, u(x_n)) - f(x_n, y_n)) +
  \dfrac{h^2}{2}u''(\xi_n)
\end{align}

Введём обозначение погрешности в $n$-ой точке
\begin{equation}
  E_n = u(x_n) - y_n
\end{equation}

Будем полагать, что функция $f$ удовлетворяет условию Липшица с $\const L$ по
второму аргумент, тогда $\eqref{17-some-cons} \Rightarrow$

\begin{equation}
  \abs{E_{n + 1}} \leqslant \abs{E_n} + hL(\abs{u(x_n) - y_n}) + \dfrac{h^2}{2}
  \abs{u''(\xi_n)}.
\end{equation}

\begin{definition}
  $M_2 = \underset{x \in [a, b]}{\max} \abs{u''(x)}$
\end{definition}

\begin{equation}
  \label{17-error-approx}
  \abs{E_{n + 1}} \leqslant \abs{E_n}(1 + hL) + \dfrac{h^2}{2}M_2, n = 0, 1,
  \ldots
\end{equation}

Слагаемое $\dfrac{h^2}{2}M_2$ - оценка локальной погрешности метода, которая
возникает на очередном шаге.

Для оценки погрешности $E_n$ рассмотрим обобщение неравенства
\eqref{17-error-approx}. Будем полагать, что $\exists \delta > 0, M > 0$,
такие, что последовательность $d_0, d_1, \ldots$ удовлетворяет неравенству
\begin{equation}
  d_{n + 1} \leqslant (1 + \delta)d_n + M, n = 0, 1, \ldots
\end{equation}

Тогда
\begin{align}
  \nonumber
  &d_1 \leqslant (1 + \delta)d_0 + M\\
  \nonumber
  &d_2 \leqslant (1 + \delta)d_1 + M \leqslant (1 + \delta)^2d_0 +
  M(1 + (1 + \delta))\\
  \nonumber
  &\ldots\\
  &d_n \leqslant (1 + \delta)^n + M(1 + (1 + \delta) + \ldots
  + (1 + \delta)^{n - 1})\\
  \label{17-dlim}
  &d_n \leqslant (1 + \delta)^nd_0 + M\dfrac{(1 + \delta)^n - 1}{\delta}
\end{align}

Из разложения экспоненты:
\begin{align}
  &e^{\delta} = 1 + \delta + \dfrac{\delta^2}{2}e^{\xi}, 0 < \xi < \delta
  \Rightarrow 1 + \delta \leqslant e^{\delta}\\
  \label{17-expn}
  &(1 + \delta)^n \leqslant e^{n\delta}
\end{align}

Подставим \eqref{17-expn} в \eqref{17-dlim} и получим оценку:
\begin{equation}
  \label{17-dlim2}
  d_n \leqslant e^{n\delta}d_0 + M\dfrac{e^{n\delta} - 1}{\delta}
\end{equation}

Применим неравенство \eqref{17-dlim2} к формуле \eqref{17-error-approx}
\begin{align}
  &\abs{E_n} \leqslant e^{nhL}\abs{E_0} + \dfrac{hM_2}{2L}
  \left(e^{nhL} - 1\right)\\
  \nonumber
  &nh = x_n - a, E_0 = u_0 - y_0 = 0\\
  &\abs{u(x_n) - y_n} \leqslant \dfrac{hM_2}{2L}\left(e^{L(b - a)} - 1\right)\\
  \label{17-max-error}
  &\underset{h}\max\abs{u(x_n) - y_n} \leqslant \dfrac{hM_2}{2L}
  \left(e^{L(b - a)} - 1\right) \xrightarrow[h \to 0]{} 0
\end{align}

В общем случае следует учитывать погрешность округления. На практике, когда
вычисляется $f(x_n, y_n)$, на самом деле мы находим $f(x_n, y_n) + \varepsilon_n$.
Кроме этого, когда по формуле Эйлера мы находим
\begin{equation}
  y_{n + 1} = y_n + h(f(x_n, y_n) + \varepsilon_n) + \rho_n
\end{equation}
появляется погрешность $\rho_n$.

Будем полагать, что $\abs{\rho_n} \leqslant \rho, \abs{\varepsilon_n} \leqslant
\varepsilon, \forall h \leqslant h_0$. Тогда формулу \eqref{17-max-error}
надо изменить следующим образом
\begin{equation}
  \label{17-real-error}
  \underset{h}\max\abs{u(x_n) - y_n} \leqslant \dfrac{1}{L}
  \left(e^{L(b - a)} - 1\right)\left(\dfrac{hM_2}{2} + \varepsilon +
  \dfrac{\rho}{h}\right)
\end{equation}

Из оценки \eqref{17-real-error} следует, что повышать точность за счёт уменьшения
шага $h$ можно только до некоторого предела, за которым погрешность округления
будет доминировать.
