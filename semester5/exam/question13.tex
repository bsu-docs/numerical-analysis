\begin{col-answer-preambule}
	\begin{plan}
    \item пока пусто
	\end{plan}
\end{col-answer-preambule}

\colquestion{Наилучшее приближение в гильбертовом пространстве}

Гильбертовое пространство является полным нормированным, $||x||^2 = (x_1, \ldots, x_n)$.

Для Гилбертова пространства $H$ элемент наилучшего приближения единственен и его построение сводится к решению системы линейных уравнений.

Обозначим $G = \text{span } \{ g_1, \ldots, g_n \}, g_i \in H, i = \overline{1, n}, f \in H$.

$|| f - \phi || = \inf\limits_{h \in G} || f - h||$.

$G$ называется линейным многообразием.

\begin{lemma}
  Пусть $\phi \in H$ - элемент наилучшего приближения. Тогда $(f - \phi) \perp G$ (ортогонален всем элементам).
\end{lemma}

\begin{lemma}
  Если погрешность $(f - \phi) \perp G$, где $G$ - линейная оболочка гилбертова пространства, то $\phi$ - элемент наилучшего приближения.
\end{lemma}

Пусть элемент наилучшего приближения $\phi$ имеет представление $\sum\limits_{j=1}^n \alpha_j g_{j}$. Коэффициенты $\alpha_j$ пока неизвестны:
\begin{equation*}
  f - \phi = f - \sum\limits_{j=1}^n \alpha_j g_{j}
\end{equation*}

\begin{equation}\label{eq:13_8}
  \left(f - \sum\limits_{j=1}^n \alpha_j g_{j}, g_i \right) \overset{\text{п. 1}}{=} 0,
\end{equation}
где $j = \overline{1, n}$ - свойство ортогональности, т.е. $(f - \phi) \perp G$.

\eqref{eq:13_8} - СЛАУ относительно $\alpha$.

Запишем систему \eqref{eq:13_8} в классической форме:
\begin{equation}\label{eq:13_9}
  \sum\limits_{j=1}^n \alpha_j (g_i, g_i) = (f, g_j), j = \overline{1, n}.
\end{equation}

Матричная система \eqref{eq:13_9} - матрица Грама. В силу того, что базисные элементы $g_1, \ldots, g_n$ линейно-независимы, определитель матрицы Грама $\ne 0 \Rightarrow$ \eqref{eq:13_9} имеет единственное решение относительно коэффициента $\alpha$.

\begin{notes}
  \item Если элементы $g_1, \ldots, g_n$ является ортонормированными, т.е. $(g_i, g_j) = \delta_{i, j}, i, j = \overline{1,n}$, то система \eqref{eq:13_8} имеет диагональную матрицу и решение находится как $d_j = (f, g_j), j = \overline{1, n}$. Тогда элемент наилучшего приближения $\phi = \sum\limits_{i=1}^n (f, g_i)g_i$.

  Коэффициент $\alpha_j$ имеет название коэффициента Фурье, а сам многочлен $\phi$ носит название многочлена Фурье.

  \item Если $(f, \phi) = || \phi ||^2$, тогда для элемента наилучшего приближения имеем
  \begin{equation*}
    ||f - \phi||^2 = ||f||^2 - ||\phi||^2.
  \end{equation*}

  В силу равенства Парсеваля $\left( ||f||^2 = \sum\limits_{k=1}^\infty |\alpha_k|^2 \right)$ имеем:

  \begin{equation*}
    ||f - \phi||^2 = \int\limits_{k = n + 1}^{
    \infty
    } |\alpha_k|^2
  \end{equation*}

  Значит, при $n \to \infty$ норма погрешности $|| f - \phi||$ неограниченно убывает, т.е. элемент наименьшего приближения $\phi$ среднеквадратичного сходится к $f$.

  \item Типичным примером гилбертова пространства является пространство $L_2 [a, b]$ - пространство функций $f(x)$, интегрируемых с квадратом на отрезке $[a, b]$, причем:

  \begin{equation*}
  (f, g)_{L_2} = \int\limits_a^b \rho(x) f(x) \overline{g(x)} dx,
  \end{equation*}

  \begin{equation*}
    \left| \left| f \right| \right|_{L_2}^2 = \left( \int\limits_a^b \rho(x) f^2(x) dx \right)^{\frac{1}{2}}
  \end{equation*}

  $\rho(x) \geqslant 0$ - весовая функция.

  $\rho(x) = 0$ на граничном числе точек, мера которого равна 0.

  \item Коэффициенты Фурье $\alpha_i$ дают наилучшие в системе наименьших квадратов приближения, когда $f(x)$ разлагается по ортогональному базису элементов $g_i$.

  Т.о., ортогональные функции $g_i$, нахождение коэффициентов Фурье и идея приближения в смысле наименьших квадратов тесно взаимосвязаны.
\end{notes}
