\begin{col-answer-preambule}
	\begin{plan}
    \item пока пусто
	\end{plan}
\end{col-answer-preambule}

\colquestion{Конечные разности и их свойства}

Основным оператором в исчислении конечных разностей является оператор $\Delta$, который определяется как:
\begin{equation*}
  \Delta f(x) = f(x + h) - f(x).
\end{equation*}

Из определения следует, что оператор $\Delta$ - линейный, т.е.
\begin{equation*}
  \Delta \left[ a f(x) + bg(x) \right] = a \Delta f(x) + b \Delta g(x).
\end{equation*}

Исследуем $\Delta \left[ f(x) g(x) \right]$.
\begin{equation*}
  \Delta \left[ f(x) g(x) \right] = \left[ def. \right ] = f(x+h) g(x+h) - f(x)g(x) =
\end{equation*}
\begin{equation*}
  = f(x+h)g(x+h) + f(x+h)g(x) - f(x+h)g(x) - f(x)g(x) = f(x+h) \Delta g(x) + g(x) \Delta f(x).
\end{equation*}

\begin{equation*}
  \Delta \left(\dfrac{f(x)}{g(x)} \right) = \left[ def. \right] = \dfrac{f(x+h)}{g(x+h)} - \dfrac{f(x)}{g(x)} = \dfrac{f(x+h)g(x) - f(x)g(x+h)}{g(x+h)g(x)} =
\end{equation*}
\begin{equation*} 
  = \dfrac{1}{g(x+h)g(x)} \Big[ f(x+h)g(x) - g(x+h)f(x) + f(x)g(x)-f(x)g(x) \Big] = \dfrac{1}{g(x+h)g(x)} \cdot \Big[\Delta f(x)g(x) - \Delta g(x)f(x) \Big]
\end{equation*}

В формуле выше $f$ и $g$ можно поменять местами.

\begin{equation*}
  \Delta \left (\Delta f \right) = \Delta^2 f.
\end{equation*}

\begin{equation*}
  \Delta^n f(x) = \Delta \left( \Delta^{n-1} f(x) \right) = \Delta^{n-1} \left( \Delta f(x) \right).
\end{equation*}

\begin{theorem}[основная теорема исчисления конечных разностей]
  Для многочлена степени $n$:
  \begin{equation*}
    f(x) = \alpha_0 + \alpha_1 x + \ldots + \alpha_n x^n,
  \end{equation*}
  где $\alpha_n \ne 0$, конечная разность $n$-ого порядка равна:
  \begin{equation*}
    \Delta^n f = \alpha_n n! h^n, \Delta^{n+1} f = 0.
  \end{equation*} 
\end{theorem}

\begin{lemma}
  Если $f(x)$ - многочлена степени $n$, то $\Delta f$ есть многочлен степени $n-1$.
\end{lemma}
\begin{proof}
  Рассмотрим в качестве $f = x^n$, тогда
  \begin{equation*}
    \Delta f(x) = (x+h)^n - x^n = \sum\limits_{k=0}^n C_n^k h^{n-k}x^k - x^n = n h x^{n-1} + \dfrac{n (n-1)}{1 \cdot 2} h^2 x^{n-2} + \ldots + h^n,
  \end{equation*}
  т.е. $\Delta f(x)$ - полином степени $n-1$.

  Т.о., $\Delta x^n$ является полиномом степени $n-1$.

  Используя свойство линейности, определяем, что оператор $\Delta$ уменьшает степень каждого члена полинома на 1.

  Кроме того, член $n x^{n-1} h \alpha_n$, которые получаем в результате применения $\Delta$ к последнему слагаемому, отличен от нуля, т.к. $\alpha_n \ne 0$.
\end{proof}
\begin{proof}
  теоремы перед леммой. Применим к многочлену $n$-ой степени лемму $n$ раз и убедимся, что $n$-ая разность постоянна, а коэффициент при $\alpha_n$ - $\left(n! h^n \right)$, а все следующие разности превращаются в ноль.
\end{proof}
