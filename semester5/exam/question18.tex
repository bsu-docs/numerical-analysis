\begin{col-answer-preambule}
	\begin{plan}
    \item пока пусто
	\end{plan}
\end{col-answer-preambule}

\colquestion{Методы последовательного повышения порядка точности}

Будем рассматривать уравнение $u' = f(x, y)$. Проинтегрируем на $[x_n, x_{n + 1}]$

\begin{equation}
  \label{18-integral}
  u(x_{n + 1}) = u(x_n) + h\int\limits_0^1f(x_n + \alpha h, u(x_n + \alpha h))
  d\alpha
\end{equation}

Заменим $t = x_n + \alpha h$. Заменим в \eqref{18-integral} интеграл квадратурной
суммой общего вида, получим
\begin{equation}
  \label{18-sum}
  y_{n + 1} = y_n + h\sum\limits_{i = 0}^qA_if(x_n + \alpha_ih, y(x_n + \alpha_ih)
\end{equation}

Получим 2 набора параметров $A_i$ и $\alpha_i, i = \overline{0, q}$. В
\eqref{18-sum} $2q + 2$ параметров.

Параметры $A_i$ и $\alpha_i$ выбираем так, чтобы квадратурная формула, которую
мы использовали
\begin{equation}
  \label{18-quad-formula}
  \int\limits_0^1f_{n + \alpha}d\alpha \approx
  \sum\limits_{i = 0}^qA_if_{n + \alpha_i},
\end{equation}
была точна для всех полиномов степени $k - 1$, где $0 < k \leqslant 2q + 2$.

В результате получим систему из $k$ уравнений, в которые входят $2q + 2$
неизвестных параметров.
\begin{equation}
  \label{18-system}
  \sum\limits_{i = 0}^qA_i\alpha_i^j = \dfrac{1}{j + 1}, j = \overline{0, k - 1}
\end{equation}

В дальнейшем будем использовать обозначение
\begin{align*}
  f_{n + \alpha}^{[m]} = f(x_n + \alpha h, y^{[m]}(x_n + \alpha h)),
\end{align*}
где $y_{n + \alpha}^{[m]}$ - приближённое решение в точке $x_n + \alpha h$ с
погрешностью $O(h^m)$.

Систему \eqref{18-system} можно получить также из требования, чтобы разложить
в ряд Тейлора по степеням $h$
\begin{equation}
  u(x_n + h) - u(x_n) \approx h\sum\limits_{i = 0}^qA_iu'(x_n + \alpha h)
\end{equation}
совпадающее до членов при $h^k$ включительно. При этом локальная погрешность
\begin{equation}
  \psi_{n + 1} = u_{n + 1} - u_n - h\sum\limits_{i = 0}^qA_i f_{n + \alpha_i}
\end{equation}
будет иметь вид
\begin{equation}
  \psi_{n + 1} = h^{k + 1}u_n^{(k + 1)}\left[\dfrac{1}{(k + 1)!} -
  \dfrac{1}{k!}\sum\limits_{i = 0}^qA_i d_i^k\right] + O(h^{k + 2})
\end{equation}

Поскольку весовая функция в формуле \eqref{18-quad-formula} в среднем равна $1$,
то квадратурная формула может быть построена единственным образом с НАСТ
$= 2q + 1, \forall q \geqslant 0$, то когда $k = 2q + 2$, то система
\eqref{18-system} имеет единственное решение, причём $0 \leqslant A_i \leqslant 1,
0 \leqslant \alpha_i \leqslant 1, i = \overline{0, q}$.

Для определения неизвестных величин $y_{n + \alpha_i}$ в \eqref{18-sum} можно
строить аналогичные методы
\begin{equation}
  y_{n + \alpha_i}^{[k]} = y_n + \alpha_i h\sum\limits_{j = 0}^{q_1}B_jf_n +
  \alpha_i\beta_j, q_1 \leqslant q
\end{equation}
Значение $y_{n + \alpha_i\beta_j}$ также определяется по аналогичным формулам
с погрешностью $O(h^{k - 1})$. Неизвестные параметры $\beta_j, B_j$ будут
определятся из систем, аналогичных \eqref{18-system}. С каждым шагом порядок
точности будет понижаться, т.e. $q \geqslant q_1 \geqslant \ldots \geqslant 1$.
\begin{equation}
  \sum\limits_{i = 0}^{q_1}B_i\beta_i^j = \dfrac{1}{j + 1}, j = \overline{0, k - 2}
\end{equation}

Завершать данную схему будут явные формулы Эйлера
\begin{equation}
  y_{n + \alpha_i\beta_j\cdot\ldots\cdot\gamma_m}^{[2]} = y_n^{[k + 1]} +
  \alpha_i\beta_j\cdot\ldots\cdot\gamma_m hf_n^{[k + 1]}
\end{equation}

Построим в качестве примера метод второго порядка точности, т.е. $k = 2$. Если
взять $q = 0$, то система \eqref{18-system} имеет единственное решение
$A_0 = 1, \alpha_0 = \dfrac{1}{2}$. Получаем формулу
\begin{align}
  y_{n + 1}^{[3]} = y_n^{[3]} + hf_{n + \frac{1}{2}}^{[2]}\\
  y_{n + 1}^{[2]} = y_n^{[3]} + \dfrac{h}{2}f_n^{[3]}\\
\end{align}
