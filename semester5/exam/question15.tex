\begin{col-answer-preambule}
	\begin{plan}
    \item пока пусто
	\end{plan}
\end{col-answer-preambule}

\colquestion{Метод Пикара и метод рядов Тейлора}

\subsection{Метод Пикара}

Рассмотрим задачу Коши для однородного дифференциального уравнения:
\begin{equation}
  \label{15-picar1}
  \begin{cases}
    &u'(x) = f(x, u), u = u(x), x \in [x_0, x_l]\\
    &u(x_0) = u_0
  \end{cases}
\end{equation}

Проинтегрируем  уравнение \eqref{15-picar1}
\begin{equation}
  \label{15-picar-integral}
  u(x) = u(x_0) + \int\limits_{x_0}^xf(t, u(t))dt
\end{equation}

$y$ - приближённое решение, $s$ - номер итерации.
\begin{equation}
  \label{15-picar-formula}
  \begin{cases}
    &y_s(x) = u_0 + \dint\limits_{x_0}^xf(t, y_{s - 1}(t))dt\\
    &y_0(t) = u_0
  \end{cases}
\end{equation}

Этот метод удобен, если интеграл можно вычислить аналитически. Докажем
сходимость метода Пикара.

Пусть в некоторой ограниченной области $G$ функция $f(x, u)$ непрерывная и
удовлетворяет условию Лившица по переменной $u$:
\begin{align}
  \abs{f(x_1, u_1) - f(x_1, u_2)} \leqslant L\abs{u_1 - u_2}\\
  \label{15-limits}
  \begin{cases}
    &\abs{x - x_0} \leqslant E, \forall x \in G\\
    &\abs{u - u_0} \leqslant V, E, V - \const
  \end{cases}
\end{align}

\eqref{15-limits} - условия ограниченности, выполняются в силу ограниченности
области $G$.

\begin{align}
  \eqref{15-picar-integral}, \eqref{15-picar-formula} \Rightarrow
  \abs{y_s(x) - u(x)} = \abs{\int\limits_{x_0}^x f(t, y_{s - 1}(t))dt -
                             \int\limits_{x_0}^x f(t, u(t))dt}\\
  \abs{y_s(x) - u(x)} \leqslant \int\limits_{x_0}^x
      \abs{f(t, y_{s - 1}(t)) - f(t, u(t))}dt
\end{align}

Обозначим $z_s(x) = y_s(x) - u(x)$ - погрешность в точке $x$.

\begin{equation}
  \abs{z_s(x)} \leqslant L\int\limits_{x_0}^x\abs{z_{s - 1}(t)}dt
\end{equation}

Если $s = 0$, то
\begin{equation}
  \label{15-error-formula}
  \begin{split}
    &\abs{z_0(x)} = \abs{u_0 - u(x)} \leqslant V \text{ - погрешность начального
    приближения}.\\
    &\abs{z_1(x)} \leqslant LV\abs{x - x_0}\\
    &\abs{z_2(x)} \leqslant \dfrac{1}{2}L^2V\abs{(x - x_0)^2}\\
    &\ldots\\
    &\abs{z_s(x)} \leqslant \dfrac{1}{s!}L^sV\abs{(x - x_0)^s}\\
  \end{split}
\end{equation}

Формула Стирлинга:
\begin{align*}
  n! \approx \dfrac{\sqrt{2\pi}n^{n + \frac{1}{2}}}{e^n}(1 + \varepsilon_n),
  \lim\limits_{n \to \infty}\varepsilon_n = 0
\end{align*}
\begin{equation}
  \eqref{15-error-formula} \Leftrightarrow \abs{z_s(x)} \leqslant
  \dfrac{1}{s!}L^sVE^s
\end{equation}

Используя формулу Стирлинга
\begin{equation}
  \label{15-error-approx}
  \abs{z_s(x)} \leqslant \dfrac{v}{\sqrt{2\pi s}}\left(\dfrac{eEL}{s}\right)^s
\end{equation}

$\eqref{15-error-approx} \Rightarrow \abs{z_s(x)} \xrightarrow[s \to \infty]{}0
\Rightarrow $ итерационный процесс сходится.

\subsection{Метод рядов Тейлора}

Рассмотрим
\begin{equation}
  \label{15-taylor-example}
  \begin{cases}
    &u' = f(x, u), x \in [x_0, x_l]\\
    &u(x_0) = u_0\\
  \end{cases}
\end{equation}

Продифференцируем \eqref{15-taylor-example} по $x$:
\begin{equation}
  \label{15-taylor-diff}
  \begin{split}
    &u'' = f_x + f_u\cdot u' = f_x + f\cdot f_u\\
    &u''' = f_{xx} + 2f_{xu}u' + f_{uu}u'^2 + f_u u''\\
    &\ldots
  \end{split}
\end{equation}

Подставим в формулу \eqref{15-taylor-diff} в качестве $x = x_0, u = u_0$,
последовательно находим значения $u'(x_0), u''(x_0), u'''(x_0)$ и т. д.
Получаем ряд Тейлора:
\begin{equation}
  u(x) \approx y_n(x) = \sum\limits_{i = 0}^n\dfrac{u^{(i)}(x_0)}{i!}\cdot
  (x - x_0)^i
\end{equation}

Если $\abs{x - x_0}$ не превышает радиуса сходимости ряда Тейлора, то
приближенное решение $y_n(x) \xrightarrow[n \to \infty]{} u(x)$.

Иногда полезно разбить исходный отрезок $[x_0, x_l]$ на $N$ частей
$[x_{j - 1}, x_j], j = \overline{1, N}, x_N = x_l$. Отрезки не обязательно
равные. На каждом отрезке применим метод рядов Тейлора для более точного
решения.

Рассмотрим произвольный отрезок $[x_j, x_{j + 1}]$. Будем считать, что $y_j$
найдено. Значит, мы можем найти $u^{(i)}(x_j)$. Тогда применяя метод рядов,
можно приблизить на этом отрезке

\begin{equation}
  \begin{split}
    &u(x) \approx v_j(x) = \sum\limits_{i = 0}^n\dfrac{u_j^{(i)}}{i!}(x - x_j)^i\\
    &y_{j + 1} = v_j(x_{j + 1})
  \end{split}
\end{equation}

При использовании метода рядов необходимо находить значения
$\approx \dfrac{n(n + 1)}{2}$ различных функций, поэтому на практике обычно
ограничиваются первым и вторым порядком точности (2-3 производные).
