\begin{col-answer-preambule}
	\begin{plan}
    \item пока пусто
	\end{plan}
\end{col-answer-preambule}

\colquestion{Усточивость линейных многошаговых методов}

Будем рассматривать задачу Коши
\begin{equation}
  \begin{cases}
    u' = f(x, u), x > 0\\
    u(0) = u_0
  \end{cases}
\end{equation}

Возьмём равномерную сетку узлов $\omega_h = \set{x_n = nh, n = 0, 1, \ldots}$.
Будем рассматривать линейный $m$-шаговый метод.
\begin{align}
  \label{22-multistep}
  &a_0y_n + a_1y_{n - 1} + \ldots + a_my_{n - m} = h\sum\limits_{k = 0}^mb_kf_{n-k}\\
  \nonumber
  &n = m, m + 1, \ldots
\end{align}

Коэфициенты $a_k, b_k = \const, a_0 \neq 0$. Для счёта по формуле \eqref{22-multistep}
необходимо задать $m$ начальных значений $y_0 = u_0, y_1, \ldots y_{m - 1}$. Обычно они
находятся с помощью одношагового метода Рунге-Кутты того же порядка точности, что и метод
\eqref{22-multistep}.

Запишем соответствующее однородное уравнение
\begin{align}
  \label{22-le}
  &a_0\delta_n + a_1\delta_{n - 1} + \ldots + a_m\delta_{n - m} = 0\\
  \nonumber
  &n = m, m + 1, \ldots
\end{align}

Будем искать частные решения \eqref{22-le} в виде $\delta_n = q^n, q = \const$, тогда для
опредления постоянной $q$ получим уравнение
\begin{equation}
  \label{22-char}
  a_0q^m + a_1q^{m - 1} + \ldots + a_m = 0
\end{equation}

\begin{definition}
  Уравнение \eqref{22-char} - характеристическое уравнение метода \eqref{22-multistep}.
\end{definition}

\begin{definition}
  \eqref{22-multistep} - линейный двухшаговый метод, удовлетворяет условию корней, если
  все корни $q_1, \ldots q_m$ характеристического уравнения \eqref{22-char} лежат внутри
  или на границе единичного круга комплексной плоскости. Причём на границе этого круга нет
  кратных корней.
\end{definition}

\begin{definition}
  Однородное уравнение \eqref{22-le} устойчиво по начальным данным, если $\exists \const
  M > 0$, независящая от номера узла $n$, такая, что при любых начальных данных
  $\delta_0, \ldots \delta_{m - 1}$ выполняется следующая оценка решения
\end{definition}
\begin{equation}
  \label{22-approx}
  \abs{\delta_n} \leqslant M \underset{0 \leqslant i \leqslant m - 1}{\max}\abs{\delta_i},
  n = m, m + 1, \ldots
\end{equation}

Таким образом устойчивость по начальным данным означает равномерную по $n$ ограниченность
решения задачи Коши.

\begin{theorem}
  Условие корней необходимо и достаточно для устойчивости метода \eqref{22-le} по
  начальным данным.
\end{theorem}
\begin{proof}

  \begin{enumerate}
    \item $\Rightarrow$. Пусть имеется корень $\abs{q} > 1$. Зададим начальные данные
    $\delta_i = q^i (i = \overline{0, m - 1})$. Тогда уравнение \eqref{22-le} имеет
    решение в точке $\delta_n = q^n(n \geqslant m)$, которое неограничено возрастает при
    $n \to \infty$. Оценка \eqref{22-approx} не выполняется.

    Следовательно, условие $\abs{q_k} \leqslant 1, k = \overline{1, m}$ - необходимое
    условие устойчивости. Пусть характеристическое уравнение \eqref{22-char} имеет корень
    $q$ с кратностью $r > 1$, причём этот корень находится на границе единичного круга на
    комплексной плоскости $\abs{q} = 1$. В этом случае однородное уравнение
    \eqref{22-le}
    имеет решение вида
    \begin{align*}
      \delta_n = q^n\cdot n^{r - 1}
    \end{align*}
    и оценка \eqref{22-approx} снова не выполняется.
    \item $\Leftarrow$. Без доказательства.
  \end{enumerate}
\end{proof}

Можно показать, что если уравнение \eqref{22-le} устойчиво по начальным данным, то для
неоднородного случая
\begin{equation}
  \label{22-nonuniform}
  a_0y_n + a_1y_{n - 1} + \ldots + a_my_{n - m} = hg_{n - m}, n = m, m + 1, \ldots
\end{equation}
выполняется оценка
\begin{equation}
  \abs{y_n} \leqslant M_1\underset{1 \leqslant j \leqslant m}{\abs{y_j}} +
  M_2\sum\limits_{k = 0}^{n - m}h\abs{g_k}
\end{equation}
которая означает устойчивость \eqref{22-nonuniform} по правой части и по начальным
данным.
