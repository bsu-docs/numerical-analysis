\documentclass[a4paper,12pt]{article}
\usepackage[utf8]{inputenc}
\usepackage[russian]{babel}
\usepackage{mathtools}
\usepackage{amssymb}
\usepackage{amsthm}
\usepackage{tikz}
\usepackage{multicol}
\usepackage{xparse}
\usepackage{enumitem}
\usepackage{centernot}
\usepackage{comment}
\usepackage{chngcntr}

\usepackage[left=1cm,right=1cm,top=1cm,bottom=2cm]{geometry}

\usepackage{setspace}
%\doublespace
\usepackage{epstopdf}
\usepackage{graphicx}
\usepackage{titlesec}

\newtheorem*{theorem}{Теорема}
\newtheorem*{lemma}{Лемма}

\newcommand{\norma}{}
\usepackage{mainstyle}
\renewcommand{\theenumii}{\asbuk{enumii}}

% For cyrillic symbols in "enumerate" environment.
\renewcommand{\theenumii}{\asbuk{enumii}}
\AddEnumerateCounter{\asbuk}{\@asbuk}{ы}

\counterwithin*{equation}{subsection}
%====================================================================================
%   ENVIRONMENTS

\newenvironment{definition}
{\begin{statement}{Определение}}
    {\end{statement}}

\newenvironment{characteristics}
{\begin{statementItemed}{Свойства}}
    {\end{statementItemed}}

\newenvironment{note}
{\begin{statementDotted}{Замечание}

        }
    {\end{statementDotted}}

\newenvironment{notes}
{\begin{statementItemed}{Замечания}}
    {\end{statementItemed}}

\newenvironment{proofUndotted}
{{\raggedright \textit{Доказательство}}}
{\begin{flushright}
       \boxed{}
 \end{flushright}
}

\newenvironment{noteo}{}{}
\RenewDocumentEnvironment{noteo}{o}
{{\raggedright \textbf{Замечание}\IfValueTF{#1}{ (\textit{#1})}{}.}$  $

}
{}

\newenvironment{consequence}{}{}
\RenewDocumentEnvironment{consequence}{o}
{{\raggedright \textbf{Следствие}\IfValueTF{#1}{ (\textit{#1})}{}.}$  $

    }
{}

\NewDocumentEnvironment{consequences}{o}
{\raggedright \textbf{Следствия}\IfValueTF{#1}{(\textit{#1})}{}:\begin{enumerate}}
    {\end{enumerate}}


\RenewDocumentEnvironment{theorem}{o}
{{\raggedright
  \textbf{Теорема}\IfValueTF{#1}{ (\textit{#1})}{}}.$  $

}

\NewDocumentEnvironment{characteristics_}{o}
{\raggedright \textbf{Свойства}\IfValueTF{#1}{(\textit{#1})}{}:\begin{enumerate}}
{\end{enumerate}}

\NewDocumentEnvironment{plan}{o}
{\raggedright \textbf{Краткий план}\IfValueTF{#1}{(\textit{#1})}{}:\begin{enumerate}}
	{\end{enumerate}}

\newenvironment{theoremNamed}{}{}
\RenewDocumentEnvironment{theoremNamed}{m o}
{{\raggedright
  \textbf{Теорема #1}\IfValueTF{#2}{ (\textit{#2})}{}}.$  $

}


\RenewDocumentEnvironment{lemma}{o}
{{\raggedright
        \textbf{Лемма}\IfValueTF{#1}{ (\textit{#1})}{}}.$  $

}

\newenvironment{lemmaNamed}{}{}
\RenewDocumentEnvironment{lemmaNamed}{m o}
{{\raggedright
        \textbf{Лемма #1}\IfValueTF{#2}{ (\textit{#2})}{}}.$  $

}


\newenvironment{exercise}
{\begin{statementDotted}{Упражнение}}
    {\end{statementDotted}}

\newenvironment{example}
{\begin{statementDotted}{Пример}}
    {\end{statementDotted}}

\newenvironment{examples}
{\begin{statementItemed}{Примеры}}
    {\end{statementItemed}}

% =================================
% math functions
\newcommand{\abs}[1]{
    \left\lvert #1 \right\rvert
}

\newcommand{\maxf}[1]{
    \max \left\{ #1 \right\}
}

\newcommand{\suchthat}{
    \;\ifnum\currentgrouptype=16 \middle\fi|\;
}

\newcommand{\arc}[1]{
    \buildrel\,\,\frown\over{#1}
}

\DeclareMathOperator{\const}{\text{const}}

\DeclareMathOperator{\fix}{\text{fix }}
\DeclareMathOperator{\diam}{diam\,}
\DeclareMathOperator{\mes}{mes}
\DeclareMathOperator{\divergence}{div}

\DeclareMathOperator{\arcsh}{arcsh}
\DeclareMathOperator{\arth}{{arth}}
\DeclareMathOperator{\arcth}{{arcth}}
\DeclareMathOperator{\sgn}{sgn}

\newcommand{\parenthesis}[1]{%
    \left( #1 \right)
}

\newcommand{\plot}[1]{$ \text{Г}_{#1} $}

\newcommand{\limlim}[2]
{ \lim\limits_{ \substack{ #1 \\ #2 } } }

\newcommand{\limitslimits}[2]
{ \limits_{ \substack{ #1 \\ #2 } } }

\newcommand{\limlimlim}[3]
{ \lim\limits_{ \substack{ #1 \\ #2 \\ #3 } } }

\newcommand{\nullFrac}{\dfrac{ }{}}

\renewcommand{\emptyset}{\varnothing}

\newcommand{\diint}{\displaystyle\iint}

\newcommand{\liml}{\lim\limits}
\newcommand{\intl}{\int\limits}
\newcommand{\iintl}{\iint\limits}
\newcommand{\diintl}{\diint\limits}
\newcommand{\dintl}{\dint\limits}

\newcommand{\suml}{\sum\limits}
\newcommand{\sumnzi}{\sum\limits_{n=0}^{\infty}}
\newcommand{\limninf}{\lim\limits_{n\to\infty}}
\newcommand{\dintlzi}{\dintl_0^{+\infty}}


%========================================================
% text style

\newcommand{\important}[1]{\textit{#1}}

\newcommand{\dint}{\displaystyle\int}
\newcommand{\dsum}{\displaystyle\sum}

\newcommand{\oiint}[2]{
    \begin{tikzpicture}[baseline=(C.base)]
        \node(C) {$ \displaystyle \iintl_{#1}^{#2} $};
        \draw (0,0.15) circle (0.25);
        %\node[draw,circle,inner sep=1pt](C) ++ (0, 0.1) {$ \;\;\;\; $};
    \end{tikzpicture}
}

\newcommand{\circled}[1]{
    \begin{tikzpicture}[baseline=(C.base)]
    \node[draw,circle,inner sep=1pt](C) {#1};
    \end{tikzpicture}
}

\newcommand{\eqlhopital}{%
    \overset{\circled{Л}}{=}}

\newcommand{\neqlhopital}{%
    \overset{\circled{Л}}{\neq}}

\newcommand{\sqcase}[1]{%
    \left[\begin{matrix}#1\end{matrix}\right]
}

\newcommand{\Arg}{%
  \operatorname{Arg}
}

\newcommand{\Ln}{%
  \operatorname{Ln}
}

\newcommand{\Arsh}{%
  \operatorname{Arsh}
}

\newcommand{\Arch}{%
  \operatorname{Arch}
}

\newcommand{\Arth}{%
  \operatorname{Arth}
}

\newcommand{\Arcth}{%
  \operatorname{Arcth}
}

\newcommand{\Arcsin}{%
  \operatorname{Arcsin}
}

\newcommand{\Arccos}{%
  \operatorname{Arccos}
}

\newcommand{\Arctg}{%
  \operatorname{Arctg}
}

\newcommand{\Arcctg}{%
  \operatorname{Arcctg}
}

\renewcommand{\th}{%
  \operatorname{th}
}

\renewcommand{\cth}{%
  \operatorname{cth}
}


\newcommand{\dvert}{\left.\nullFrac\right\vert}

\renewcommand{\r}[1]{$\overset{\text{ }_\bullet\text{ }}{\text{#1}}$}

\renewcommand{\norma}[1]{\left\lvert\left\lvert#1\right\rvert\right\rvert}
\newcommand{\norm}[1]{\left\lvert\left\lvert#1\right\rvert\right\rvert}

\newcommand{\RN}{\mathbb{R}^n}
\newcommand{\R}[1]{\mathbb{R}^{#1}}

% =================================
% sets utilities

\newcommand{\defineset}[2]{
    \left\{ #1 \, \middle\vert \, #2 \right\}
}

\newcommand{\set}[1]{
    \left\{ #1 \right\}
}

\newcommand{\colquestion}[1]{\section{#1}}
\newenvironment{col-answer-preambule}
               {\ignorespaces}
               {\ignorespacesafterend}

\begin{document}
\begin{center}
  \LARGE\underline{\textbf{Ответы к экзамену по курсу}}\\
  \LARGE\underline{\textbf{ ``Методы Численного анализа''}}\\
  \Large\textbf{(1-ый семестр 2016/2017 учебного года, специальность ``Информатикa'')}
\end{center}

{
  % \renewcommand{\contentsname}{Содержание}
  \tableofcontents
}
\newpage
\begin{col-answer-preambule}
	\begin{plan}
    \item пока пусто
	\end{plan}
\end{col-answer-preambule}

\colquestion{Интерполяционный многочлен Лагранжа. Оценка погрешности}

\textit{Интерполяционный многочлен Лагранжа.} \newline
Будем предполагать, что $f(x_i)$ известна в т. $x_i, i = \overline{0, n} \Rightarrow$ будем рассматривать интерполяционный многочлен $n$-ой степени. 

Надо построить $P_n(x)$, чтобы $P_n(x_i) = f(x_i), i = \overline{0, n}$.

Получаем СЛАУ, которую решать в лоб не будем. Коэффициенты $a_i$ будем искать в виде ЛК $f(x_i)$, т.е. будем искать многочлен $P_n(x)$ в виде:
\begin{equation*}
  P_n(x) = \sum\limits_{i = 0}^n l_i (x) f(x_i),
\end{equation*}
где $l_i (x)$ - многочлены $n$-ой степени.

Должно выполняться: $P_n (x_k) = f(x_k), k = \overline{0, n},$, подставляем в предыдущее равенство:
\begin{equation*}
  P_n (x_k) = f(x_k) = \sum\limits_{i = 0}^n l_i (x_k) f(x_i), k = \overline{0, n}.
\end{equation*}
Неравенство выше выполняется, если $l_i (x_k) = \begin{cases} 0, i \ne k, \\ 1, i = k. \end{cases}$

По теореме Виетта:
\begin{equation*}
  l_i(x) = (x - x_0) (x - x_1) \ldots (x - x_{i - 1}) (x - x_{i + 1}) \ldots (x - x_n) \cdot C_i.
\end{equation*}

$C_i$ найдем из условия $l_i (x_i) = 1$.

\begin{equation*}
  (x_i - x_0) (x_i - x_1) \ldots (x_i - x_{i - 1}) (x_i - x_{i + 1}) \ldots (x_i - x_n) \cdot C_i = 1.
\end{equation*}

\begin{equation*}
  C_i = \Big[(x_i - x_0) (x_i - x_1) \ldots (x_i - x_{i - 1}) (x_i - x_{i + 1}) \ldots (x_i - x_n)\Big]^{-1}.
\end{equation*}

Обозначим через $w(x) = (x - x_0) (x - x_1) \ldots (x - x_n)$.

Легко видеть, что $w^{'}(x_i) = \left.\dfrac{d w(x)}{dx}\right|_{x = x_i} = (x_i - x_0) (x_i - x_1) \ldots (x_i - x_{i - 1}) (x_i - x_{i + 1}) \ldots (x_i - x_n)$.

\begin{equation*}
  l_i(x) = \dfrac{w(x)}{(x - x_i) w^{'} (x_i)}
\end{equation*}

\begin{equation*}
  P_n(x) = \sum\limits_{i=0}^n \dfrac{w(x)}{(x - x_i) w^{'} (x_i)} f(x_i) \text{ - интерполяционный многочлен в форме Лагранжа.}
\end{equation*}

Какова же разность
\begin{equation*}
  f(x) - P_n(x), x \in [a, b], x \ne x_i \text{?}
\end{equation*}

\textit{Оценка погрешности.} \newline
Будем рассматривать функцию
\begin{equation*}
  \phi(z) = f(z) - P_n(z) - K w(z).
\end{equation*}

Постоянную $K$ выберем так, чтобы $\phi(x) = 0$.

Возможность такого выбора обусловлена следующим:
\begin{equation*}
  K = \dfrac{f(x) - P_n(x)}{w(x)},.
\end{equation*}
где деление возможно, т.к. $w(x) \ne 0, x \ne x_i$.

$\phi(z)$ на отрезке $[a, b]$ обращается в ноль $n + 2$ раза при всех $x_i, i = \overline{0, n}$ и кроме того в точке $x$.

Предположим, что $f(x)$ $n+1$ раз непрерывно-дифференцируема на $[a, b]$. Тогда из теоремы Ролля:

\begin{theorem}[Ролля]
  Если вещественная функция, непрерывная на отрезке $[ a , b ]$ и дифференцируемая на интервале $(a,b)$, принимает на концах отрезка $[a,b]$ одинаковые значения, то на интервале $(a, b)$ найдётся хотя бы одна точка, в которой производная функции равна нулю.
\end{theorem} $\Rightarrow$
\begin{enumerate}
  \item $\phi^{'}(z)$ обращается в ноль $n+1$ раз,
  \item $\phi^{''}(z)$ обращается в ноль $n$ раз,
  \item \ldots,
  \item $\phi^{(n+1)}(z)$ обращается в ноль по крайней мере 1 раз, т.е. $\exists \; \xi \in [a,b] \text{ | } \phi^{(n+1)}(\xi) = 0$.
\end{enumerate}

\begin{equation*}
  \phi^{(n+1)}(z) = f^{(n+1)}(z) - K (n + 1)! \Rightarrow K = \dfrac{f^{(n+1)}(\xi)}{(n+1)!}.
\end{equation*}
\begin{equation*}
  0 = f(x) - P_n(x) - \dfrac{f^{(n+1)}(\xi)}{(n+1)!} \cdot w(x) \Rightarrow f(x) - P_n(x) = \dfrac{f^{(n+1)}(\xi)}{(n+1)!} \cdot w(x).
\end{equation*}

\newpage
\begin{col-answer-preambule}
	\begin{plan}
    \item пока пусто
	\end{plan}
\end{col-answer-preambule}

\colquestion{Оценка погрешности на равномерной сетке узлов}

\begin{lemma}
  На равномерной сетке узлов
  \begin{equation*}
    w_n = \{ x_i = a + ih \text{ : } i = \overline{0, n}, h = \dfrac{b - a}{n} \}
  \end{equation*}
  $\forall \; x \in [a, b]$ выполняется оценка
  \begin{equation*}
    \prod\limits_{i = 0}^{n} \left| x - x_i \right| \leqslant \dfrac{1}{4} h ^{n + 1} n!
  \end{equation*}
\end{lemma}
\begin{proof}
  Зафиксируем $x$ и выберем индекс $j$ такой, что $x_j \leqslant x \leqslant x_{j + 1}$. Тогда справедливо неравенство:
  \begin{equation*}
    \left|x - x_j \right| \left| x - x_{j + 1} \right| \leqslant \dfrac{h^2}{4} \text{ (док-ть самостоятельно)}.
  \end{equation*}
  Используя неравенство выше, находим оценку:
  \begin{equation*}
    \prod\limits_{i = 0}^{n} \left| x - x_i \right| \leqslant \dfrac{h^2}{4} \prod\limits_{i = 0}^{j-1} (x - x_i) \prod_{i = j + 2}^{n} (x_i - x).
  \end{equation*}

  Т.к. сетка равномерная, то $x_i = a + ih, x_{j+1} = a + (j+1)h \Rightarrow$
  \begin{equation*}
    x_{j + 1} - x_{i} = (j - i + 1) h, 
  \end{equation*}
  \begin{equation*}
    x_i - x_j = (i - j) h.
  \end{equation*}
  \begin{equation*}
    \prod\limits_{i = 0}^n \left| x - x_i \right| \leqslant \dfrac{h^2}{4} \prod\limits_{i = 0}^{j-1} (x_{j + 1} - x_i) \prod_{i = j + 2}^{n} (x_i - x_j)
  \end{equation*}
  \begin{equation*}
    \prod\limits_{i = 0}^n \left| x - x_i \right| \leqslant \dfrac{h^2}{4} h^j h^{n - j - 1} \prod\limits_{i = 0}^{j-1} (j - i + 1) \prod_{i = j + 2}^{n} (i - j) \leqslant \dfrac{1}{4} h^{n + 1} (j+1)! (n-j)!
  \end{equation*}
  \begin{equation*}
    (j + 1)! (n - j)! \leqslant n!, 0 \leqslant j \leqslant n - 1  \text{ (док-ть самостоятельно)}.
  \end{equation*}
  \begin{equation*}
    \prod\limits_{i = 0}^n \left| x - x_i \right| \leqslant \dfrac{1}{4} h ^{n + 1} n!
  \end{equation*}
\end{proof}

\begin{theorem}
  Пусть функция $f \in \mathbb{C}^{n + 1}\left( [a, b] \right)$ и $\left| f^{(n+1)}\right| \leqslant M$. Если $P_n(x)$ - интерполяционный полином степени не выше $n$ на равномерной сетке узлов, то погрешность
  \begin{equation*}
    \left| f(x) - P_n(x) \right| \leqslant \dfrac{1}{4 (n + 1)} M h^{n+1}.
  \end{equation*}
\end{theorem}
\begin{proof}
  $\Rightarrow$ из оценки погрешности интерполяционного полинома Лагранжа
  \begin{equation*}
    f - P_n = \dfrac{1}{(n+1)!} f^{(n+1)} (\xi) w(x), \text{ где } w(x) = (x - x_0) \ldots (x - x_n)
  \end{equation*}
  и доказанной леммы.
\end{proof}

Из доказанной теоремы следует, что если сетка равномерна, то мы легко можем оценить погрешность интерполяционного полинома.

\newpage
\begin{col-answer-preambule}
	\begin{plan}
    \item пока пусто
	\end{plan}
\end{col-answer-preambule}

\colquestion{Разделённые разности и их свойства}

Разделенные разности нулевого порядка совпадают со значениями функциями $f(x_i) = f_i$. Разделенная разность первого порядка записывается как
\begin{equation*}
  f(x_i, x_j) = \dfrac{f_j - f_i}{x_j - x_i}.
\end{equation*}

Разделенная разность второго порядка:
\begin{equation*}
  f(x_i, x_j, x_k) = \dfrac{f(x_j, x_k) - f(x_i, x_j)}{x_k - x_i}.
\end{equation*}

Разделенная разность $k$-ого порядка:
\begin{equation*}
  f(x_1, x_2, \ldots, x_{k+1}) = \dfrac{f(x_2, \ldots, x_{k+1}) - f(x_1, \ldots, x_k)}{x_{k+1} - x_1}.
\end{equation*}

\begin{lemma}
  Разделенную разность можно представить в виде суммы
  \begin{equation*}
    f(x_1, \ldots, x_n) = \sum\limits_{j=1}^n \dfrac{f_j}{w^{'}_n (x_j)},
  \end{equation*}
  где $w(x) = (x - x_1) \ldots (x - x_n)$.
\end{lemma}
\begin{proof}
ММИ.
  \begin{enumerate}
    \item $n = 1$. $f_1 = f_1$.
    \item $n = 2$. $f(x_1, x_2) = \dfrac{f_2 - f_1}{x_2 - x_1} = \dfrac{f_1}{x_1 - x_2} + \dfrac{f_2}{x_2 - x_1}$.
    \item $n = k$. $k-1$ порядка:
    \begin{equation*}
      f(x_1, \ldots, x_k) = \sum\limits_{j=1}^k \dfrac{f_j}{w^{'}_k(x_j)}
    \end{equation*}
    \begin{equation*}
      f(x_2, \ldots, x_{k+1}) = \sum\limits_{j=2}^{k+1} \dfrac{f_j}{w^{'}_k(x_j)}
    \end{equation*}
    
    \item докажем для $n = k + 1$. $k$-ый порядок:
    \begin{equation*}
      f(x_1, \ldots, x_{k+1}) = \dfrac{f(x_2, \ldots, x_{k+1}) - f(x_1, \ldots, x_k)}{x_{k+1} - x_k} = \dfrac{1}{x_1 - x_{k + 1}} \sum\limits_{j=1}^k \dfrac{f_j}{w^{'}(x_j)} + \dfrac{1}{x_{k + 1} - x_1} \sum\limits_{j=2}^{k+1} \dfrac{f_j}{w^{'}(x_j)} =  \sum\limits_{j=1}^{k+1} \dfrac{f_j}{w^{'}_{k+1}(x_j)}.
    \end{equation*}
    Найдем коэффициент при $f_j$:
    \begin{equation*}
      \dfrac{1}{x_{k+1} - x_1} \left( \dfrac{1}{\prod\limits_{i = 2, i \ne j}^{k+1} (x_j - x_i)} - \dfrac{1}{\prod\limits_{i = 1, i \ne j}^{k} (x_j - x_i)} \right) = \dfrac{1}{x_{k+1} - x_1} \cdot \dfrac{1}{\prod\limits_{i = 1, i \ne j}^{k+1} (x_j - x_i)} \left( (x_j - x_i) - (x_j - x_{k+1}) \right) = 
    \end{equation*}
    \begin{equation*}
      = \dfrac{1}{\prod\limits_{i = 1, i \ne j}^{k+1} (x_j - x_i)} = \dfrac{1}{w^{'}_{k+1}(x_j)}
    \end{equation*}
  \end{enumerate}
\end{proof}
\begin{consequences}
  \item Разделенная разность - линейный оператор функции $f$:
  \begin{equation*}
    \left( \alpha_1 f_1(x) + \alpha_2 f_2 (x) \right) (x_1, x_2, \ldots, x_n) = \alpha_1 f_1(x_1, \ldots, x_n) + \alpha_2 f_2(x_1, \ldots, x_n).
  \end{equation*}
  \item Разделенная разность симметрична относительно своих аргументов, т.е. не меняется от перестановки $\forall \; x_i, x_j$.
  \item Разделенные разности удобно записывать в виде таблицы:
  \begin{center}
    \begin{tabular}{ | c | c | c | c |}
      \hline
      $x_1$ & $f_1$ &              &                \\ \hline
            &       & $f(x_1, x_2)$ &                  \\ \hline
      $x_2$ & $f_2$ &               & $f(x_1, x_2, x_3)$ \\ \hline
            &       & $f(x_2, x_3)$ &                   \\ \hline
      $x_3$ & $f_3$ &               & $f(x_2, x_3, x_4)$\\ \hline
            &       & $f(x_3, x_4)$ &                   \\ \hline
      $x_4$ & $f_4$ &               &                 \\
        \hline
    \end{tabular}
  \end{center}
\end{consequences}

\newpage
\begin{col-answer-preambule}
	\begin{plan}
    \item пока пусто
	\end{plan}
\end{col-answer-preambule}

\colquestion{Интерполяционный многочлен Ньютона}

Интерполяционный многочлен Ньютона представляет собой другую форму записи интерполяционного многочлена.

Она полезна, т.к. позволяет легко увеличивать или уменьшать число использованных узлов без повторных числений.

Пусть есть набор точек $(x_i, f_i), i = \overline{0, n}$. Построим интерполяционный многочлен для этой сеточной функции:
\begin{equation*}
  P_n(x) = f_0 + (x - x_0) P_{n - 1}(x) \Rightarrow P_{n-1} (x) = \dfrac{P_n(x) - f_0}{x - x_0},
\end{equation*}
где $x_1, \ldots, x_n$ в качестве $x$. Для задания $P_{n-1}(x)$ нужно $n$ коэффициентов = $n$ уравений.
\begin{equation*}
  P_{n-1}(x_i) = \dfrac{P_n(x_i) - f_0}{x_i - x_0} = \dfrac{f_i - f_0}{x_i - x_0}, i = \overline{1, n}.
\end{equation*}
Значит, искомый полином проходит через точки $\left( x_i, \dfrac{f_i - f_0}{x_i - x_0} \right)$, где $\dfrac{f_i - f_0}{x_i - x_0}$ - разделенная разность первого порядка.

Обозначим $\dfrac{f_i - f_0}{x_i - x_0} = f(x_0, x_i)$.

Значит, $P_{n-1}(x)$ можно представить в виде:
\begin{equation*}
  P_{n-1}(x) = f(x_0, x_1) + (x - x_1)P_{n-2}(x).
\end{equation*}

Для нахождения $P_{n-2} (x)$ проделаем те же действия, что и для $P_{n-1} (x):$

\begin{equation*}
  P_{n-2}(x) = \dfrac{P_{n-1}(x) - f(x_0, x_1)}{x - x_1}
\end{equation*}

\begin{equation*}
  \dfrac{f(x_1, x_i) - f(x_0, x_1)}{x_i - x_1} = f(x_0, x_1, x_i), i = \overline{2, n}.
\end{equation*}

\begin{equation}\label{eq:begin_table}
  P_n(x) = f_0 + (x - x_0) f(x_0, x_1) + (x - x_0)(x - x_1)f(x_0, x_1, x_2) + \ldots + (x-x_0)(x-x_1)\ldots(x-x_{n-1})f(x_0, x_1,\ldots, x_n)
\end{equation}

Погрешность интерационного многочлена через разделенные разности:
\begin{equation*}
  f(x) - P_n(x) = f(x) - \sum\limits_{i=0}^n f_i \prod\limits_{j=0, j \ne i}^n \dfrac{x - x_j}{x_i - x_j} = \underbrace{\prod\limits_{j=0}^n (x - x_j)}_{w(x)} \underbrace{\left[ \dfrac{f(x)}{\prod\limits_{j=0}^n (x - x_j)} + \sum\limits_{i=0}^n \dfrac{f_i}{(x_i - x)\prod\limits_{j=0, j \ne i}^n (x_i - x_j)}  \right]}_{f(x, x_0, x_1, \ldots, x_n)} =
\end{equation*}
\begin{equation*}
  = \left[ \text{для } f(x, x_0, x_1, \ldots, x_n) \text{ см Лемму о представлении РР в виде суммы в пред вопросе} \right] = 
\end{equation*}
\begin{equation*}
  = w(x) f(x, x_0, x_1, \ldots, x_n) = \dfrac{1}{(n+1)!} f^{(n+1)} (\xi) w(x).
\end{equation*}
Отсюда получаем, что $f(x, x_0, x_1, \ldots, x_n) = \dfrac{f^{(n+1)} (\xi)}{(n+1)!}$.

\begin{notes}
  \item При построении формулы Ньютона порядок расположения узолв $x_0, \ldots, x_n$ значения не имеет. В качестве точки $x_0$ мы можем взять и точу $x_n$. Пусть $x_0 := x_n$. Тогда:
  \begin{equation}\label{eq:end_table}
    P_n(x) = f_n + (x - x_n) f(x_n, x_{n-1}) + (x - x_n) (x - x_{n-1}) f(x_n, x_{n-1}, x_{n-2}) + \ldots + (x - x_n)(x - x_{n-1}) \cdot \ldots \cdot (x - x_1)f(x_n, \ldots, x_0).
  \end{equation}
  Если узлы упорядочены по возрастанию $x_0 < x_1 < \ldots < x_n$, то формула записи \eqref{eq:begin_table} носит название формула записи Ньютона для начала таблицы, а \eqref{eq:end_table} - форма записи Ньютона для конца таблицы.
  \item Хотя теоретически нет необходимости упорядочивать множество узлов $\{ x_i \} $ по возрастанию и убыванию, гладкость в таблице разделенных разностей нарушается, если такого порядка нет. Поэтому, программируя, используется упорядоченное множество сетки.
\end{notes}

\newpage
\begin{col-answer-preambule}
	\begin{plan}
    \item пока пусто
	\end{plan}
\end{col-answer-preambule}

\colquestion{Конечные разности и их свойства}

\begin{notes}
  \item пока пусто
\end{notes}

\newpage
\begin{col-answer-preambule}
	\begin{plan}
    \item пока пусто
	\end{plan}
\end{col-answer-preambule}

\colquestion{Интерполяционный многочлен Ньютона на равномерной сетке узлов}

Если функция задана на равномерной сетке узлов, это означает, что:
\begin{equation*}
  \begin{cases}
    \Delta x_i = x_{i + 1} - x_i = h \forall \; i, \\
    \Delta f_i = f_{i+1} - f_i.
  \end{cases}
\end{equation*}
\begin{equation*}
  \Delta^2 f_i = f_{i+2} - 2 f_{i+1} + f_i.
\end{equation*}
\begin{equation*}
  f(x_i) = f_i.
\end{equation*}

Эти конечные разности соответствуют соответствующим разделенным разностям:
\begin{equation*}
  \underbrace{f(x_1, x_2)}_{\text{разделенная разность}} = [x_2, x_1] = \dfrac{f_2 - f_1}{x_2 - x_1} = \dfrac{f_2 - f_1}{h} = \underbrace{\dfrac{1}{h} \Delta f_1}_{\text{конечная разность}}.
\end{equation*}

Разделенная разность второго порядка:
\begin{equation*}
  f(x_1, x_2, x_3) = [x_3, x_2, x_1] = \dfrac{1}{x_3 - x_1} \Big( f(x_3, x_2) - f(x_1, x_2) \Big) = \dfrac{1}{2h} \left(\dfrac{\Delta f_2}{h} - \dfrac{\Delta f_1}{h} \right) = \dfrac{\Delta^2 f_1}{2! h^2}.
\end{equation*}

\begin{equation*}
  f(x_1, x_2, \ldots, x_n) = \dfrac{\Delta^{n-1} f}{(n-1)! h^{n-1}}.
\end{equation*}

Конечные разности логично использовать для аппроксимации конечных производных.

\begin{enumerate}
  \item $\Delta f_1 \sim h f^{'} \left(x_1 + \dfrac{h}{2} \right),$
  \item $\Delta^2 f_1 \sim h^2 f^{(2)} (x_1 + h)$,
  \item $\ldots$,
  \item $\Delta^n f_1 \sim h^n f^{(n)} (x_1 + \dfrac{n h}{2}).$
\end{enumerate}

Формула Ньютона для начала таблицы для равномерной сетки узлов имеет вид:
\begin{equation*}
  P_n(x) = f_0 + (x - x_0) \dfrac{\Delta f_0}{h} + \ldots + (x - x_0)(x - x_0 - h) (x - x_0 - (n-1)h) \dfrac{\Delta^n f_0}{n! h^n}.
\end{equation*}
\begin{equation*}
  t = \dfrac{x - x_0}{h} \Rightarrow x - x_0 = th, x - x_1 = x - x_0 - h = h(t-1).
\end{equation*}
\begin{equation*}
  P_n(x) = P_n (x_0 + th) = f_0 + \dfrac{t}{1!} \Delta f_0 + \dfrac{t (t-1)}{2!} \Delta^2 f_0 + \ldots + \dfrac{t (t-1) \ldots (t - n + 1)}{n!} \Delta^n f_0.
\end{equation*}
\begin{equation*}
  R_n(x) = h^{n+1} \dfrac{t(t-1)\ldots(t-n)}{(n+1)!} f^{(n+1)} (\xi).
\end{equation*}

Запишем представления раздельных разностей в конце таблицы:
$\begin{cases}
  f(x_n, x_{n-1}) = \dfrac{\Delta f_{n-1}}{1! n}, \\
  f(x_n, x_{n-1}, x_{n-2}) = \dfrac{\Delta^2 f_{n-2}}{2! h^2}, \\
  f(x_n, \ldots, x_0) = \dfrac{\Delta^n f_0}{n! h^n}.
\end{cases}$

\begin{equation*}
  t = \dfrac{x - x_n}{h},
\end{equation*}

\begin{equation*}
  P_n(x_n + th) = f_n + \dfrac{t}{1!} \Delta f_{n-1} + \dfrac{t (t + 1)}{2!} \Delta^2 f_{n-2} + \ldots + \dfrac{t (t + 1) \ldots (t + n - 1)}{n!} \Delta^n f_0,
\end{equation*}

\begin{equation*}
  R_n(x) = h^{n+1} \dfrac{t(t+1)\ldots (t+n)}{(n+1)!} f^{(n+1)} (\xi).
\end{equation*}

\newpage
\begin{col-answer-preambule}
	\begin{plan}
    \item пока пусто
	\end{plan}
\end{col-answer-preambule}

\colquestion{Многочлен Чебышева}

В классе ортогональных функций ортогональные многочлены имеют ряд свойств.

Пусть $f(x), g(x)$ - полиномы.

$(f, g) = 0$ и $\int\limits_a^b f(x) g(x) dx = 0$ - полиномы ортогональны.

\begin{characteristics_}
  \item Ортогональные многочлены удовлетворяют трехчленному рекуррентному соотношению.
  \item Их легко вычислить и превращать в степенной ряд.
  \item Их нули разделяют друг друга (чередование нулей).
\end{characteristics_}

Многочлены Чебышева обладают как свойствами рядов Фурье, так и ортогональных полиномов.

В сущности они являются функциями Фурье.

\begin{equation*}
  T_n(x) = \cos \Big(n \Theta(x)\Big), \Theta(x) = \arccos(x).
\end{equation*}

Покажем, что $T_n(x)$ является многочленом

\begin{equation}\label{eq:cheb_2}
  T_0(x) = 1, T_1(x) = x, x \in [-1, 1].
\end{equation}

\begin{equation*}
  \cos \Big((n+1) \Theta\Big) + \cos \Big((n-1) \Theta\Big) = 2 \cos (\Theta) \cos \Big(n \Theta\Big).
\end{equation*}

\begin{equation*}
  \Theta = \arccos x.
\end{equation*}

\begin{equation}\label{eq:cheb_4}
  T_{n+1}(x) = 2x T_n - T_{n-1}(x), x \in [-1, 1].
\end{equation}

Равенство выше называется трехчленной рекуррентной формулой.

Корни полинома Чебышева:
\begin{equation}\label{eq:set_5}
  x_k = \cos \left( \dfrac{\pi \left(k + \frac{1}{2}\right)}{n} \right), k = \overline{0, n - 1}.
\end{equation}

Покажем, что полиномы Чебышева $T_l (x), l = \overline{0, n - 1}$ на множестве $\{ x_k \}$ из \eqref{eq:set_5} являются ортогональными.

\begin{equation*}
  \left( T_l, T_m \right) = \sum\limits_{k = 0}^{n-1} T_l (x_k) T_m (x_k) = \sum\limits_{k = 0}^{n-1} \cos \left( \dfrac{\pi l}{n} \left(k + \frac{1}{2}\right) \right) \cos \left( \dfrac{\pi m}{n} \left(k + \frac{1}{2}\right) \right) =
\end{equation*}

\begin{equation*}
  = \sum\limits_{k = 0}^{n-1} \left( \dfrac{1}{2} \cos\left( \dfrac{\pi (l - m)}{n} \left(k + \frac{1}{2}\right) \right)
  + \dfrac{1}{2} \cos\left( \dfrac{\pi (l + m)}{n} \left(k + \frac{1}{2}\right) \right)  \right) = \delta_{lm} M_l,
\end{equation*}
где $\delta_{ij}$ - символ Кронекера, т.е. $ \delta_{ij} = \begin{cases} 1, i = j, \\ 0, i \ne j. \end{cases}$
\begin{equation*}
  M_l = (1 + \delta_{lm}) \dfrac{\pi}{2}.
\end{equation*}

\begin{equation*}
  (T_l, T_m) = 0 \; \forall \; l \ne m.
\end{equation*}

Если функция $f(x)$ задана на множестве узлов $\{ x_k \}$ по формуле \eqref{eq:set_5}, то можно построить интерполяционный полином.

\begin{equation}\label{eq:cheb_8}
  P_{n-1}(x) = \sum\limits_{l = 0}^{n-1} C_l T_l,
\end{equation}
где $C_l = \sum\limits_{k = 0}^{n-1} \dfrac{f_k T_l (x_k)}{M_l}$.

\begin{proof}
  \begin{equation*}
    \left( P_{n-1}(x), T_m \right) = \sum\limits_{k = 0}^{n-1} P_{n-1}(x_k) T_m (x_k) = \sum\limits_{k = 0}^{n-1} f_k T_m(x_k)
  \end{equation*}
  \begin{equation*}
    \left( P_{n-1}(x), T_m \right) =( \sum\limits_{l = 0}^{n-1} C_l T_l (x_l), T_m ) = \sum\limits_{l = 0}^{n-1} C_l (T_l, T_m) = C_m M_m.
  \end{equation*}
  \begin{equation*}
	C_m = \sum\limits_{k = 0}^{n-1} \dfrac{f_k T_m (x_k)}{M_m}
  \end{equation*}
\end{proof}

Из \eqref{eq:cheb_2}, \eqref{eq:cheb_4} следует, что
\begin{equation*}
  T_n (x) = 2^{n-1} x^n + \ldots (n \geqslant 1).
\end{equation*}

\begin{equation*}
  \left| \cos \Big( n \arccos x \Big) \right| = 1.
\end{equation*}

\begin{equation*}
  n \arccos x = \pi m, m = \overline{0, n}, x \in [-1, 1].
\end{equation*}

Отсюда находим $x_m = \cos \left( \dfrac{\pi m}{n} \right)$.

\begin{equation*}
  T_n (x_m) = \cos \left( n \arccos \dfrac{\pi m}{n} \right) = \cos \Big( \pi m \Big) = (-1)^m.
\end{equation*}

\newpage
\begin{col-answer-preambule}
	\begin{plan}
    \item пока пусто
	\end{plan}
\end{col-answer-preambule}

\colquestion{Минимизация остатка интерполирования}

\textit{Критерий}.\newline
Чебышев показал, что из всех полиномов степени $n$, $P_n(x)$ со старшим коэффициентом равным $1$, у полинома
\begin{equation*}
  2^{1-n} T_n(x) = \overline{T}(x).
\end{equation*}

точная верхняя грань абсолютных значений на $[-1, 1]$ наименьшая и равна $2^{1-n}$, т.к. $\max\limits_{[-1, 1]} |T_n(x)| = 1$.

$\left|\left| \overline{T_n} - 0 \right| \right|_{\infty, [-1, 1]} = 2^{1-n} \Rightarrow \overline{T_n} (x)$ - полином, наименее отклоняющийся от нуля.

В случае отрезка произвольной длины $x \in [a, b]$ сделаем линейную замену переменных, которая отображает $[a, b]$ на $[-1, 1]$.

\begin{equation}\label{eq:8_2}
  x = \dfrac{b-a}{2} t + \dfrac{a+b}{2} = \psi(t), t \in [-1, 1].
\end{equation}

\begin{equation*}
  P_n(x) = x^n + P_{n-1}(x) = \psi^n (x) + P_{n-1} \Big( \psi(t) \Big) = \left(\dfrac{b-a}{2}\right)^n \overline{P_n}(t).
\end{equation*}

\begin{equation*}
  \left|\left| P_n(x) \right| \right|_{\infty, [a, b]} = \left(\dfrac{b-a}{2}\right)^n \left|\left| \overline{P_n}(t) \right| \right|_{\infty, [-1, 1]} \geqslant 2^{1-n} \left( \dfrac{b-a}{2} \right)^n = (b-a)^n 2^{1-2n}.
\end{equation*}

Равенство в этой формуле достигается при
\begin{equation*}
  \overline{T_n}\overset{[a, b]}{(x)} = (b - a)^n 2^{1-2n} T_n \left( \dfrac{2x - a - b}{b - a} \right).
\end{equation*}

$\overline{T_n}\overset{[a, b]}{(x)}$ называется наименее отклоняющимся от нуля на отрезке $[a, b]$.

В силу замены переменных \eqref{eq:8_2} корни $\overline{T_n}\overset{[a, b]}{(x)}$ находятся по формуле:
\begin{equation*}
  x_m = \dfrac{a+b}{2} + \dfrac{b-a}{2} \cos \Big( \dfrac{\pi (m + \frac{1}{2})}{n} \Big), m = \overline{0, n-1}.
\end{equation*}

Для оценки остатка интерполирования функции $f(x)$ на Чебышевской сетке узов $\{ x_m \}$ будем использовать равномерную норму:
\begin{equation*}
  \left| \left| f(x) \right| \right|_{\infty} = \sup\limits_{[a, b]} |f(x)|.
\end{equation*}

Из представления остатка в общем виде следует
\begin{equation*}
  \left| \left| f(x) - P_{n-1}(x) \right| \right| \leqslant \dfrac{1}{n!} \left| \left| f^{(n)}(x) \right| \right|  \left| \left| w_n(x) \right| \right|.
\end{equation*}

Будем минимизировать правую часть в неравенстве выше.
\begin{equation*}
  w_n = (x - x_1) \ldots (x - x_n), \text{deg }w_n = n, \text{ старший коэффициент равен 1}.
\end{equation*}

Поэтому в качестве узлов интерполирования $x_1, \ldots, x_n$ мы можем взять корни многочлена Чебышева. В этом случае $w_n$ будет иметь вид:

\begin{equation*}
  w_n = \overline{T_n}\overset{[a, b]}{(x)} = (b - a)^n 2^{1-2n} T_n \left( \dfrac{2x - a - b}{b - a} \right).
\end{equation*}

Из равенства выше следует что $|| w_n (x) || = (b - a)^n 2^{1-2n} \Rightarrow$ на Чебышевском наборе узлов оценка погрешности интерполирования имеет вид:
\begin{equation*}
  \left| \left| f(x) - P_{n}(x) \right| \right| \leqslant \dfrac{1}{n!} \left| \left| f^{(n)}(x) \right| \right| (b - a)^n 2^{1-2n}.
\end{equation*}

Мы получили неуменьшаемую оценку погрешности интерполяции.

\newpage
\begin{col-answer-preambule}
	\begin{plan}
    \item пока пусто
	\end{plan}
\end{col-answer-preambule}

\colquestion{Интерполирование с кратными узлами}

Задача кратного интерполирования заключается в построении полинома наименьшей степени $n$, который удовлетворяет условиям:
\begin{equation}\label{eq:9_1}
  H_m^{(j)}(x_i) = f^{(j)}(x_i) = f_i^{(j)}, i = \overline{0, n}, j = \overline{0, m - 1}.
\end{equation}

Все узлы $x_i$ будем считать различнымми. Через $m_i$ будем обозначать кратность $i$-того узла.

Полином $H_m(x)$, удовлетворяющий условию \eqref{eq:9_1}, называется полиномом Эрмита.

Число заданных условий в \eqref{eq:9_1} равно:
\begin{equation*}
  \sum\limits_{i=0}^n m_i = m + 1.
\end{equation*}

\begin{theorem}
  Интерполяционный полином $H_m(x)$, удолетворяющий условию \eqref{eq:9_1}, определяется единственным образом.
\end{theorem}

\begin{theorem}
  Полином Эрмита $H_m(x)$, определенный условиями \eqref{eq:9_1}, существует.
\end{theorem}

Рассмотрим погрешность интерполирования $R_m(x) = f(x) - H_m(x)$.

\begin{theorem}
  Пусть узлы $x_i, i = \overline{0, n}$ и точка $x \in [a, b], f(x) \in \mathbb{C}^{m + 1} [a, b]$.

  Тогда
  \begin{equation*}
    \exists \; \overline{x} \in [a, b] \; | \; R_m(x) = \dfrac{w_{m+1}(x)}{(m+1)!} f^{(m+1)}(\overline{x}),
  \end{equation*}
  где $w_{m+1}(x) = (x - x_0)^{m_0} \ldots (x - x_n)^{m_n}$.
\end{theorem}

Построим $H_m(x)$ на основе расширенной таблицы разделенных разностей. Введем набор узлов
\begin{equation*}
  x_{ij} = x_i + j \varepsilon, \varepsilon > 0, i = \overline{0, n}, j = \overline{0, m_i - 1}.
\end{equation*}

Очевидно, что все $x_{ij}$ различны и стремятся к $x_i$ при $\varepsilon \to 0$ (по построению).

Таблица разделенных разностей для расширенного набора узлов будет иметь вид:

\begin{center}
  \begin{tabular}{ | c | c | c | c |}
    \hline
    $f(x_{00})$          &                       &              &                \\ \hline
                        & $f(x_{00}, x_{01})   $ &              &                \\ \hline
    $f(x_{01})$          &                       &$f(x_{00}, x_{01}, x_{02})   $ &                  \\ \hline
                        & $f(x_{01}, x_{02} )  $ &              &                \\ \hline
    $\ldots$            &                 &               &  $f(x_{00}, \ldots, x_{n m_n - 1}   $  \\ \hline
    $f(x_{0 m_0-1})$     &       &  &                   \\ \hline
    $f(x_{1 0})$         &  &               & \\ \hline
    $f(x_{1 m_1-1})$     &       &  &                   \\ \hline
    $f(x_{x_n m_n - 1})$ &  &               &                 \\
      \hline
  \end{tabular}
\end{center}

Выражая разделенные разности через производные и переходя к пределу при $\varepsilon \to 0$, получаем:
\begin{equation*}
  \lim\limits_{\varepsilon \to 0} f(x_{il}, \ldots, x_{ik}) = \dfrac{f^{(k-l)} (x_i)}{(k-l)!}.
\end{equation*}

\newpage
\begin{col-answer-preambule}
	\begin{plan}
    \item пока пусто
	\end{plan}
\end{col-answer-preambule}

\colquestion{Интерполяционный сплайн второго порядка}

Поскольку интерпретация многочленами высокой степени не желательна, то в случае большого числа узлов сетки используют кусочно-полиномиальную интерполяцию. В этом случае исходный отрезок (интервал) разбивают на подынтервалы, затем на каждом подынтервале строится интерполяционный полином второй или третьей степени. Недостаток такого подхода состоит в том, что производная аппроксимирующей функции на стыках интервалах терпит разрыв.

Такая интерполяция называется кусочно-параболической (если аппроксимирующий полином - парабола ($P_2(x)$)) или кусочно-кубической (в случае $P_3(x)$).

Добиться необходимой гладкости можно с помощью сплайн-функции. Рассмотрим на $[a, b]$ сетку узлов $w = \{ a = x_0 < \ldots < x_m = b \}$ и введем опредление.

Функция $S(x) \in \mathbb{C}^{p-1} [a, b]$ - сплайн степени $p$, если на каждом отрезке $\Delta_i = [x_i, x_{i + 1}], i = \overline{0, m - 1}$, является полиномом степени $p \geqslant 2$. 

\begin{equation*}
  S(x) = S_i(x) = \sum\limits_{k = 0}^p a_k^{(i)} (x - x_i)^k, i = \overline{0, m - 1},
\end{equation*}
где $(i)$ означает, что $a_k$ принадлежит $\Delta_i$.

Условие непрерывности сплайн-функции означает непрерывность $S(x)$ и ее производной вплоть до порядка $p-1$ во всех внутренних узлах сетки. Т.о., это условие дает $p (m-1)$ уравнений относительно неизвестных коэффициентов $a_k^{(i)}$. 

В задачах интерполяции $S(x)$ должна совпадать со значениями функции $f(x)$ в узлах сетки: $S(x_i) = f(x_i), i = \overline{0, m}$. Получаем еще $(m + 1)$ уравнение (условие).

Не хватает еще $(p - 1)$ уравнения. Эти уравнения обычно находят из известных граничных условий на отрезке $[a, b]$.

\newpage
\begin{col-answer-preambule}
	\begin{plan}
    \item пока пусто
	\end{plan}
\end{col-answer-preambule}

\colquestion{Интерполяционный кубический сплайн}

Интерполяционным кубическим сплайном называется функция
\begin{equation*}
  S(x) = S_i (x) = a_0^{(i)} + a_1^{(i)}(x - x_i) + a_2^{(i)}(x-x_i)^2 + a_3^{(i)} (x - x_i)^3, i = \overline{0, m - 1},
\end{equation*}
где $S(x) \in \mathbb{C}^{2} [a, b], x \in [x_i, x_{i + 1}]$.

\begin{equation}\label{eq:11_2}
  S(x_i) = f(x_i), i = \overline{0, m}.
\end{equation}

Условия непрерывности сплайна $S(x)$ и ее производных $S^{'}(x), S^{''}(x)$ во
внутренних узлах сетки дают $3 (m - 1)$ условия. Условие интерполяции \eqref{eq:11_2}
дают еще $m + 1$ условие. Для однозначного определения сплайна $S(x)$ всего требуется $4m$ условий. Недостающие 2 условия определяют из граничных условий.

\textit{Типичные граничные условия}:

Если известны $f^{'}(a), f^{'}(b)$, тогда
\begin{equation*}
  \begin{cases} S^{'}(a) = f^{'}(a), \\ S^{'}(b) = f^{'}(b). \end{cases}
\end{equation*}

Если известны $f^{''}(a), f^{''}(b)$, тогда
\begin{equation*}
  \begin{cases} S^{''}(a) = f^{''}(a), \\ S^{''}(b) = f^{''}(b). \end{cases}
\end{equation*}

Если $f(x)$ - периодическая с длиной периода $b - a$, то тогда полагают:
\begin{equation*}
  \begin{cases} S(a) = S(b), \\ S^{'} (a) = S^{'} (b), \\  S^{''} (a) = S^{''} (b) \end{cases}.
\end{equation*}

\textit{Рассмотрим построение сплайн-функции на $i$-том отрезке}:

Введем обозначения: $M_i = S^{''}(x_i), i = \overline{0, n}$.

Т.к. вторая производная сплайна является линейной функцией, то ее можно представлять в виде интерполяционного полинома Лагранжа.

\begin{equation}\label{eq:11_7}
  S^{''}_{i - 1} (x) = M_{i - 1} \dfrac{x_i - x}{h_i} + M_i \dfrac{x - x_{i -1}}{h_i},
\end{equation}
где $h_i = x_i - x_{i - 1}$.

$M_i, M_{i - 1}$ называются моментами.

Интегрируя дважды \eqref{eq:11_7}, получаем:
\begin{equation}\label{eq:11_8}
  S_{i - 1}(x) = M_{i - 1} \dfrac{(x_i - x)^3}{6 h_i} + M_i \dfrac{(x - x_{i_1})^3}{6 h_i} +  C_{i - 1} (x - x_{i - 1}) + B_{i - 1}.
\end{equation}

Найдем константы $C$ и $B$. Для определения $B_i$ положим $x = x_{i - 1}$, а для определния $C_i$ положим $x = x_{i}$, тогда из \eqref{eq:11_8} $\Rightarrow$
\begin{equation*}
  \begin{cases} B_{i - 1} = f_{i - 1} - \dfrac{h_i^2}{6} M_{i - 1}, \\ C_{i - 1} = \dfrac{f_i - f_{i - 1}}{h_i} - \dfrac{h_i}{6}(M_i - M_{i - 1}). \end{cases}
\end{equation*}

Первая производная сплайна должна быть непрерывна в точке $x_i$. Имеем:
\begin{equation}\label{eq:11_9}
  S^{'}_{i - 1}(x_i) = \dfrac{h_i}{6} \cdot M_{i - 1} + \dfrac{h_i}{3} \cdot M_i + \dfrac{f_i - f_{i - 1}}{h_i}.
\end{equation}

Для сплайна на отрезке $[x_i, x_{i+1}]$ получаем:
\begin{equation*}
  S_{i}(x) \overset{\eqref{eq:11_8}}{=} M_i \dfrac{(x_{i+1} - x)^3}{6 h_{i+1}} + M_{i+1} \dfrac{(x - x_i)^3}{6 h_{i+1}} +  C_i (x - x_i) + B_i.
\end{equation*}

Продифференцируем и получим $S^{'}_{i}(x_i)$:
\begin{equation}\label{eq:11_10}
  S^{'}_{i}(x_i) = - \dfrac{h_{i+1}}{3} M_{i}  - \dfrac{h_{i+1}}{6} M_{i+1} + \dfrac{f_{i+1} - f_i}{h_{i+1}}, i = \overline{1, m - 1}.
\end{equation}

Приравнивая \eqref{eq:11_9} и \eqref{eq:11_10}, получаем систему из $m - 1$ уравнения относительно неизвестных моментов, которую можно записать в виде:

\begin{equation}\label{eq:11_11}
  \nu_i M_{i - 1} + 2 M_i + \lambda_i M_{i + 1} = d_i, i = \overline{1, m - 1},
\end{equation}

где
\begin{equation*}
  \begin{cases}
    d_i = \dfrac{6}{h_i + h_{i + 1}} \left( \dfrac{f_{i + 1} - f_i}{h_{i + 1}} - \dfrac{f_{i} - f_{i - 1}}{h_{i}} \right), \\
    \nu_i = \dfrac{h_i}{h_i + h_{i + 1}}, \\
    \lambda_i = \dfrac{h_{i + 1}}{h_i + h_{i + 1}}.
  \end{cases}
\end{equation*}

Система \eqref{eq:11_11} содержит $m+1$ неизвестных коэффициентов $M_i$ и состоит из $m-1$ уравнения, значит, система \eqref{eq:11_11} недоопределена.

Недостающие 2 уравнения находят из граничных условий.

Будем их записывать в общем виде:

\begin{equation*}
  2 M_0 + \lambda_0 M_1 = d_0,
\end{equation*}

\begin{equation*}
  \nu_m M_{m - 1} + 2 M_m = d_m.
\end{equation*}

В матричной форме система уравнений относительно моментом $M_m$ имеет вид:

\begin{equation}\label{eq:11_14}
\begin{pmatrix}
  2 & \lambda_0 & 0 & \cdots & 0 & 0 & 0\\
  \nu_1 & 2 & \lambda_1 & \cdots & 0 & 0 & 0 \\
  \vdots  & \vdots & \vdots & \ddots & \vdots & \vdots & \vdots \\
  0 & 0 & 0 & \cdots & \nu_{m - 1} & 2 & \lambda_{m - 1} \\
  0 & 0 & 0 & \cdots & 0 & \nu_m & 2
 \end{pmatrix} \begin{pmatrix}
  M_0 \\
  M_1 \\
  \vdots \\
  M_{m-1} \\
  M_m
 \end{pmatrix}
 =
 \begin{pmatrix}
  d_0 \\
  d_1 \\
  \vdots \\
  d_{m-1} \\
  d_m
 \end{pmatrix}
 \end{equation}
Система \eqref{eq:11_14} решается методом прогонки.

Определив моменты $M_i$ из системы \eqref{eq:11_14}, находим по формулам \eqref{eq:11_9} $C_i$, $B_i$, затем - сами сплайны по формуле \eqref{eq:11_8}.

\begin{equation}\label{eq:11_15}
  S_{i - 1}(x) = M_{i - 1} \dfrac{(x_i - x)^3}{6 h_i} + M_i \dfrac{(x - x_{i-1})^3}{6 h_i} + \left( f_i - \frac{1}{6} h_i^2 M_{i-1} \right) \dfrac{x_i-x}{h_i} + \left( f_i - \frac{1}{6} h_i^2 M_i \right) \dfrac{x - x_{i-1}}{h_i},
\end{equation}
где $x \in [x_{i-1}, x_i], i = \overline{1, m}$.

Определим параметры $\lambda_0, \nu_m, d_0, d_m$ для случая, когда известны $f^{'}(a), f^{'}(b)$.

Продифференцируем \eqref{eq:11_15}:
\begin{equation}\label{eq:11_16}
  S^{'}_{i - 1}(x) = M_{i - 1} \left[ \dfrac{h_i}{6} - \dfrac{1}{2 h_i} (x_i - x)^2 \right] + M_i \left[\dfrac{1}{2 h_i} (x - x_{i-1})^2 - \dfrac{h_i}{6} \right]   + \dfrac{f_i-f_{i-1}}{h_i}, i = \overline{1, m}.
\end{equation}

\begin{enumerate}
  \item $i = 1$:
  \begin{equation*}
  S^{'}_{0}(a) = M_{0} \left[ \dfrac{h_1}{6} - \dfrac{h_1}{2} \right] + M_1 \left[- \dfrac{h_1}{6} \right]   + \dfrac{f_1-f_{0}}{h_1} = f^{'}(a) = f^{'}_0, \text{т.к. $a = x_0$}.
  \end{equation*}
  \item $i = m$:
  \begin{equation*}
  S^{'}_{m-1}(b) = M_{m-1} \left[ \dfrac{h_m}{6} \right] + M_m \left[- \dfrac{h_m}{6} + \dfrac{h_m}{2}\right]   + \dfrac{f_m-f_{m-1}}{h_m} = f^{'}(b) = f^{'}_m.
  \end{equation*}
\end{enumerate}

Запишем эти коэффициенты как:
\begin{equation*}
  \begin{cases}
    2 M_0 + M_1 = \dfrac{6}{h_1} \left( \dfrac{f_1 - f_0}{h_1} - f_0 \right), \\
    M_{m + 1} + 2 M_m = \dfrac{6}{h_m} \left(f^{'}_m - \dfrac{f_m - f_{m-1}}{h_m} \right).
  \end{cases}
\end{equation*}

Очевидно, что:
\begin{equation*}
  \begin{cases}
    \lambda_0 = 1, \\
    d_0 = \dfrac{6}{h_1} \left( \dfrac{f_1 - f_0}{h_1} - f_0 \right), \\
    \nu_m = 1, \\
    d_m =  \dfrac{6}{h_m} \left(f^{'}_m - \dfrac{f_m - f_{m-1}}{h_m} \right).
  \end{cases}
\end{equation*}

Та же самая схема для известных значений вторых производных и для периодической функции.

\newpage
\begin{col-answer-preambule}
	\begin{plan}
    \item пока пусто
	\end{plan}
\end{col-answer-preambule}

\colquestion{Наилучшее приближение в линейном векторном пространстве}

В ряде случаев функцию следует аппроксимировать (приближать) не путем интерполяции, а с помощью построения наилучшего приближения.

Пусть $H$ - линейное нормированное пространство.

Требуется найти наилучшие приближения элемента $f \in H$ с помощью ЛК $\sum\limits_{j=1}^n c_j g_{j}$, которые являются линейно-независимыми $g_j \in H, j = \overline{1, n}$.

Т.о., требуется найти элемент $\phi = \sum\limits_{j=1}^n \alpha_j g_j$ такой, что $\Delta = \left| \left| f - \phi \right| \right| = \inf\limits_{c_1, \ldots, c_n} \left| \left| f - \sum\limits_{j=1}^n c_j g_j \right| \right|$.

Если такой элемент существует $\phi \in H$ существует, то он называется элементом наилучшего приближения.

\begin{theorem}
  Элемент наилучшего приближения в линейном нормированном пространстве существует.
\end{theorem}

\textit{Замечание.}

Элемент наилучшего приближения может быть не единственным.

Пространство $H$ называется строго нормированным, если из условия $|| f + g|| = ||f|| + ||g||, ||f|| ||g|| \ne 0,$ следует $f = \alpha g, \alpha \ne 0$.

\begin{theorem}
  Если пространство $H$ строго нормированно, то элемент наилучшего приближения единственен.
\end{theorem}

\newpage
\begin{col-answer-preambule}
	\begin{plan}
    \item пока пусто
	\end{plan}
\end{col-answer-preambule}

\colquestion{Наилучшее приближение в гильбертовом пространстве}

Гильбертовое пространство является полным нормированным, $||x||^2 = (x_1, \ldots, x_n)$.

Для Гилбертова пространства $H$ элемент наилучшего приближения единственен и его построение сводится к решению системы линейных уравнений.

Обозначим $G = \text{span } \{ g_1, \ldots, g_n \}, g_i \in H, i = \overline{1, n}, f \in H$.

$|| f - \phi || = \inf\limits_{h \in G} || f - h||$.

$G$ называется линейным многообразием.

\begin{lemma}
  Пусть $\phi \in H$ - элемент наилучшего приближения. Тогда $(f - \phi) \perp G$ (ортогонален всем элементам).
\end{lemma}

\begin{lemma}
  Если погрешность $(f - \phi) \perp G$, где $G$ - линейная оболочка гилбертова пространства, то $\phi$ - элемент наилучшего приближения.
\end{lemma}

Пусть элемент наилучшего приближения $\phi$ имеет представление $\sum\limits_{j=1}^n \alpha_j g_{j}$. Коэффициенты $\alpha_j$ пока неизвестны:
\begin{equation*}
  f - \phi = f - \sum\limits_{j=1}^n \alpha_j g_{j}
\end{equation*}

\begin{equation}\label{eq:13_8}
  \left(f - \sum\limits_{j=1}^n \alpha_j g_{j}, g_i \right) \overset{\text{п. 1}}{=} 0,
\end{equation}
где $j = \overline{1, n}$ - свойство ортогональности, т.е. $(f - \phi) \perp G$.

\eqref{eq:13_8} - СЛАУ относительно $\alpha$.

Запишем систему \eqref{eq:13_8} в классической форме:
\begin{equation}\label{eq:13_9}
  \sum\limits_{j=1}^n \alpha_j (g_i, g_i) = (f, g_j), j = \overline{1, n}.
\end{equation}

Матричная система \eqref{eq:13_9} - матрица Грама. В силу того, что базисные элементы $g_1, \ldots, g_n$ линейно-независимы, определитель матрицы Грама $\ne 0 \Rightarrow$ \eqref{eq:13_9} имеет единственное решение относительно коэффициента $\alpha$.

\begin{notes}
  \item Если элементы $g_1, \ldots, g_n$ является ортонормированными, т.е. $(g_i, g_j) = \delta_{i, j}, i, j = \overline{1,n}$, то система \eqref{eq:13_8} имеет диагональную матрицу и решение находится как $d_j = (f, g_j), j = \overline{1, n}$. Тогда элемент наилучшего приближения $\phi = \sum\limits_{i=1}^n (f, g_i)g_i$.

  Коэффициент $\alpha_j$ имеет название коэффициента Фурье, а сам многочлен $\phi$ носит название многочлена Фурье.

  \item Если $(f, \phi) = || \phi ||^2$, тогда для элемента наилучшего приближения имеем
  \begin{equation*}
    ||f - \phi||^2 = ||f||^2 - ||\phi||^2.
  \end{equation*}

  В силу равенства Парсеваля $\left( ||f||^2 = \sum\limits_{k=1}^\infty |\alpha_k|^2 \right)$ имеем:

  \begin{equation*}
    ||f - \phi||^2 = \int\limits_{k = n + 1}^{
    \infty
    } |\alpha_k|^2
  \end{equation*}

  Значит, при $n \to \infty$ норма погрешности $|| f - \phi||$ неограниченно убывает, т.е. элемент наименьшего приближения $\phi$ среднеквадратичного сходится к $f$.

  \item Типичным примером гилбертова пространства является пространство $L_2 [a, b]$ - пространство функций $f(x)$, интегрируемых с квадратом на отрезке $[a, b]$, причем:

  \begin{equation*}
  (f, g)_{L_2} = \int\limits_a^b \rho(x) f(x) \overline{g(x)} dx,
  \end{equation*}

  \begin{equation*}
    \left| \left| f \right| \right|_{L_2}^2 = \left( \int\limits_a^b \rho(x) f^2(x) dx \right)^{\frac{1}{2}}
  \end{equation*}

  $\rho(x) \geqslant 0$ - весовая функция.

  $\rho(x) = 0$ на граничном числе точек, мера которого равна 0.

  \item Коэффициенты Фурье $\alpha_i$ дают наилучшие в системе наименьших квадратов приближения, когда $f(x)$ разлагается по ортогональному базису элементов $g_i$.

  Т.о., ортогональные функции $g_i$, нахождение коэффициентов Фурье и идея приближения в смысле наименьших квадратов тесно взаимосвязаны.
\end{notes}

\newpage
\begin{col-answer-preambule}
	\begin{plan}
    \item пока пусто
	\end{plan}
\end{col-answer-preambule}

\colquestion{Метод наименьших квадратов}

\begin{notes}
  \item пока пусто
\end{notes}

\newpage
\begin{col-answer-preambule}
	\begin{plan}
    \item пока пусто
	\end{plan}
\end{col-answer-preambule}

\colquestion{Метод Пикара и метод рядов Тейлора}

\subsection{Метод Пикара}

Рассмотрим задачу Коши для однородного дифференциального уравнения:
\begin{equation}
  \label{15-picar1}
  \begin{cases}
    &u'(x) = f(x, u), u = u(x), x \in [x_0, x_l]\\
    &u(x_0) = u_0
  \end{cases}
\end{equation}

Проинтегрируем  уравнение \eqref{15-picar1}
\begin{equation}
  \label{15-picar-integral}
  u(x) = u(x_0) + \int\limits_{x_0}^xf(t, u(t))dt
\end{equation}

$y$ - приближённое решение, $s$ - номер итерации.
\begin{equation}
  \label{15-picar-formula}
  \begin{cases}
    &y_s(x) = u_0 + \dint\limits_{x_0}^xf(t, y_{s - 1}(t))dt\\
    &y_0(t) = u_0
  \end{cases}
\end{equation}

Этот метод удобен, если интеграл можно вычислить аналитически. Докажем
сходимость метода Пикара.

Пусть в некоторой ограниченной области $G$ функция $f(x, u)$ непрерывная и
удовлетворяет условию Лившица по переменной $u$:
\begin{align}
  \abs{f(x_1, u_1) - f(x_1, u_2)} \leqslant L\abs{u_1 - u_2}\\
  \label{15-limits}
  \begin{cases}
    &\abs{x - x_0} \leqslant E, \forall x \in G\\
    &\abs{u - u_0} \leqslant V, E, V - \const
  \end{cases}
\end{align}

\eqref{15-limits} - условия ограниченности, выполняются в силу ограниченности
области $G$.

\begin{align}
  \eqref{15-picar-integral}, \eqref{15-picar-formula} \Rightarrow
  \abs{y_s(x) - u(x)} = \abs{\int\limits_{x_0}^x f(t, y_{s - 1}(t))dt -
                             \int\limits_{x_0}^x f(t, u(t))dt}\\
  \abs{y_s(x) - u(x)} \leqslant \int\limits_{x_0}^x
      \abs{f(t, y_{s - 1}(t)) - f(t, u(t))}dt
\end{align}

Обозначим $z_s(x) = y_s(x) - u(x)$ - погрешность в точке $x$.

\begin{equation}
  \abs{z_s(x)} \leqslant L\int\limits_{x_0}^x\abs{z_{s - 1}(t)}dt
\end{equation}

Если $s = 0$, то
\begin{equation}
  \label{15-error-formula}
  \begin{split}
    &\abs{z_0(x)} = \abs{u_0 - u(x)} \leqslant V \text{ - погрешность начального
    приближения}.\\
    &\abs{z_1(x)} \leqslant LV\abs{x - x_0}\\
    &\abs{z_2(x)} \leqslant \dfrac{1}{2}L^2V\abs{(x - x_0)^2}\\
    &\ldots\\
    &\abs{z_s(x)} \leqslant \dfrac{1}{s!}L^sV\abs{(x - x_0)^s}\\
  \end{split}
\end{equation}

Формула Стирлинга:
\begin{align*}
  n! \approx \dfrac{\sqrt{2\pi}n^{n + \frac{1}{2}}}{e^n}(1 + \varepsilon_n),
  \lim\limits_{n \to \infty}\varepsilon_n = 0
\end{align*}
\begin{equation}
  \eqref{15-error-formula} \Leftrightarrow \abs{z_s(x)} \leqslant
  \dfrac{1}{s!}L^sVE^s
\end{equation}

Используя формулу Стирлинга
\begin{equation}
  \label{15-error-approx}
  \abs{z_s(x)} \leqslant \dfrac{v}{\sqrt{2\pi s}}\left(\dfrac{eEL}{s}\right)^s
\end{equation}

$\eqref{15-error-approx} \Rightarrow \abs{z_s(x)} \xrightarrow[s \to \infty]{}0
\Rightarrow $ итерационный процесс сходится.

\subsection{Метод рядов Тейлора}

Рассмотрим
\begin{equation}
  \label{15-taylor-example}
  \begin{cases}
    &u' = f(x, u), x \in [x_0, x_l]\\
    &u(x_0) = u_0\\
  \end{cases}
\end{equation}

Продифференцируем \eqref{15-taylor-example} по $x$:
\begin{equation}
  \label{15-taylor-diff}
  \begin{split}
    &u'' = f_x + f_u\cdot u' = f_x + f\cdot f_u\\
    &u''' = f_{xx} + 2f_{xu}u' + f_{uu}u'^2 + f_u u''\\
    &\ldots
  \end{split}
\end{equation}

Подставим в формулу \eqref{15-taylor-diff} в качестве $x = x_0, u = u_0$,
последовательно находим значения $u'(x_0), u''(x_0), u'''(x_0)$ и т. д.
Получаем ряд Тейлора:
\begin{equation}
  u(x) \approx y_n(x) = \sum\limits_{i = 0}^n\dfrac{u^{(i)}(x_0)}{i!}\cdot
  (x - x_0)^i
\end{equation}

Если $\abs{x - x_0}$ не превышает радиуса сходимости ряда Тейлора, то
приближенное решение $y_n(x) \xrightarrow[n \to \infty]{} u(x)$.

Иногда полезно разбить исходный отрезок $[x_0, x_l]$ на $N$ частей
$[x_{j - 1}, x_j], j = \overline{1, N}, x_N = x_l$. Отрезки не обязательно
равные. На каждом отрезке применим метод рядов Тейлора для более точного
решения.

Рассмотрим произвольный отрезок $[x_j, x_{j + 1}]$. Будем считать, что $y_j$
найдено. Значит, мы можем найти $u^{(i)}(x_j)$. Тогда применяя метод рядов,
можно приблизить на этом отрезке

\begin{equation}
  \begin{split}
    &u(x) \approx v_j(x) = \sum\limits_{i = 0}^n\dfrac{u_j^{(i)}}{i!}(x - x_j)^i\\
    &y_{j + 1} = v_j(x_{j + 1})
  \end{split}
\end{equation}

При использовании метода рядов необходимо находить значения
$\approx \dfrac{n(n + 1)}{2}$ различных функций, поэтому на практике обычно
ограничиваются первым и вторым порядком точности (2-3 производные).

\newpage
\begin{col-answer-preambule}
	\begin{plan}
    \item пока пусто
	\end{plan}
\end{col-answer-preambule}

\colquestion{Методы Эйлера, трапеций, средней точки}

\begin{equation}
  \label{16-problem}
  \begin{cases}
    u' = f(x, u)\\
    u(x_0) = u_0.
  \end{cases}
\end{equation}

Проинтегрируем \eqref{16-problem} на отрезке $[x_n, x_{n + 1}], x_{n + 1} - x_n = h$.

\begin{equation}
  \label{16-problem-integral}
  u_{n + 1} = u_n + \int\limits_{x_n}^{x_{n + 1}}f(t, u(t))dt, u_n \equiv u(x_n)
\end{equation}

Используем для вычисления интеграла в формуле \eqref{16-problem-integral}
правило левых прямоугольников
\begin{align*}
  \int\limits_A^Bfdx \approx f(A)(B - A)
\end{align*}

\begin{equation}
  \label{16-left-rects}
  u_{n + 1} = u_n + hf_n + R_2(f)
\end{equation}

Отбрасывая в \eqref{16-left-rects} $R_2(f)$, получаем
\textbf{явную фомрулу Эйлера}
\begin{equation}
  y_{n + 1} = y_n + hf(x_n, y_n) = y_n + hf_n
\end{equation}

Если для вычисления интеграла в формуле \eqref{16-problem-integral} применить
формулу правых прямоугольников
\begin{align*}
  \int\limits_A^Bfdx \approx f(B)(B - A)
\end{align*}

получим \textbf{неявную формулу Эйлера}
\begin{equation}
  y_{n + 1} = y_n + hf(x_{n + 1, y_{n + 1}}) = y_n + hf_{n + 1}
\end{equation}

В общем случае неявный метод Эйлера представляет собой неявное уравнение
отностельно искомого значения $y_{n + 1}$. Для решеня неявного уравнения можно
использовать итерационный метод (например метод простой итерации).

\begin{equation}
  y_{n + 1}^{k + 1} = y_n + hf(x_{n + 1}, y_{n + 1}^k), k = 0, 1, \ldots;
  n = \overline{0, N - 1}.
\end{equation}

Чтобы итерационный метод сходился, достаточно потребовать, чтобы
\begin{equation}
  h\abs{\dfrac{\delta f}{\delta y_{n + 1}}} < 1, \forall n
\end{equation}

В качестве нулевого приближения возьмём:
\begin{enumerate}
  \item $y_{n + 1}^0 = y_n$
  \item $y_{n + 1}^0 = y_n + hf_n$
\end{enumerate}

Локальная погрешность явного и неявного метода Эйлера (это погрешность
нахождения $u(x + h)$ при известном значении $u(x)$) имеет порядок $O(h^2)$.

Рассмотрим для неявного метода Эйлера:
\begin{equation}
  \begin{split}
    r_{n + 1} &= u(x_n + h) - u(x_n) - hf(x_{n} + h, u(x_n + h)) =\\
    &=u_n + hu'_n + \dfrac{h^2}{2}u''_n + O(h^3) - u_n - h(u'_n + hu''_n + O(h^2))=\\
    &= -\dfrac{h^2}{2}u''_n + O(h^3) = O(h^2)
  \end{split}
\end{equation}

Неявный метод Эйлера сложнее в реализации, но имеет значительное преимущество
перед явным за счёт своей устойчивости.

Если интеграл в правой части вычислить по формуле трапеций, то получим:
\begin{equation}
  \begin{split}
    \begin{cases}
      y_{n + 1} = y_n + \dfrac{h}{2}(f_n + f_{n + 1})\\
      y_0 = u_0, n = 0, 1, \ldots
    \end{cases}
    r_{n + 1} = O(h^3)
  \end{split}
\end{equation}

Применим для вычисления \eqref{16-problem-integral} правило средних
прямоугольников (формулу средней точки)
\begin{equation}
  \label{16-med-point}
  y_{n + 1} = y_n + hf_{n + \frac{1}{2}}; y_0 = u_0, n = 0, 1, \ldots
\end{equation}

Чтобы вычислить $f_{n + \frac{1}{2}}$ необходимо знать значение
$y_{n + \frac{1}{2}}$. Способы вычисления:
\begin{equation}
  \label{16-half1}
  y_{n + \frac{1}{2}} \approx \frac{1}{2}(y_n + y_{n + 1})
\end{equation}
\begin{equation}
  \label{16-half2}
  y_{n + \frac{1}{2}} = y_n + \frac{h}{2}f_n
\end{equation}

Если применять \eqref{16-med-point} и \eqref{16-half1}, то получим
\begin{equation}
  y_{n + 1} = y_n + hf\left(x_n + \frac{h}{2}, \frac{1}{2}(y_n + y_{n + 1})
  \right)
\end{equation}

Если применять \eqref{16-med-point} и \eqref{16-half2}, то получим
\begin{equation}
  y_{n + 1} = y_n + hf\left(x_n + \frac{h}{2}, y_n + \frac{1}{2}f_n\right)
\end{equation}

\newpage
 \begin{col-answer-preambule}
	\begin{plan}
    \item пока пусто
	\end{plan}
\end{col-answer-preambule}

\colquestion{Сходимость явного метода Эйлера}

При иcпользовании приближённых методом основным является вопрос о сходимости.
Сформулируем понятие сходимости, когда $h \to 0$. Зафиксируем некоторую точку
$x$ и будем строить последовательность сеток $\omega_h$ таких, что $h \to 0,
x_n = x_0 + nh = x$.

\begin{definition}
  Говорят, что методя сходится в точке $x$, если разностное решение
  $\abs{y_n(x) - u(x)} \xrightarrow[h \to 0]{} 0$.
\end{definition}

\begin{definition}
  Метод сходится на отрезке $[x_0, x]$, если он сходится в каждой точке этого
  отрезка.
\end{definition}

\begin{definition}
  Говорят, что метод имеет порядок точности $p > 0$, если $\abs{y_n(x) - u(x)}
  = O(h^p), h > 0$.
\end{definition}

Исследуем сходимость явного метода Эйлера
\begin{align}
  &y_{n + 1} = y_n + hf_n\\
  \label{17-taylor}
  &u(x_{n + 1}) = u(x_n) + hu'(x_n) + \dfrac{h^2}{2}u''(\xi_n),\\
  \nonumber
  &x_n < \xi_n < x_{n + 1}, u(x) \in C^2[x_0, x]\\
  &u' = f(x, u)\\
  &\eqref{17-taylor} \Leftrightarrow u(x_{n + 1}) = u(x_n) + hf(x_n, u(x_n)) +
  \dfrac{h^2}{2}u''(\xi_n)\\
  \label{17-some-cons}
  &u(x_{n + 1}) - y_{n + 1} = u(x_n) - y_n + h(f(x_n, u(x_n)) - f(x_n, y_n)) +
  \dfrac{h^2}{2}u''(\xi_n)
\end{align}

Введём обозначение погрешности в $n$-ой точке
\begin{equation}
  E_n = u(x_n) - y_n
\end{equation}

Будем полагать, что функция $f$ удовлетворяет условию Липшица с $\const L$ по
второму аргумент, тогда $\eqref{17-some-cons} \Rightarrow$

\begin{equation}
  \abs{E_{n + 1}} \leqslant \abs{E_n} + hL(\abs{u(x_n) - y_n}) + \dfrac{h^2}{2}
  \abs{u''(\xi_n)}.
\end{equation}

\begin{definition}
  $M_2 = \underset{x \in [a, b]}{\max} \abs{u''(x)}$
\end{definition}

\begin{equation}
  \label{17-error-approx}
  \abs{E_{n + 1}} \leqslant \abs{E_n}(1 + hL) + \dfrac{h^2}{2}M_2, n = 0, 1,
  \ldots
\end{equation}

Слагаемое $\dfrac{h^2}{2}M_2$ - оценка локальной погрешности метода, которая
возникает на очередном шаге.

Для оценки погрешности $E_n$ рассмотрим обобщение неравенства
\eqref{17-error-approx}. Будем полагать, что $\exists \delta > 0, M > 0$,
такие, что последовательность $d_0, d_1, \ldots$ удовлетворяет неравенству
\begin{equation}
  d_{n + 1} \leqslant (1 + \delta)d_n + M, n = 0, 1, \ldots
\end{equation}

Тогда
\begin{align}
  \nonumber
  &d_1 \leqslant (1 + \delta)d_0 + M\\
  \nonumber
  &d_2 \leqslant (1 + \delta)d_1 + M \leqslant (1 + \delta)^2d_0 +
  M(1 + (1 + \delta))\\
  \nonumber
  &\ldots\\
  &d_n \leqslant (1 + \delta)^n + M(1 + (1 + \delta) + \ldots
  + (1 + \delta)^{n - 1})\\
  \label{17-dlim}
  &d_n \leqslant (1 + \delta)^nd_0 + M\dfrac{(1 + \delta)^n - 1}{\delta}
\end{align}

Из разложения экспоненты:
\begin{align}
  &e^{\delta} = 1 + \delta + \dfrac{\delta^2}{2}e^{\xi}, 0 < \xi < \delta
  \Rightarrow 1 + \delta \leqslant e^{\delta}\\
  \label{17-expn}
  &(1 + \delta)^n \leqslant e^{n\delta}
\end{align}

Подставим \eqref{17-expn} в \eqref{17-dlim} и получим оценку:
\begin{equation}
  \label{17-dlim2}
  d_n \leqslant e^{n\delta}d_0 + M\dfrac{e^{n\delta} - 1}{\delta}
\end{equation}

Применим неравенство \eqref{17-dlim2} к формуле \eqref{17-error-approx}
\begin{align}
  &\abs{E_n} \leqslant e^{nhL}\abs{E_0} + \dfrac{hM_2}{2L}
  \left(e^{nhL} - 1\right)\\
  \nonumber
  &nh = x_n - a, E_0 = u_0 - y_0 = 0\\
  &\abs{u(x_n) - y_n} \leqslant \dfrac{hM_2}{2L}\left(e^{L(b - a)} - 1\right)\\
  \label{17-max-error}
  &\underset{h}\max\abs{u(x_n) - y_n} \leqslant \dfrac{hM_2}{2L}
  \left(e^{L(b - a)} - 1\right) \xrightarrow[h \to 0]{} 0
\end{align}

В общем случае следует учитывать погрешность округления. На практике, когда
вычисляется $f(x_n, y_n)$, на самом деле мы находим $f(x_n, y_n) + \varepsilon_n$.
Кроме этого, когда по формуле Эйлера мы находим
\begin{equation}
  y_{n + 1} = y_n + h(f(x_n, y_n) + \varepsilon_n) + \rho_n
\end{equation}
появляется погрешность $\rho_n$.

Будем полагать, что $\abs{\rho_n} \leqslant \rho, \abs{\varepsilon_n} \leqslant
\varepsilon, \forall h \leqslant h_0$. Тогда формулу \eqref{17-max-error}
надо изменить следующим образом
\begin{equation}
  \label{17-real-error}
  \underset{h}\max\abs{u(x_n) - y_n} \leqslant \dfrac{1}{L}
  \left(e^{L(b - a)} - 1\right)\left(\dfrac{hM_2}{2} + \varepsilon +
  \dfrac{\rho}{h}\right)
\end{equation}

Из оценки \eqref{17-real-error} следует, что повышать точность за счёт уменьшения
шага $h$ можно только до некоторого предела, за которым погрешность округления
будет доминировать.

\newpage
\begin{col-answer-preambule}
	\begin{plan}
    \item пока пусто
	\end{plan}
\end{col-answer-preambule}

\colquestion{Методы последовательного повышения порядка точности}

\begin{notes}
  \item пока пусто
\end{notes}

\newpage
\begin{col-answer-preambule}
	\begin{plan}
    \item пока пусто
	\end{plan}
\end{col-answer-preambule}

\colquestion{Методы Рунге-Кутта}

Исходное уравнение $u' = f(x, u)$. Интегрируем на отрезке $[x_n, x_n + h]$.

\begin{equation}
  \label{19-integral}
  u_{n + 1} = u_n + h\int\limits_0^1f(x_n + \alpha h, u(x_n + \alpha h))d\alpha
\end{equation}

Для вычисления интеграла предлагается использование следующего набора параметров

\begin{tabular}{c|c|c c c c}
  $A_0$ & & & & & \\
  $A_1$ & $\alpha_1$ & $\beta_{10}$ & & & \\
  $A_2$ & $\alpha_2$ & $\beta_{20}$ & $\beta_{21}$ & & \\
  $\vdots$ & $\vdots$ & $\vdots$ & $\vdots$ & $\ddots$ & \\
  $A_q$ & $\alpha_q$ & $\beta_{q0}$ & $\beta_{q1}$ & $\ldots$ & $\beta_{qq-1}$
\end{tabular}

При помощи параметров $\alpha$ и $\beta$ последовательно находим
\begin{equation}
  \label{19-alpha-beta}
  \begin{split}
    &\phi_0 = h f(x_n, y_n)\\
    &\phi_1 = h f(x_n + \alpha_1 h, y_n + \beta_{10}\phi_0)\\
    &\ldots\\
    &\phi_q = h f(x_n + \alpha_q h, y_n + \sum\limits_{j = 0}^{q - 1}
    \beta_{qj}\phi_j)
  \end{split}
\end{equation}

Параметры $\phi_0, \phi_1, \ldots \phi_q$ находится последовательно. После этого
интеграл в \eqref{19-integral} заменяется на $\sum\limits_{i = 0}^qA_i\phi_i$.
В результате получим формулу
\begin{equation}
  \label{19-int-approx}
  y_{n + 1} = y_n + \sum\limits_{i = 0}^qA_i\phi_i
\end{equation}

Неизвестные параметры $A, \alpha, \beta$ выбираются таким образом, чтобы при
заданном значении $q$ построить метод максимально высокого порядка точности.

\begin{definition}
  Формулы \eqref{19-alpha-beta}, \eqref{19-int-approx} - метод Рунге-Кутты.
\end{definition}

Локальная погрешность
\begin{equation}
  r_q(h) = u(x_n + h) - u(x_n) - \sum\limits_{i = 0}^qA_i\phi_i
\end{equation}

Считая функцию $f$ достаточно гладной, запишем разложение в ряд Тейлора
\begin{equation}
  r_q(h) = \sum\limits_{j = 0}^k\dfrac{h^j}{j!}r_q^{(j)}(0) + O(h^{k + 1})
\end{equation}

Если параметры $A, \alpha, \beta$ выбрать таким образом, чтобы производные
\begin{equation}
  \label{19-precision}
  r_q^{(j)} = 0, \forall j = \overline{0, k}
\end{equation}
то метод будет иметь $k$-ый порядок погрешности.

\begin{examples}
  \item $q = 0$.
  \begin{align*}
    &r_0(h) = u(x_n + h) - u(x_n) - hA_0f_n\\
    &r_0'(h) = u'(x_n + h) - A_0f_n\\
    &r_0''(h) = u''(x_n + h)
  \end{align*}
  Условие \eqref{19-precision} выполняется, когда $A_0 = 1, j = \overline{0, 1}$.
  \begin{align*}
    &y_{n + 1} = y_n + hf_n\\
    &u'(x_n) = f_n
  \end{align*}
  В итоге приходим к формуле явного метода Эйлера - метода первого порядка
  точности.
  \item $q = 1$.
  \begin{equation}
    \label{19-coef1}
    u_{n + 1} - u_n = hf_n + \dfrac{h^2}{2}(f_x + ff_u)_n +
    \dfrac{h^3}{6}(f_{xx} + 2ff_{xu} + f^2f_{uu} + f(f_x + ff_u))_n + O(h^4)
  \end{equation}
  \begin{equation}
    \label{19-coef2}
    \begin{split}
      &A_0\phi_0 + A_1\phi_1 = h(A_0f_n + A_1f(x_n + \alpha_1h, u + h\beta_{10}f_n)) =\\
      &= h(A_0 + A_1)f_n + h^2A_1(\alpha_1f_x + \beta_{10}ff_u)_n +
      \dfrac{h^3}{2}A_1(\alpha_1^2f_{xx} + 2\alpha_1\beta_{10}ff_{xu} + \beta_{10}^2
      f^2f_{uu}) + O(h^4)
    \end{split}
  \end{equation}
  \begin{equation}
    \label{19-q2}
    \begin{aligned}
      &hf: &A_0 + A_1 = 1\\
      &h^2f_x: &A_1\alpha_1 = \dfrac{1}{2}\\
      &h^2ff_u: &A_1\beta_{10} = \dfrac{1}{2}
    \end{aligned}
  \end{equation}
  При выполнении условий \eqref{19-q2} нельзя добиться совпадения коэффициентов
  \eqref{19-coef1} и \eqref{19-coef2} формулы при $h^3$. Поэтому при $q = 1$
  метод Рунге-Кутты имеет второй порядок точности
  \begin{equation}
    \label{19-coef3}
    \begin{cases}
      \alpha_1 = \beta_{10} = \dfrac{1}{2A_1}\\
      A_0 = 1 - A_1
    \end{cases}
  \end{equation}
  Формула \eqref{19-coef3} даёт семейство методов Рунге-Кутты второго порядка
  точности, где $A_1$ - свободный параметр. Возьмём $A_1 = \dfrac{1}{2}$.
  \begin{equation}
    \begin{cases}
      y_{n + 1} = y_n + \dfrac{\phi_0 + \phi_1}{2}\\
      \phi_1 = hf(x_n + h, y_n + \phi_0)\\
      \phi_0 = hf_n
    \end{cases}
  \end{equation}

  Если $A_1 = 1$.
  \begin{equation}
    \begin{cases}
      y_{n + 1} = y_n + \phi_1\\
      \phi_1 = hf\left(x_n + \dfrac{h}{2}, y_n + \dfrac{\phi_0}{2}\right)\\
      \phi_0 = hf_n
    \end{cases}
  \end{equation}
\end{examples}

\newpage
\begin{col-answer-preambule}
	\begin{plan}
    \item пока пусто
	\end{plan}
\end{col-answer-preambule}

\colquestion{Экстраполяционные методы Адамса}

\begin{notes}
  \item пока пусто
\end{notes}

\newpage
\begin{col-answer-preambule}
	\begin{plan}
    \item пока пусто
	\end{plan}
\end{col-answer-preambule}

\colquestion{Интерполяционные методы Адамса}

Являются неявными методами и определяются расчётной формулой
\begin{equation}
  y_{n + 1} = y_n + h\sum\limits_{i = -1}^qA_if_{n - i}
\end{equation}

Параметры $A_i$ определяются также, как и в методе последовательного повышения
порядка точности.
\begin{equation}
  \begin{split}
    &\alpha_i = -i, i = \overline{1, q}\\
    &\sum\limits_{i = -1}^qA_i(-i)^j = \dfrac{1}{j + 1}, j = \overline{0, q + 1}
  \end{split}
\end{equation}

Система имеет единственное решение при $q \geqslant -1$.

Параметры $A_i$ можно найти, используя интерполяционный полином Лагранжа, который
строится по значениям функции $f$ в узлах
\begin{align*}
  &x: x_{n + 1}, x_n, \ldots x_{n - q}\\
  &\alpha: 1, 0, \ldots -q
\end{align*}

Получим полином $(q + 1)$-ой степени, где
\begin{align}
  &A_i = \dfrac{(-1)^{i + 1}}{(i + 1)!(q - i)!}\int\limits_0^1
  \dfrac{(\alpha - 1)\alpha(\alpha + 1)\ldots(\alpha + 1)}{(\alpha + i)}d\alpha\\
  &r_{n + 1} = h^{q + 3}u_n^{(q + 3)}\left(\dfrac{1}{(q + 3)!}
  - \dfrac{1}{(q + 2)!}\sum\limits_{i = -1}^q A_i(-i)^{q + 2}\right)
  + O(h^{q + 4}) = O(h^{q + 3})
\end{align}

Т.е. метод имеет порядок точности $(q + 2)$.

\begin{examples}
  \item $q = -1$.
  \begin{equation}
    y_{n + 1} = y_n + h f_{n + 1} \text{ совпадает с неявным методом Эйлера}
  \end{equation}
  \begin{align*}
    r_{n + 1} = -\dfrac{1}{2}h^2u''_n + O(h^3) = O(h^2)
  \end{align*}
  Для случая, когда сетка равномерная, можно подынтегральную функцию заменить
  интерполяционным полиномом Ньютона.
  \item $q = 0$.
  \begin{align}
    &y_{n + 1} = y_n + \dfrac{h}{2}(f_{n - 1} + f_n) \text{ - формула трапеций}\\
    \nonumber
    &r_{n + 1} = -\dfrac{1}{12}h^3u_n^{(3)} + O(h^4) = O(h^3)
  \end{align}
  В общем виде:
  \begin{equation}
    y_{n + 1} = y_n + \phi_{n + 1} - \dfrac{1}{2}\Delta\phi_n - \dfrac{1}{12}
    \Delta^2\phi_{n - 1} - \dfrac{1}{24}\Delta^3\phi_{n - 2} - \ldots
    - C_{q + 1}\Delta^{q + 1}\phi_{n - q}
  \end{equation}
  \begin{align*}
    \phi_i = hf_i, C_{q + 1} = \dfrac{1}{(q + 1)!}\int\limits_0^1\alpha
    (\alpha + 1)\ldots(\alpha + q)d\alpha
  \end{align*}
  \begin{equation}
    r_{n + 1} = h^{q + 3}u_n^{(q + 3)}C_{q + 2} + O(h^{q + 4})
  \end{equation}
\end{examples}

Поскольку отрезок $[x_n, x_{n + 1}]$, на котором аппроксимируется функция $f$
входит в отрезок $[x_{n - q}, x_{n + 1}]$, на котором расположены узлы интерполяции...

Интерполяционные формулы Адамса представляют собой в общем случае неявное
уравнение относительно $y_{n + 1}$. Значение $y_{n + 1}$ находится с помощью
некоторого итерационного метода.

В качестве нулевой итерации обычно берут значение, приближённое значение,
полученной с помощью эксстраполяционного метода Адамса или методом Рунге-Кутты.
При этом часто ограничиваются только одной итерацией. В этом случае вычислительный
процесс относится к типу \textbf{предиктор-корректор}.

\newpage
\begin{col-answer-preambule}
	\begin{plan}
    \item пока пусто
	\end{plan}
\end{col-answer-preambule}

\colquestion{Усточивость линейных многошаговых методов}

Будем рассматривать задачу Коши
\begin{equation}
  \begin{cases}
    u' = f(x, u), x > 0\\
    u(0) = u_0
  \end{cases}
\end{equation}

Возьмём равномерную сетку узлов $\omega_h = \set{x_n = nh, n = 0, 1, \ldots}$.
Будем рассматривать линейный $m$-шаговый метод.
\begin{align}
  \label{22-multistep}
  &a_0y_n + a_1y_{n - 1} + \ldots + a_my_{n - m} = h\sum\limits_{k = 0}^mb_kf_{n-k}\\
  \nonumber
  &n = m, m + 1, \ldots
\end{align}

Коэфициенты $a_k, b_k = \const, a_0 \neq 0$. Для счёта по формуле \eqref{22-multistep}
необходимо задать $m$ начальных значений $y_0 = u_0, y_1, \ldots y_{m - 1}$. Обычно они
находятся с помощью одношагового метода Рунге-Кутты того же порядка точности, что и метод
\eqref{22-multistep}.

Запишем соответствующее однородное уравнение
\begin{align}
  \label{22-le}
  &a_0\delta_n + a_1\delta_{n - 1} + \ldots + a_m\delta_{n - m} = 0\\
  \nonumber
  &n = m, m + 1, \ldots
\end{align}

Будем искать частные решения \eqref{22-le} в виде $\delta_n = q^n, q = \const$, тогда для
опредления постоянной $q$ получим уравнение
\begin{equation}
  \label{22-char}
  a_0q^m + a_1q^{m - 1} + \ldots + a_m = 0
\end{equation}

\begin{definition}
  Уравнение \eqref{22-char} - характеристическое уравнение метода \eqref{22-multistep}.
\end{definition}

\begin{definition}
  \eqref{22-multistep} - линейный двухшаговый метод, удовлетворяет условию корней, если
  все корни $q_1, \ldots q_m$ характеристического уравнения \eqref{22-char} лежат внутри
  или на границе единичного круга комплексной плоскости. Причём на границе этого круга нет
  кратных корней.
\end{definition}

\begin{definition}
  Однородное уравнение \eqref{22-le} устойчиво по начальным данным, если $\exists \const
  M > 0$, независящая от номера узла $n$, такая, что при любых начальных данных
  $\delta_0, \ldots \delta_{m - 1}$ выполняется следующая оценка решения
\end{definition}
\begin{equation}
  \label{22-approx}
  \abs{\delta_n} \leqslant M \underset{0 \leqslant i \leqslant m - 1}{\max}\abs{\delta_i},
  n = m, m + 1, \ldots
\end{equation}

Таким образом устойчивость по начальным данным означает равномерную по $n$ ограниченность
решения задачи Коши.

\begin{theorem}
  Условие корней необходимо и достаточно для устойчивости метода \eqref{22-le} по
  начальным данным.
\end{theorem}
\begin{proof}

  \begin{enumerate}
    \item $\Rightarrow$. Пусть имеется корень $\abs{q} > 1$. Зададим начальные данные
    $\delta_i = q^i (i = \overline{0, m - 1})$. Тогда уравнение \eqref{22-le} имеет
    решение в точке $\delta_n = q^n(n \geqslant m)$, которое неограничено возрастает при
    $n \to \infty$. Оценка \eqref{22-approx} не выполняется.

    Следовательно, условие $\abs{q_k} \leqslant 1, k = \overline{1, m}$ - необходимое
    условие устойчивости. Пусть характеристическое уравнение \eqref{22-char} имеет корень
    $q$ с кратностью $r > 1$, причём этот корень находится на границе единичного круга на
    комплексной плоскости $\abs{q} = 1$. В этом случае однородное уравнение
    \eqref{22-le}
    имеет решение вида
    \begin{align*}
      \delta_n = q^n\cdot n^{r - 1}
    \end{align*}
    и оценка \eqref{22-approx} снова не выполняется.
    \item $\Leftarrow$. Без доказательства.
  \end{enumerate}
\end{proof}

Можно показать, что если уравнение \eqref{22-le} устойчиво по начальным данным, то для
неоднородного случая
\begin{equation}
  \label{22-nonuniform}
  a_0y_n + a_1y_{n - 1} + \ldots + a_my_{n - m} = hg_{n - m}, n = m, m + 1, \ldots
\end{equation}
выполняется оценка
\begin{equation}
  \abs{y_n} \leqslant M_1\underset{1 \leqslant j \leqslant m}{\abs{y_j}} +
  M_2\sum\limits_{k = 0}^{n - m}h\abs{g_k}
\end{equation}
которая означает устойчивость \eqref{22-nonuniform} по правой части и по начальным
данным.

\newpage
\begin{col-answer-preambule}
	\begin{plan}
    \item пока пусто
	\end{plan}
\end{col-answer-preambule}

\colquestion{Простейшие разностные операторы}

\begin{notes}
  \item пока пусто
\end{notes}

\newpage
\begin{col-answer-preambule}
	\begin{plan}
    \item пока пусто
	\end{plan}
\end{col-answer-preambule}

\colquestion{Основные понятия теории разностных схем}

\begin{notes}
  \item пока пусто
\end{notes}

\newpage
\begin{col-answer-preambule}
	\begin{plan}
    \item пока пусто
	\end{plan}
\end{col-answer-preambule}

\colquestion{Интегро-интерполяционный метод}

Вывод основных дифференциальных уравнений основан на записи соответствующих
законов сохранения для элементарных объёмов и стягивания этих объёмов к нулю.
Разностные схемы, выражающие законы сохранения на сетке, нызываются
\textbf{консервативными}. При этом законы сохранения для всей сетки должны быть
алгебраическим следствием разностных уравнений. Для построения консервативных
разностных схем естественно исходить из законов сохранения для отдельных ячеек
разностной сетки. Такой метод называется методом балланса или интегрально-
интерполяционным методом.

Будем рассматривать этот подход на примере одномерного стационарного уравнения диффузии.

\begin{align}
  \label{25-eq}
  &-(k(x)u'(x))' = f(x), 0 < x < l\\
  \nonumber
  &k(x) \geqslant k_1 > 0, k_1 = \const
\end{align}

В качестве граничных условий на левой границе возьмём условие третьего рода
\begin{equation}
  \label{25-initial}
  -k(0)u'(0) + \sigma u(0) = g_1
\end{equation}
на правой границе - условие Дирихле.
\begin{equation}
  u(l) = g_2
\end{equation}
Введём обозначения
\begin{equation}
  q \equiv -k(x)u'(x)
\end{equation}

Будем строить разностную схему на равномерной сетке $\overline{\omega_h}$.
Проинтегрируем
уравнение \eqref{25-eq} на отрезке $x_{i - \frac{1}{2}} \leqslant x
\leqslant x_{i + \frac{1}{2}}$.
\begin{equation}
  \label{25-int}
  q_{i + \frac{1}{2}} - q_{i - \frac{1}{2}} = \int\limits_{x_{i - \frac{1}{2}}}^
  {x_{i + \frac{1}{2}}}f(t)dt
\end{equation}

\eqref{25-int} отражает закон сохранения количества тепла для отрезка
$[x_{i - \frac{1}{2}}, x_{i + \frac{1}{2}}]$. Величина $q_{i - \frac{1}{2}}$ даёт
количество тепла, вытекшего через сечение $x_{i - \frac{1}{2}}$, а $q_{i +\frac{1}{2}}$
- количество тепла, втекающее через сечение $x_{i + \frac{1}{2}}$. Дисбаланс этих
потоков обусловлен распределением источников тепла.

Из определения потока находим
\begin{equation}
  \label{25-diff}
  u' = - \dfrac{q}{k}
\end{equation}

Интегрируем \eqref{25-diff} на $[x_{i - 1}, x_i]$
\begin{equation}
  \label{25-int2}
  u_i - u_{i - 1} = -\int\limits_{x_{i - 1}}^{x_i}\dfrac{q}{k}dx =
  -q_{i - \frac{1}{2}}\int\limits_{x_{i - 1}}^{x_i}\dfrac{dx}{k}
\end{equation}

Введём обозначение
\begin{equation}
  \label{25-def}
  a_i \equiv \left(\dfrac{1}{h}\int\limits_{x_{i - 1}}^{x_i}\dfrac{dx}{k}\right)^{-1}
\end{equation}

Из \eqref{25-int2}, \eqref{25-def}
\begin{equation}
  \label{25-q}
  \begin{split}
    &q_{i - \frac{1}{2}} = -a_i\dfrac{u_i - u_{i - 1}}{h} = -a_i\cdot
    u_{\overline{x}, i}\\
    &q_{i + \frac{1}{2}} = -a_{i + 1}u_{x, i}
  \end{split}
\end{equation}

Если применим формулу средних прямоугольников, по получим $a_i = k(x_i - \dfrac{h}{2})$.

Если применим формулу трапеций, то получим $a_i = \dfrac{1}{2}(k_i + k_{i - 1})$.
Погрешность аппроксимации: $O(h^2)$.

Подставляем \eqref{25-q} в \eqref{25-int}, получим
\begin{align}
  &-(au_{\overline{x}})_x = \phi\\
  &\phi = \dfrac{1}{h}\int\limits_{x_{i - 1}}^{x_i}fdx
\end{align}

На правой границе было условие Дирихле
\begin{equation}
  u(l) = g_2
\end{equation}

Для аппроксимации значения на левой границе используем интегрально-интерполяционный
подход. Интегрируем уравнение \eqref{25-eq} на отрезке $\left[0, \dfrac{h}{2}\right]$.

\begin{align}
  &q_{\frac{1}{2}} - q_0 = \int\limits_0^{x_{\frac{1}{2}}}fdx\\
  &q_{\frac{1}{2}} = -a_1u_{x, 0} \text{ - аппроксимация в дробном узле}\\
  &\eqref{25-initial} \Rightarrow q_0 = g_1 - \sigma u_0
\end{align}

В итоге получаем следующую разностную аппроксимацию граничных условий \eqref{25-initial}
\begin{align}
  &-a_1u_{x, 0} + \sigma u_0 = g_1 + \dfrac{h}{2}\phi_0\\
  &\phi_0 = \dfrac{1}{h}\int\limits_0^{\frac{h}{2}}fdx
\end{align}

Аналогично строится $\phi$ на правой границе (там будут заданы условия второго или
третьего рода).

\newpage
\begin{col-answer-preambule}
	\begin{plan}
    \item пока пусто
	\end{plan}
\end{col-answer-preambule}

\colquestion{Разностные схемы повышенного порядка аппроксимации}

\begin{notes}
  \item пока пусто
\end{notes}

\newpage
\begin{col-answer-preambule}
	\begin{plan}
    \item пока пусто
	\end{plan}
\end{col-answer-preambule}

\colquestion{Разностные схемы для уравнения Пуассона}

Будем рассматривать задачу Дирихле для двумерного уравнения диффузии
\begin{align}
  \label{27-problem}
  &Lu = f(x), x = (x_1, x_2) \in \Omega\\
  \label{27-border}
  &u(x) = -g(x), x \in \sigma\Omega\\
  &Lu = \sum\limits_{\alpha = 1}^2L_{\alpha}u, L_{\alpha}u = -\dfrac{\partial}
  {\partial x_{\alpha}}\left(k(x)\dfrac{\partial u}{\partial x_{\alpha}}\right)
\end{align}

\begin{align*}
  \Omega = \set{x | x = (x_1, x_2), 0 \leqslant x_{\alpha} \leqslant l_{\alpha},
  \alpha = 1, 2}
\end{align*}

Поставим в соответствие разностную схему
\begin{align}
  \label{27-scheme1}
  &\Lambda y = \phi(x), x \in \omega, \omega \text{ - множество узлов}\\
  &y(x) = g(x), x \in \sigma\Omega\\
  \label{27-scheme2}
  &\Lambda y = \sum\limits_{\alpha = 1}^2\Lambda_{\alpha}y, \Lambda_{\alpha}y
  = -(a_{\alpha}y_{\overline{x}_{\alpha}})x_{\alpha}\\
  \nonumber
  &\overline{\omega} = \set{x | x = x_{ij} = (ih_1, jh_2), i = \overline{0, N_1},
  j = \overline{0, N_2}, h_1N_1 = l_1, h_2N_2 = l_2}
\end{align}

\begin{equation}
  \begin{split}
    &a_1(x_1, x_2) = k(x_1 - 0.5h_1, x_2)\\
    &a_2(x_1, x_2) = k(x_1, x_2 - 0.5h_2)
  \end{split}
\end{equation}

Если коэффициенты $k(x)$ и решение задачи \eqref{27-problem} - \eqref{27-border} являются
гладкими функциями, то схема \eqref{27-scheme1} - \eqref{27-scheme2} имеет второй порядок
аппроксимации $O(h_1^2 + h_2^2)$.

Рассмотрим оператор $Lu$ в \eqref{27-problem}, который содержит смешанные производные
\begin{equation}
  Lu = \sum\limits_{\alpha, \beta = 1}^{2}L_{\alpha \beta}u, \hspace{5mm} L_{\alpha \beta}u = -\dfrac{\partial}
  {\partial x_{\alpha}}\left(k_{\alpha\beta}(x)\dfrac{\partial u}{\partial x_{\beta}}\right)
\end{equation}

\begin{align*}
  \Lambda y = \sum\limits_{\alpha, \beta = 1}^2\Lambda_{\alpha\beta}y
\end{align*}
\begin{align}
  &\Lambda_{\alpha\beta}y = -\dfrac{1}{2}((k_{\alpha\beta y_{\overline{x}\beta}})_
  {x_{\alpha}} + (k_{\alpha\beta}y_{x\beta})_{\overline{x}_{\alpha}})\\
  &\Lambda_{\alpha\beta} = \dfrac{1}{2}\left(\Lambda_{\alpha\beta}^- +
  \Lambda_{\alpha\beta}^+\right);
  \Lambda_{\alpha\beta}^- = -(k_{\alpha\beta y_{\overline{x}\beta}})_{x_{\alpha}},
  \Lambda_{\alpha\beta}^+ = (k_{\alpha\beta}y_{x\beta})_{\overline{x}_{\alpha}})
\end{align}

$\Lambda_{\alpha\beta}^- + \Lambda_{\alpha\beta}^+$ имеют первый порядок
аппроксимации.

\begin{equation}
  \begin{split}
    &\Lambda_{11}y = -\dfrac{1}{2}((k_{11}y_{\overline{x_1}})_{x_1} +
    (k_{11}y_{x_1})_{\overline{x_1}}) = -(a_{11}y_{\overline{x_1}})_{x_1}\\
    &\Lambda_{2}y = -\dfrac{1}{2}((k_{2}y_{\overline{x_2}})_{x_2} +
    (k_{22}y_{x_2})_{\overline{x_2}}) = -(a_{22}y_{\overline{x_2}})_{x_2}
  \end{split}
\end{equation}

\begin{equation}
  \begin{split}
    &a_{11}(x_1, x_2) = \dfrac{1}{2}(k_{11}(x_1 - h_1, x_2) + k_{11}(x_1, x_2))\\
    &a_{22}(x_1, x_2) = \dfrac{1}{2}(k_{22}(x_1, x_2 - h_2) + k_{22}(x_1, x_2))
  \end{split}
\end{equation}
\begin{align*}
  \Lambda_{\alpha\alpha}u - L_{\alpha\alpha}u = O(\abs{h}^2) = O(h_1^2 + h_2^2)
  = O(h^2) \text{ - погрешность}
\end{align*}

С разными индексами:
\begin{align}
  &\Lambda_{12}^-y = -(k_{12}y_{\overline{x_2}})_{x_1}\\
  \label{27-diff-left}
  &u_{\overline{x_2}} = \dfrac{\partial u}{\partial x_2} - \dfrac{h_2}{2}
  \dfrac{\partial^2 u}{\partial x_2^2} + O(h_2^2) \text{ - левая разностная
  производная}\\
  \label{27-diff-right}
  &\delta_{x_1} = \dfrac{\partial \delta}{\partial x_1} + \dfrac{h_1}{2}
  \dfrac{\partial^2 \delta}{\partial x_1^2} + O(h_1^2) \text{ - правая разностная
  производная}
\end{align}

В качестве $\delta = k_{12}u_{\overline{x_2}}$. Подставим \eqref{27-diff-left} в
\eqref{27-diff-right}
\begin{align}
  \label{27-approx1}
  &\Lambda_{12}^-u = L_{12}u + \dfrac{h_1}{2}\dfrac{\partial}{\partial x_1}L){12}u
  - \dfrac{h_1}{2}\dfrac{\partial}{\partial x_2}L_{12}u + O(h^2)\\
  \label{27-approx2}
  &\Lambda_{12}^+u = L_{12}u - \dfrac{h_1}{2}\dfrac{\partial}{\partial x_1}L){12}u
  + \dfrac{h_1}{2}\dfrac{\partial}{\partial x_2}L_{12}u + O(h^2)\\
  &\eqref{27-approx1}, \eqref{27-approx2} \Rightarrow \Lambda_{12} - L_{12}u = O(h^2)
\end{align}

\begin{note}
  Можно использовать другой семиточечный шаблон
  \begin{align*}
    \Lambda_{\alpha\beta}y = -\dfrac{1}{2}((k_{\alpha\beta y_{x\beta}})_
    {x_{\alpha}} + (k_{\alpha\beta}y_{\overline{x}\beta})_{\overline{x}_{\alpha}})
  \end{align*}
\end{note}

Рассмотрим оператор $\Lambda$, когда $k(x) = 1$:
\begin{align}
  \label{27-op1}
  &\Lambda u = Lu - \dfrac{h_1^2}{12}L_1^2u - \dfrac{h_2^2}{12}L_2^2u + O(h^2)\\
  \label{27-op2}
  &L_1^2u = L_1f - L_1L_2u;\qquad L_2^2u = L_2f - L_1L_2u
\end{align}

Подставляем \eqref{27-op2} в \eqref{27-op1}, получаем
\begin{equation}
  \Lambda u = Lu - \dfrac{h_1^2}{12}L_1f - \dfrac{h_2^2}{12}L_2f +
  \dfrac{1}{12}(h_1^2 + h_2^2)L_1L_2u + O(h^4)
\end{equation}

Заменяя $L_1L_2u$ разностным выражением получаем
\begin{align}
  &L_1L_2u \approx \Lambda_1\Lambda_2 u = u_{\overline{x_1}x_1\overline{x_2}x_2}\\
  \label{27-puasson1}
  &\Lambda_1y + \Lambda_2y - \dfrac{1}{12}(h_1^2 + h_2^2)\Lambda_1\Lambda_2y = \phi(x)\\
  \label{27-puasson2}
  &\phi(x) = f(x) + \dfrac{1}{12}h_1^2f_{\overline{x_1}x_1} + \dfrac{1}{12}h_2^2
  f_{\overline{x_2}x_2}
\end{align}

\eqref{27-puasson1}, \eqref{27-puasson2} аппроксимирует уравнение Пуассона на решении
с четвёртым порядком аппроксимации. ${u \in C^6(\Omega), f \in C^4(\Omega)}$ - гладкие
функции.

\newpage
\begin{col-answer-preambule}
	\begin{plan}
    \item пока пусто
	\end{plan}
\end{col-answer-preambule}

\colquestion{Аппроксимация краевых условий 2-го и 3-го рода}

\begin{notes}
  \item пока пусто
\end{notes}

\newpage
\begin{col-answer-preambule}
	\begin{plan}
    \item пока пусто
	\end{plan}
\end{col-answer-preambule}

\colquestion{Монотонные разностные схемы}

\begin{notes}
  \item пока пусто
\end{notes}

\newpage
\begin{col-answer-preambule}
	\begin{plan}
    \item пока пусто
	\end{plan}
\end{col-answer-preambule}

\colquestion{Явная левостороняя схема для уравнения переноса}

\begin{notes}
  \item пока пусто
\end{notes}

\newpage
\begin{col-answer-preambule}
	\begin{plan}
    \item пока пусто
	\end{plan}
\end{col-answer-preambule}

\colquestion{Неявная левостороняя схема для уравнения переноса}

\begin{notes}
  \item пока пусто
\end{notes}

\newpage
\begin{col-answer-preambule}
	\begin{plan}
    \item пока пусто
	\end{plan}
\end{col-answer-preambule}

\colquestion{Начальная краевая задача для уравнения переноса}

\begin{notes}
  \item пока пусто
\end{notes}

\newpage
\begin{col-answer-preambule}
	\begin{plan}
    \item пока пусто
	\end{plan}
\end{col-answer-preambule}

\colquestion{Явная схема для уравнения теплопроводности}

\begin{notes}
  \item пока пусто
\end{notes}

\newpage
\begin{col-answer-preambule}
	\begin{plan}
    \item пока пусто
	\end{plan}
\end{col-answer-preambule}

\colquestion{Шеститочечная схема для уравнения теплопроводности}

\begin{notes}
  \item пока пусто
\end{notes}

\newpage
\end{document}
