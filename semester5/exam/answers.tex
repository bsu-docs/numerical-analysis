\documentclass[a4paper]{article}
\usepackage[utf8]{inputenc}
\usepackage[russian]{babel}
\usepackage{mathtools}
\usepackage{amssymb}
\usepackage{amsthm}
\usepackage{tikz}
\usepackage{multicol}
\usepackage{xparse}
\usepackage{enumitem}
\usepackage{centernot}
\usepackage{comment}

\usepackage[left=1cm,right=1cm,top=1cm,bottom=1cm]{geometry}

\usepackage{setspace}
%\doublespace
\usepackage{epstopdf}
\usepackage{graphicx}
\usepackage{titlesec}

\newtheorem*{theorem}{Теорема}
\newtheorem*{lemma}{Лемма}

\newcommand{\norma}{}
\usepackage{mainstyle}
\renewcommand{\theenumii}{\asbuk{enumii}}

% For cyrillic symbols in "enumerate" environment.
\renewcommand{\theenumii}{\asbuk{enumii}}
\AddEnumerateCounter{\asbuk}{\@asbuk}{ы}


%====================================================================================
%   ENVIORONMENTS

\newenvironment{definition}
{\begin{statement}{Определение}}
    {\end{statement}}

\newenvironment{characteristics}
{\begin{statementItemed}{Свойства}}
    {\end{statementItemed}}

\newenvironment{note}
{\begin{statementDotted}{Замечание}

        }
    {\end{statementDotted}}

\newenvironment{notes}
{\begin{statementItemed}{Замечания}}
    {\end{statementItemed}}

\newenvironment{proofUndotted}
{{\raggedright \textit{Доказательство}}}
{\begin{flushright}
       \boxed{}
 \end{flushright}
}

\newenvironment{noteo}{}{}
\RenewDocumentEnvironment{noteo}{o}
{{\raggedright \textbf{Замечание}\IfValueTF{#1}{ (\textit{#1})}{}.}$  $

}
{}

\newenvironment{consequence}{}{}
\RenewDocumentEnvironment{consequence}{o}
{{\raggedright \textbf{Следствие}\IfValueTF{#1}{ (\textit{#1})}{}.}$  $

    }
{}

\NewDocumentEnvironment{consequences}{o}
{\raggedright \textbf{Следствия}\IfValueTF{#1}{(\textit{#1})}{}:\begin{enumerate}}
    {\end{enumerate}}


\RenewDocumentEnvironment{theorem}{o}
{{\raggedright
  \textbf{Теорема}\IfValueTF{#1}{ (\textit{#1})}{}}.$  $

}

\NewDocumentEnvironment{plan}{o}
{\raggedright \textbf{Краткий план}\IfValueTF{#1}{(\textit{#1})}{}:\begin{enumerate}}
	{\end{enumerate}}

\newenvironment{theoremNamed}{}{}
\RenewDocumentEnvironment{theoremNamed}{m o}
{{\raggedright
  \textbf{Теорема #1}\IfValueTF{#2}{ (\textit{#2})}{}}.$  $

}


\RenewDocumentEnvironment{lemma}{o}
{{\raggedright
        \textbf{Лемма}\IfValueTF{#1}{ (\textit{#1})}{}}.$  $

}

\newenvironment{lemmaNamed}{}{}
\RenewDocumentEnvironment{lemmaNamed}{m o}
{{\raggedright
        \textbf{Лемма #1}\IfValueTF{#2}{ (\textit{#2})}{}}.$  $

}


\newenvironment{exercise}
{\begin{statementDotted}{Упражнение}}
    {\end{statementDotted}}

\newenvironment{example}
{\begin{statementDotted}{Пример}}
    {\end{statementDotted}}

\newenvironment{examples}
{\begin{statementItemed}{Примеры}}
    {\end{statementItemed}}

% =================================
% math functions
\newcommand{\abs}[1]{
    \left\lvert #1 \right\rvert
}

\newcommand{\maxf}[1]{
    \max \left\{ #1 \right\}
}

\newcommand{\suchthat}{
    \;\ifnum\currentgrouptype=16 \middle\fi|\;
}

\newcommand{\arc}[1]{
    \buildrel\,\,\frown\over{#1}
}

\DeclareMathOperator{\const}{\text{const}}

\DeclareMathOperator{\fix}{\text{fix }}
\DeclareMathOperator{\diam}{diam\,}
\DeclareMathOperator{\mes}{mes}
\DeclareMathOperator{\divergence}{div}

\DeclareMathOperator{\arcsh}{arcsh}
\DeclareMathOperator{\arth}{{arth}}
\DeclareMathOperator{\arcth}{{arcth}}
\DeclareMathOperator{\sgn}{sgn}

\newcommand{\parenthesis}[1]{%
    \left( #1 \right)
}

\newcommand{\plot}[1]{$ \text{Г}_{#1} $}

\newcommand{\limlim}[2]
{ \lim\limits_{ \substack{ #1 \\ #2 } } }

\newcommand{\limitslimits}[2]
{ \limits_{ \substack{ #1 \\ #2 } } }

\newcommand{\limlimlim}[3]
{ \lim\limits_{ \substack{ #1 \\ #2 \\ #3 } } }

\newcommand{\nullFrac}{\dfrac{ }{}}

\renewcommand{\emptyset}{\varnothing}

\newcommand{\diint}{\displaystyle\iint}

\newcommand{\liml}{\lim\limits}
\newcommand{\intl}{\int\limits}
\newcommand{\iintl}{\iint\limits}
\newcommand{\diintl}{\diint\limits}
\newcommand{\dintl}{\dint\limits}

\newcommand{\suml}{\sum\limits}
\newcommand{\sumnzi}{\sum\limits_{n=0}^{\infty}}
\newcommand{\limninf}{\lim\limits_{n\to\infty}}
\newcommand{\dintlzi}{\dintl_0^{+\infty}}


%========================================================
% text style

\newcommand{\important}[1]{\textit{#1}}

\newcommand{\dint}{\displaystyle\int}
\newcommand{\dsum}{\displaystyle\sum}

\newcommand{\oiint}[2]{
    \begin{tikzpicture}[baseline=(C.base)]
        \node(C) {$ \displaystyle \iintl_{#1}^{#2} $};
        \draw (0,0.15) circle (0.25);
        %\node[draw,circle,inner sep=1pt](C) ++ (0, 0.1) {$ \;\;\;\; $};
    \end{tikzpicture}
}

\newcommand{\circled}[1]{
    \begin{tikzpicture}[baseline=(C.base)]
    \node[draw,circle,inner sep=1pt](C) {#1};
    \end{tikzpicture}
}

\newcommand{\eqlhopital}{%
    \overset{\circled{Л}}{=}}

\newcommand{\neqlhopital}{%
    \overset{\circled{Л}}{\neq}}

\newcommand{\sqcase}[1]{%
    \left[\begin{matrix}#1\end{matrix}\right]
}

\newcommand{\Arg}{%
  \operatorname{Arg}
}

\newcommand{\Ln}{%
  \operatorname{Ln}
}

\newcommand{\Arsh}{%
  \operatorname{Arsh}
}

\newcommand{\Arch}{%
  \operatorname{Arch}
}

\newcommand{\Arth}{%
  \operatorname{Arth}
}

\newcommand{\Arcth}{%
  \operatorname{Arcth}
}

\newcommand{\Arcsin}{%
  \operatorname{Arcsin}
}

\newcommand{\Arccos}{%
  \operatorname{Arccos}
}

\newcommand{\Arctg}{%
  \operatorname{Arctg}
}

\newcommand{\Arcctg}{%
  \operatorname{Arcctg}
}

\renewcommand{\th}{%
  \operatorname{th}
}

\renewcommand{\cth}{%
  \operatorname{cth}
}


\newcommand{\dvert}{\left.\nullFrac\right\vert}

\renewcommand{\r}[1]{$\overset{\text{ }_\bullet\text{ }}{\text{#1}}$}

\renewcommand{\norma}[1]{\left\lvert\left\lvert#1\right\rvert\right\rvert}
\newcommand{\norm}[1]{\left\lvert\left\lvert#1\right\rvert\right\rvert}

\newcommand{\RN}{\mathbb{R}^n}
\newcommand{\R}[1]{\mathbb{R}^{#1}}

% =================================
% sets utilities

\newcommand{\defineset}[2]{
    \left\{ #1 \, \middle\vert \, #2 \right\}
}

\newcommand{\set}[1]{
    \left\{ #1 \right\}
}

\newcommand{\colquestion}[1]{\section{#1}}
\newenvironment{col-answer-preambule}
               {\ignorespaces}
               {\ignorespacesafterend}

\begin{document}
\begin{center}
  \LARGE\underline{\textbf{Ответы к экзамену по курсу}}\\
  \LARGE\underline{\textbf{ ``Методы Численного анализа''}}\\
  \Large\textbf{(1-ый семестр 2016/2017 учебного года, специальность ``Информатикa'')}
\end{center}

{
  % \renewcommand{\contentsname}{Содержание}
  \tableofcontents
}
\newpage
\begin{col-answer-preambule}
	\begin{plan}
    \item пока пусто
	\end{plan}
\end{col-answer-preambule}

\colquestion{Интерполяционный многочлен Лагранжа. Оценка погрешности}

\begin{notes}
  \item пока пусто
\end{notes}

\newpage
\begin{col-answer-preambule}
	\begin{plan}
    \item пока пусто
	\end{plan}
\end{col-answer-preambule}

\colquestion{Оценка погрешности на равномерной сетке узлов}

\begin{notes}
  \item пока пусто
\end{notes}

\newpage
\begin{col-answer-preambule}
	\begin{plan}
    \item пока пусто
	\end{plan}
\end{col-answer-preambule}

\colquestion{Разделённые разности и их свойства}

\begin{notes}
  \item пока пусто
\end{notes}

\newpage
\begin{col-answer-preambule}
	\begin{plan}
    \item пока пусто
	\end{plan}
\end{col-answer-preambule}

\colquestion{Интерполяционный многочлен Ньютона}

Интерполяционный многочлен Ньютона представляет собой другую форму записи интерполяционного многочлена.

Она полезна, т.к. позволяет легко увеличивать или уменьшать число использованных узлов без повторных числений.

Пусть есть набор точек $(x_i, f_i), i = \overline{0, n}$. Построим интерполяционный многочлен для этой сеточной функции:
\begin{equation*}
  P_n(x) = f_0 + (x - x_0) P_{n - 1}(x) \Rightarrow P_{n-1} (x) = \dfrac{P_n(x) - f_0}{x - x_0},
\end{equation*}
где $x_1, \ldots, x_n$ в качестве $x$. Для задания $P_{n-1}(x)$ нужно $n$ коэффициентов = $n$ уравений.
\begin{equation*}
  P_{n-1}(x_i) = \dfrac{P_n(x_i) - f_0}{x_i - x_0} = \dfrac{f_i - f_0}{x_i - x_0}, i = \overline{1, n}.
\end{equation*}
Значит, искомый полином проходит через точки $\left( x_i, \dfrac{f_i - f_0}{x_i - x_0} \right)$, где $\dfrac{f_i - f_0}{x_i - x_0}$ - разделенная разность первого порядка.

Обозначим $\dfrac{f_i - f_0}{x_i - x_0} = f(x_0, x_i)$.

Значит, $P_{n-1}(x)$ можно представить в виде:
\begin{equation*}
  P_{n-1}(x) = f(x_0, x_1) + (x - x_1)P_{n-2}(x).
\end{equation*}

Для нахождения $P_{n-2} (x)$ проделаем те же действия, что и для $P_{n-1} (x):$

\begin{equation*}
  P_{n-2}(x) = \dfrac{P_{n-1}(x) - f(x_0, x_1)}{x - x_1}
\end{equation*}

\begin{equation*}
  \dfrac{f(x_1, x_i) - f(x_0, x_1)}{x_i - x_1} = f(x_0, x_1, x_i), i = \overline{2, n}.
\end{equation*}

\begin{equation}\label{eq:begin_table}
  P_n(x) = f_0 + (x - x_0) f(x_0, x_1) + (x - x_0)(x - x_1)f(x_0, x_1, x_2) + \ldots + (x-x_0)(x-x_1)\ldots(x-x_{n-1})f(x_0, x_1,\ldots, x_n)
\end{equation}

Погрешность интерационного многочлена через разделенные разности:
\begin{equation*}
  f(x) - P_n(x) = f(x) - \sum\limits_{i=0}^n f_i \prod\limits_{j=0, j \ne i}^n \dfrac{x - x_j}{x_i - x_j} = \underbrace{\prod\limits_{j=0}^n (x - x_j)}_{w(x)} \underbrace{\left[ \dfrac{f(x)}{\prod\limits_{j=0}^n (x - x_j)} + \sum\limits_{i=0}^n \dfrac{f_i}{(x_i - x)\prod\limits_{j=0, j \ne i}^n (x_i - x_j)}  \right]}_{f(x, x_0, x_1, \ldots, x_n)} =
\end{equation*}
\begin{equation*}
  = \left[ \text{для } f(x, x_0, x_1, \ldots, x_n) \text{ см Лемму о представлении РР в виде суммы в пред вопросе} \right] = 
\end{equation*}
\begin{equation*}
  = w(x) f(x, x_0, x_1, \ldots, x_n) = \dfrac{1}{(n+1)!} f^{(n+1)} (\xi) w(x).
\end{equation*}
Отсюда получаем, что $f(x, x_0, x_1, \ldots, x_n) = \dfrac{f^{(n+1)} (\xi)}{(n+1)!}$.

\begin{notes}
  \item При построении формулы Ньютона порядок расположения узолв $x_0, \ldots, x_n$ значения не имеет. В качестве точки $x_0$ мы можем взять и точу $x_n$. Пусть $x_0 := x_n$. Тогда:
  \begin{equation}\label{eq:end_table}
    P_n(x) = f_n + (x - x_n) f(x_n, x_{n-1}) + (x - x_n) (x - x_{n-1}) f(x_n, x_{n-1}, x_{n-2}) + \ldots + (x - x_n)(x - x_{n-1}) \cdot \ldots \cdot (x - x_1)f(x_n, \ldots, x_0).
  \end{equation}
  Если узлы упорядочены по возрастанию $x_0 < x_1 < \ldots < x_n$, то формула записи \eqref{eq:begin_table} носит название формула записи Ньютона для начала таблицы, а \eqref{eq:end_table} - форма записи Ньютона для конца таблицы.
  \item Хотя теоретически нет необходимости упорядочивать множество узлов $\{ x_i \} $ по возрастанию и убыванию, гладкость в таблице разделенных разностей нарушается, если такого порядка нет. Поэтому, программируя, используется упорядоченное множество сетки.
\end{notes}

\newpage
\begin{col-answer-preambule}
	\begin{plan}
    \item пока пусто
	\end{plan}
\end{col-answer-preambule}

\colquestion{Конечные разности и их свойства}

Основным оператором в исчислении конечных разностей является оператор $\Delta$, который определяется как:
\begin{equation*}
  \Delta f(x) = f(x + h) - f(x).
\end{equation*}

Из определения следует, что оператор $\Delta$ - линейный, т.е.
\begin{equation*}
  \Delta \left[ a f(x) + bg(x) \right] = a \Delta f(x) + b \Delta g(x).
\end{equation*}

Исследуем $\Delta \left[ f(x) g(x) \right]$.
\begin{equation*}
  \Delta \left[ f(x) g(x) \right] = \left[ def. \right ] = f(x+h) g(x+h) - f(x)g(x) =
\end{equation*}
\begin{equation*}
  = f(x+h)g(x+h) + f(x+h)g(x) - f(x+h)g(x) - f(x)g(x) = f(x+h) \Delta g(x) + g(x) \Delta f(x).
\end{equation*}

\begin{equation*}
  \Delta \left(\dfrac{f(x)}{g(x)} \right) = \left[ def. \right] = \dfrac{f(x+h)}{g(x+h)} - \dfrac{f(x)}{g(x)} = \dfrac{f(x+h)g(x) - f(x)g(x+h)}{g(x+h)g(x)} =
\end{equation*}
\begin{equation*} 
  = \dfrac{1}{g(x+h)g(x)} \Big[ f(x+h)g(x) - g(x+h)f(x) + f(x)g(x)-f(x)g(x) \Big] = \dfrac{1}{g(x+h)g(x)} \cdot \Big[\Delta f(x)g(x) - \Delta g(x)f(x) \Big]
\end{equation*}

В формуле выше $f$ и $g$ можно поменять местами.

\begin{equation*}
  \Delta \left (\Delta f \right) = \Delta^2 f.
\end{equation*}

\begin{equation*}
  \Delta^n f(x) = \Delta \left( \Delta^{n-1} f(x) \right) = \Delta^{n-1} \left( \Delta f(x) \right).
\end{equation*}

\begin{theorem}[основная теорема исчисления конечных разностей]
  Для многочлена степени $n$:
  \begin{equation*}
    f(x) = \alpha_0 + \alpha_1 x + \ldots + \alpha_n x^n,
  \end{equation*}
  где $\alpha_n \ne 0$, конечная разность $n$-ого порядка равна:
  \begin{equation*}
    \Delta^n f = \alpha_n n! h^n, \Delta^{n+1} f = 0.
  \end{equation*} 
\end{theorem}

\begin{lemma}
  Если $f(x)$ - многочлена степени $n$, то $\Delta f$ есть многочлен степени $n-1$.
\end{lemma}
\begin{proof}
  Рассмотрим в качестве $f = x^n$, тогда
  \begin{equation*}
    \Delta f(x) = (x+h)^n - x^n = \sum\limits_{k=0}^n C_n^k h^{n-k}x^k - x^n = n h x^{n-1} + \dfrac{n (n-1)}{1 \cdot 2} h^2 x^{n-2} + \ldots + h^n,
  \end{equation*}
  т.е. $\Delta f(x)$ - полином степени $n-1$.

  Т.о., $\Delta x^n$ является полиномом степени $n-1$.

  Используя свойство линейности, определяем, что оператор $\Delta$ уменьшает степень каждого члена полинома на 1.

  Кроме того, член $n x^{n-1} h \alpha_n$, которые получаем в результате применения $\Delta$ к последнему слагаемому, отличен от нуля, т.к. $\alpha_n \ne 0$.
\end{proof}
\begin{proof}
  теоремы перед леммой. Применим к многочлену $n$-ой степени лемму $n$ раз и убедимся, что $n$-ая разность постоянна, а коэффициент при $\alpha_n$ - $\left(n! h^n \right)$, а все следующие разности превращаются в ноль.
\end{proof}

\newpage
\begin{col-answer-preambule}
	\begin{plan}
    \item пока пусто
	\end{plan}
\end{col-answer-preambule}

\colquestion{Интерполяционный многочлен Ньютона на равномерной сетке узлов}

Если функция задана на равномерной сетке узлов, это означает, что:
\begin{equation*}
  \begin{cases}
    \Delta x_i = x_{i + 1} - x_i = h \forall \; i, \\
    \Delta f_i = f_{i+1} - f_i.
  \end{cases}
\end{equation*}
\begin{equation*}
  \Delta^2 f_i = f_{i+2} - 2 f_{i+1} + f_i.
\end{equation*}
\begin{equation*}
  f(x_i) = f_i.
\end{equation*}

Эти конечные разности соответствуют разделенным разностям:
\begin{equation*}
  \underbrace{f(x_1, x_2)}_{\text{разделенная разность}} = [x_2, x_1] = \dfrac{f_2 - f_1}{x_2 - x_1} = \dfrac{f_2 - f_1}{h} = \underbrace{\dfrac{1}{h} \Delta f_1}_{\text{конечная разность}}.
\end{equation*}

Разделенная разность второго порядка:
\begin{equation*}
  f(x_1, x_2, x_3) = [x_3, x_2, x_1] = \dfrac{1}{x_3 - x_1} \Big( f(x_3, x_2) - f(x_1, x_2) \Big) = \dfrac{1}{2h} \left(\dfrac{\Delta f_2}{h} - \dfrac{\Delta f_1}{h} \right) = \dfrac{\Delta^2 f_1}{2! h^2}.
\end{equation*}

\begin{equation*}
  f(x_1, x_2, \ldots, x_n) = \dfrac{\Delta^{n-1} f}{(n-1)! h^{n-1}}.
\end{equation*}

Конечные разности можно использовать для аппроксимации конечных производных.

\begin{enumerate}
  \item $\Delta f_1 \sim h f^{'} \left(x_1 + \dfrac{h}{2} \right),$
  \item $\Delta^2 f_1 \sim h^2 f^{(2)} (x_1 + h)$,
  \item $\ldots$,
  \item $\Delta^n f_1 \sim h^n f^{(n)} (x_1 + \dfrac{n h}{2}).$
\end{enumerate}

Формула Ньютона для начала таблицы для равномерной сетки узлов имеет вид:
\begin{equation*}
  P_n(x) = f_0 + (x - x_0) \dfrac{\Delta f_0}{h} + \ldots + (x - x_0)(x - x_0 - h)\ldots(x - x_0 - (n-1)h) \dfrac{\Delta^n f_0}{n! h^n}.
\end{equation*}
\begin{equation*}
  t = \dfrac{x - x_0}{h} \Rightarrow x - x_0 = th, x - x_1 = x - x_0 - h = h(t-1).
\end{equation*}
\begin{equation*}
  P_n(x) = P_n (x_0 + th) = f_0 + \dfrac{t}{1!} \Delta f_0 + \dfrac{t (t-1)}{2!} \Delta^2 f_0 + \ldots + \dfrac{t (t-1) \ldots (t - n + 1)}{n!} \Delta^n f_0.
\end{equation*}
\begin{equation*}
  R_n(x) = h^{n+1} \dfrac{t(t-1)\ldots(t-n)}{(n+1)!} f^{(n+1)} (\xi).
\end{equation*}

Запишем представления разделённых разностей в конце таблицы:
$\begin{cases}
  f(x_n, x_{n-1}) = \dfrac{\Delta f_{n-1}}{1! n}, \\
  f(x_n, x_{n-1}, x_{n-2}) = \dfrac{\Delta^2 f_{n-2}}{2! h^2}, \\
  f(x_n, \ldots, x_0) = \dfrac{\Delta^n f_0}{n! h^n}.
\end{cases}$

\begin{equation*}
  t = \dfrac{x - x_n}{h},
\end{equation*}

\begin{equation*}
  P_n(x_n + th) = f_n + \dfrac{t}{1!} \Delta f_{n-1} + \dfrac{t (t + 1)}{2!} \Delta^2 f_{n-2} + \ldots + \dfrac{t (t + 1) \ldots (t + n - 1)}{n!} \Delta^n f_0,
\end{equation*}

\begin{equation*}
  R_n(x) = h^{n+1} \dfrac{t(t+1)\ldots (t+n)}{(n+1)!} f^{(n+1)} (\xi).
\end{equation*}

\newpage
\begin{col-answer-preambule}
	\begin{plan}
    \item пока пусто
	\end{plan}
\end{col-answer-preambule}

\colquestion{Многочлен Чебышева}

\begin{notes}
  \item пока пусто
\end{notes}

\newpage
\begin{col-answer-preambule}
	\begin{plan}
    \item пока пусто
	\end{plan}
\end{col-answer-preambule}

\colquestion{Минимизация остатка интерполирования}

\textit{Критерий}.\newline
Чебышев показал, что из всех полиномов степени $n$, $P_n(x)$ со старшим коэффициентом равным $1$, у полинома
\begin{equation*}
  2^{1-n} T_n(x) = \overline{T}(x).
\end{equation*}

точная верхняя грань абсолютных значений на $[-1, 1]$ наименьшая и равна $2^{1-n}$, т.к. $\max\limits_{[-1, 1]} |T_n(x)| = 1$.

$\left|\left| \overline{T_n} - 0 \right| \right|_{\infty, [-1, 1]} = 2^{1-n} \Rightarrow \overline{T_n} (x)$ - полином, наименее отклоняющийся от нуля.

В случае отрезка произвольной длины $x \in [a, b]$ сделаем линейную замену переменных, которая отображает $[a, b]$ на $[-1, 1]$.

\begin{equation}\label{eq:8_2}
  x = \dfrac{b-a}{2} t + \dfrac{a+b}{2} = \psi(t), t \in [-1, 1].
\end{equation}

\begin{equation*}
  P_n(x) = x^n + P_{n-1}(x) = \psi^n (t) + P_{n-1} \Big( \psi(t) \Big) = \left(\dfrac{b-a}{2}\right)^n \overline{P_n}(t).
\end{equation*}

\begin{equation*}
  \left|\left| P_n(x) \right| \right|_{\infty, [a, b]} = \left(\dfrac{b-a}{2}\right)^n \left|\left| \overline{P_n}(t) \right| \right|_{\infty, [-1, 1]} \geqslant 2^{1-n} \left( \dfrac{b-a}{2} \right)^n = (b-a)^n 2^{1-2n}.
\end{equation*}

Равенство в этой формуле достигается при
\begin{equation*}
  \overline{T_n}\overset{[a, b]}{(x)} = (b - a)^n 2^{1-2n} T_n \left( \dfrac{2x - a - b}{b - a} \right).
\end{equation*}

$\overline{T_n}\overset{[a, b]}{(x)}$ называется наименее отклоняющимся от нуля на отрезке $[a, b]$.

В силу замены переменных \eqref{eq:8_2} корни $\overline{T_n}\overset{[a, b]}{(x)}$ находятся по формуле:
\begin{equation*}
  x_m = \dfrac{a+b}{2} + \dfrac{b-a}{2} \cos \Big( \dfrac{\pi (m + \frac{1}{2})}{n} \Big), m = \overline{0, n-1}.
\end{equation*}

Для оценки остатка интерполирования функции $f(x)$ на Чебышевской сетке узов $\{ x_m \}$ будем использовать равномерную норму:
\begin{equation*}
  \left| \left| f(x) \right| \right|_{\infty} = \sup\limits_{[a, b]} |f(x)|.
\end{equation*}

Из представления остатка в общем виде следует
\begin{equation*}
  \left| \left| f(x) - P_{n-1}(x) \right| \right| \leqslant \dfrac{1}{n!} \left| \left| f^{(n)}(x) \right| \right|  \left| \left| w_n(x) \right| \right|.
\end{equation*}

Будем минимизировать правую часть в неравенстве выше.
\begin{equation*}
  w_n = (x - x_1) \ldots (x - x_n), \text{deg }w_n = n, \text{ старший коэффициент равен 1}.
\end{equation*}

Поэтому в качестве узлов интерполирования $x_1, \ldots, x_n$ мы можем взять корни многочлена Чебышева. В этом случае $w_n$ будет иметь вид:

\begin{equation*}
  w_n = \overline{T_n}\overset{[a, b]}{(x)} = (b - a)^n 2^{1-2n} T_n \left( \dfrac{2x - a - b}{b - a} \right).
\end{equation*}

Из равенства выше следует что $|| w_n (x) || = (b - a)^n 2^{1-2n} \Rightarrow$ на Чебышевском наборе узлов оценка погрешности интерполирования имеет вид:
\begin{equation*}
  \left| \left| f(x) - P_{n}(x) \right| \right| \leqslant \dfrac{1}{n!} \left| \left| f^{(n)}(x) \right| \right| (b - a)^n 2^{1-2n}.
\end{equation*}

Мы получили неуменьшаемую оценку погрешности интерполяции.

\newpage
\begin{col-answer-preambule}
	\begin{plan}
    \item пока пусто
	\end{plan}
\end{col-answer-preambule}

\colquestion{Интерполирование с кратными узлами}

Задача кратного интерполирования заключается в построении полинома наименьшей степени $n$, который удовлетворяет условиям:
\begin{equation}\label{eq:9_1}
  H_m^{(j)}(x_i) = f^{(j)}(x_i) = f_i^{(j)} (x_i) = f_i^{j}(1), i = \overline{0, n}, j = \overline{0, m - 1}.
\end{equation}

Все узлы $x_i$ будем считать различнымми. Через $m_i$ будем обозначать кратность $i$-того узла.

Полином $H_m(x)$, удовлетворяющий условию \eqref{eq:9_1}, называется полиномом Эрмита.

Число заданных условий в \eqref{eq:9_1} равно:
\begin{equation*}
  \sum\limits_{i=0}^n m_i = m + 1.
\end{equation*}

\begin{theorem}
  Интерполяционный полином $H_m(x)$, удолетворяющий условию \eqref{eq:9_1}, определяется единственным образом.
\end{theorem}

\begin{theorem}
  Полином Эрмита $H_m(x)$, определенный условиями \eqref{eq:9_1}, существует.
\end{theorem}

Рассмотрим погрешность интерполирования $R_m(x) = f(x) - H_m(x)$.

\begin{theorem}
  Пусть узлы $x_i, i = \overline{0, n}$ и точка $x \in [a, b], f(x) \in \mathbb{C}^{m + 1} [a, b]$.

  Тогда
  \begin{equation*}
    \exists \; \overline{x} \in [a, b] \; | \; R_m(x) = \dfrac{w_{m+1}(x)}{(m+1)!} f^{(m+1)}(\overline{x}),
  \end{equation*}
  где $w_{m+1}(x) = (x - x_0)^{m_0} \ldots (x - x_n)^{m_n}$.
\end{theorem}

Построим $H_m(x)$ на основе расширенной таблицы разделенных разностей. Введем набор узлов
\begin{equation*}
  x_{ij} = x_i + j \varepsilon, \varepsilon > 0, i = \overline{0, n}, j = \overline{0, m_i - 1}.
\end{equation*}

Очевидно, что все $x_{ij}$ различны и стремятся к $x_i$ при $\varepsilon \to 0$ (по построению).

Таблица разделенных разностей для расширенного набора узлов будет иметь вид:

\begin{center}
  \begin{tabular}{ | c | c | c | c |}
    \hline
    $f(x_{00})$          &                       &              &                \\ \hline
                        & $f(x_{00}, x_{01})   $ &              &                \\ \hline
    $f(x_{01})$          &                       &$f(x_{00}, x_{01}, x_{02})   $ &                  \\ \hline
                        & $f(x_{01}, x_{02} )  $ &              &                \\ \hline
    $\ldots$            &                 &               &  $f(x_{00}, \ldots, x_{n m_n - 1}   $  \\ \hline
    $f(x_{0 m_0-1})$     &       &  &                   \\ \hline
    $f(x_{1 0})$         &  &               & \\ \hline
    $f(x_{1 m_1-1})$     &       &  &                   \\ \hline
    $f(x_{x_n m_n - 1})$ &  &               &                 \\
      \hline
  \end{tabular}
\end{center}

Выражая разделенные разности через производные и переходя к пределу при $\varepsilon \to 0$, получаем:
\begin{equation*}
  \lim\limits_{\varepsilon \to 0} f(x_{il}, \ldots, x_{ik}) = \dfrac{f^{(k-l)} (x_i)}{(k-l)!}.
\end{equation*}

\newpage
\begin{col-answer-preambule}
	\begin{plan}
    \item пока пусто
	\end{plan}
\end{col-answer-preambule}

\colquestion{Интерполяционный сплайн второго порядка}

\begin{notes}
  \item пока пусто
\end{notes}

\newpage
\begin{col-answer-preambule}
	\begin{plan}
    \item пока пусто
	\end{plan}
\end{col-answer-preambule}

\colquestion{Интерполяционный кубический сплайн}

\begin{notes}
  \item пока пусто
\end{notes}

\newpage
\begin{col-answer-preambule}
	\begin{plan}
    \item пока пусто
	\end{plan}
\end{col-answer-preambule}

\colquestion{Наилучшее приближение в линейном векторном пространстве}

В ряде случаев функцию следует аппроксимировать (приближать) не путем интерполяции, а с помощью построения наилучшего приближения.

Пусть $H$ - линейное нормированное пространство.

Требуется найти наилучшие приближения элемента $f \in H$ с помощью ЛК $\sum\limits_{j=1}^n c_j g_{j}$, которые являются линейно-независимыми $g_j \in H, j = \overline{1, n}$.

Т.о., требуется найти элемент $\phi = \sum\limits_{j=1}^n \alpha_j g_j$ такой, что $\Delta = \left| \left| f - \phi \right| \right| = \inf\limits_{c_1, \ldots, c_n} \left| \left| f - \sum\limits_{j=1}^n c_j g_j \right| \right|$.

Если такой элемент существует $\phi \in H$ существует, то он называется элементом наилучшего приближения.

\begin{theorem}
  Элемент наилучшего приближения в линейном нормированном пространстве существует.
\end{theorem}

\textit{Замечание.}

Элемент наилучшего приближения может быть не единственным.

Пространство $H$ называется строго нормированным, если из условия $|| f + g|| = ||f|| + ||g||, ||f|| ||g|| \ne 0,$ следует $f = \alpha g, \alpha \ne 0$.

\begin{theorem}
  Если пространство $H$ строго нормированно, то элемент наилучшего приближения единственен.
\end{theorem}

\newpage
\begin{col-answer-preambule}
	\begin{plan}
    \item пока пусто
	\end{plan}
\end{col-answer-preambule}

\colquestion{Наилучшее приближение в гильбертовом пространстве}

Гильбертовое пространство является полным нормированным, $||x||^2 = (x_1, \ldots, x_n)$.

Для Гилбертова пространства $H$ элемент наилучшего приближения единственен и его построение сводится к решению системы линейных уравнений.

Обозначим $G = \text{span } \{ g_1, \ldots, g_n \}, g_i \in H, i = \overline{1, n}, f \in H$.

$|| f - \phi || = \inf\limits_{h \in G} || f - h||$.

$G$ называется линейным многообразием.

\begin{lemma}
  Пусть $\phi \in H$ - элемент наилучшего приближения. Тогда $(f - \phi) \perp G$ (ортогонален всем элементам).
\end{lemma}

\begin{lemma}
  Если погрешность $(f - \phi) \perp G$, где $G$ - линейная оболочка гилбертова пространства, то $\phi$ - элемент наилучшего приближения.
\end{lemma}

Пусть элемент наилучшего приближения $\phi$ имеет представление $\sum\limits_{j=1}^n \alpha_j g_{j}$. Коэффициенты $\alpha_j$ пока неизвестны:
\begin{equation*}
  f - \phi = f - \sum\limits_{j=1}^n \alpha_j g_{j}
\end{equation*}

\begin{equation}\label{eq:13_8}
  \left(f - \sum\limits_{j=1}^n \alpha_j g_{j}, g_i \right) \overset{\text{п. 1}}{=} 0,
\end{equation}
где $j = \overline{1, n}$ - свойство ортогональности, т.е. $(f - \phi) \perp G$.

\eqref{eq:13_8} - СЛАУ относительно $\alpha$.

Запишем систему \eqref{eq:13_8} в классической форме:
\begin{equation}\label{eq:13_9}
  \sum\limits_{j=1}^n \alpha_j (g_i, g_i) = (f, g_j), j = \overline{1, n}.
\end{equation}

Матричная система \eqref{eq:13_9} - матрица Грама. В силу того, что базисные элементы $g_1, \ldots, g_n$ линейно-независимы, определитель матрицы Грама $\ne 0 \Rightarrow$ \eqref{eq:13_9} имеет единственное решение относительно коэффициента $\alpha$.

\begin{notes}
  \item Если элементы $g_1, \ldots, g_n$ является ортонормированными, т.е. $(g_i, g_j) = \delta_{i, j}, i, j = \overline{1,n}$, то система \eqref{eq:13_8} имеет диагональную матрицу и решение находится как $d_j = (f, g_j), j = \overline{1, n}$. Тогда элемент наилучшего приближения $\phi = \sum\limits_{i=1}^n (f, g_i)g_i$.

  Коэффициент $\alpha_j$ имеет название коэффициента Фурье, а сам многочлен $\phi$ носит название многочлена Фурье.

  \item Если $(f, \phi) = || \phi ||^2$, тогда для элемента наилучшего приближения имеем
  \begin{equation*}
    ||f - \phi||^2 = ||f||^2 - ||\phi||^2.
  \end{equation*}

  В силу равенства Парсеваля $\left( ||f||^2 = \sum\limits_{k=1}^\infty |\alpha_k|^2 \right)$ имеем:

  \begin{equation*}
    ||f - \phi||^2 = \int\limits_{k = n + 1}^{
    \infty
    } |\alpha_k|^2
  \end{equation*}

  Значит, при $n \to \infty$ норма погрешности $|| f - \phi||$ неограниченно убывает, т.е. элемент наименьшего приближения $\phi$ среднеквадратичного сходится к $f$.

  \item Типичным примером гилбертова пространства является пространство $L_2 [a, b]$ - пространство функций $f(x)$, интегрируемых с квадратом на отрезке $[a, b]$, причем:

  \begin{equation*}
  (f, g)_{L_2} = \int\limits_a^b \rho(x) f(x) \overline{g(x)} dx,
  \end{equation*}

  \begin{equation*}
    \left| \left| f \right| \right|_{L_2}^2 = \left( \int\limits_a^b \rho(x) f^2(x) dx \right)^{\frac{1}{2}}
  \end{equation*}

  $\rho(x) \geqslant 0$ - весовая функция.

  $\rho(x) = 0$ на граничном числе точек, мера которого равна 0.

  \item Коэффициенты Фурье $\alpha_i$ дают наилучшие в системе наименьших квадратов приближения, когда $f(x)$ разлагается по ортогональному базису элементов $g_i$.

  Т.о., ортогональные функции $g_i$, нахождение коэффициентов Фурье и идея приближения в смысле наименьших квадратов тесно взаимосвязаны.
\end{notes}

\newpage
\begin{col-answer-preambule}
	\begin{plan}
    \item пока пусто
	\end{plan}
\end{col-answer-preambule}

\colquestion{Метод наименьших квадратов}

\begin{notes}
  \item пока пусто
\end{notes}

\newpage
\begin{col-answer-preambule}
	\begin{plan}
    \item пока пусто
	\end{plan}
\end{col-answer-preambule}

\colquestion{Метод Пикара и метод рядов Тейлора}

\begin{notes}
  \item пока пусто
\end{notes}

\newpage
\begin{col-answer-preambule}
	\begin{plan}
    \item пока пусто
	\end{plan}
\end{col-answer-preambule}

\colquestion{Методы Эйлера, трапеций, средней точки}

\begin{notes}
  \item пока пусто
\end{notes}

\newpage
\begin{col-answer-preambule}
	\begin{plan}
    \item пока пусто
	\end{plan}
\end{col-answer-preambule}

\colquestion{Сходимость метода Эйлера}

\begin{notes}
  \item пока пусто
\end{notes}

\newpage
\begin{col-answer-preambule}
	\begin{plan}
    \item пока пусто
	\end{plan}
\end{col-answer-preambule}

\colquestion{Методы последовательного повышения порядка точности}

\begin{notes}
  \item пока пусто
\end{notes}

\newpage
\begin{col-answer-preambule}
	\begin{plan}
    \item пока пусто
	\end{plan}
\end{col-answer-preambule}

\colquestion{Методы Рунге-Кутта}

\begin{notes}
  \item пока пусто
\end{notes}

\newpage
\begin{col-answer-preambule}
	\begin{plan}
    \item пока пусто
	\end{plan}
\end{col-answer-preambule}

\colquestion{Экстраполяционные методы Адамса}

\begin{notes}
  \item пока пусто
\end{notes}

\newpage
\begin{col-answer-preambule}
	\begin{plan}
    \item пока пусто
	\end{plan}
\end{col-answer-preambule}

\colquestion{Интерполяционные методы Адамса}

\begin{notes}
  \item пока пусто
\end{notes}

\newpage
\begin{col-answer-preambule}
	\begin{plan}
    \item пока пусто
	\end{plan}
\end{col-answer-preambule}

\colquestion{Усточивость линейных многошаговых методов}

\begin{notes}
  \item пока пусто
\end{notes}

\newpage
\begin{col-answer-preambule}
	\begin{plan}
    \item пока пусто
	\end{plan}
\end{col-answer-preambule}

\colquestion{Простейшие разностные операторы}

% \subsection{Основные понятия в теории разностных схем}
%
% Сеточные (разностные) методы основаны на переходе от функций непрерывного аргумента к
% функциям дискретного аргумента. Например, если на отрезке $[0, 1]$ ввести точки-узлы $x_i$,
% которые образуют множество
% \begin{equation*}
%   \omega_h = \set{x_i = ih, i = \overline{0, h}, nh = l}, h \text{ - шаг сетки}
% \end{equation*}
% то приближённое решение ищется в узлах сетки $\omega_h$ и обозначается $y_n(x_i),
% i = \overline{0, n}$ ($y_n$ - функция дискретного аргумента).
%
% Для нахождения этой сеточной функции формулируется разностная задача
% \begin{equation}
%   Lu = F
% \end{equation}
% где $u$ - искомая функция, $F$ - вектор правой части, содержащий входные данные задачи.
% Например, рассмотрим задачу Коши
%
% \begin{equation}
%   \begin{cases}
%     u' = f(x, u)\\
%     u(0) = u_0
%   \end{cases}
% \end{equation}
% соответствующая разностная задача записывается в следующем виде
% \begin{equation}
%   Lu = \begin{cases}
%     u' - 1, x > 0\\
%     u(0), x = 0
%   \end{cases}
%   F = \begin{cases}
%     0, x > 0\\
%     u_0, x = 0
%   \end{cases}
% \end{equation}
%
% Одна из аппроксимирующих эту задачу разностных схем имеет вид
% \begin{equation}
%   \begin{cases}
%     \dfrac{y_{i + 1} - y_i}{h} = f_i, i = \overline{0, n - 1}\\
%     y_0 = u_0
%   \end{cases}
% \end{equation}
% В операторной форме
% \begin{align}
%   &\Omega_ny_n = \phi_n \text{ - в общем случае}\\
%   &\Omega_ny_n = \begin{cases}
%     \dfrac{y_{i + 1} - y_i}{h}\\
%     y_0
%   \end{cases}
%   \phi_n = \begin{cases}
%     f_i\\
%     u_0
%   \end{cases}
% \end{align}
%
% Введём норму остаточной функции
% \begin{equation}
%   \abs{\abs{y_n}}_h = \underset{0 \leq}{max}
% \end{equation}

Область решения $\overline{\Omega} = [0, l]$.

Сетка узлов на этой области
\begin{align*}
  \overline{\omega_h} = \set{x_i = ih, i = \overline{0, n}, hn = l}
\end{align*}

Построим аппроксимацию производной
\begin{align*}
  Lu = u'
\end{align*}

Функцию $u$ будем считать достаточно гладкой: $u(x) \in C^k(\Omega), k > 2$. Поставим
в соответствие оператору $Lu$ разностный оператор $\Lambda_h$.

\begin{definition}
  Множество узлов сетки, которое используется для построения оператора $\Lambda_h$
  называется \textbf{шаблоном}.
\end{definition}

\textbf{Погрешность аппроксимации} оператора $Lu$ разностным оператором $\Lambda_h$
в $i$-ом узле определяется как
\begin{align*}
  \psi_i = \Lambda_hu_i - (Lu)_i
\end{align*}

Будем использовать разложение в ряд Тейлора в окрестности точки $x_i$, где
\begin{align*}
  u_{i\pm 1} = u_i \pm hu_i' + \dfrac{1}{2}h^2u_i'' \pm \dfrac{1}{6}h^3u_i''' + O(h^3)
\end{align*}

Используя это разложение можно построить разностную схему оператора левой разностной
производной:
\begin{align}
  \label{23-left}
  u_{\overline{x}} = \dfrac{u_i - u_{i - 1}}{h} = u_i' - \dfrac{h}{2}u_i'' + O(h^2)
\end{align}
Оператор правой разностной производной
\begin{align}
  \label{23-right}
  u_{\overline{x}} = \dfrac{u_{i + 1} - u_i}{h} = u_i' + \dfrac{h}{2}u_i'' + O(h^2)
\end{align}

Минимальный шаблон - 2 узла. $u_{\overline{x}, i + 1} = u_{x, i}$. Если будем
использовать шаблон из трёх узлов, то можно построить центральную разностную
производную
\begin{equation}
  \label{23-center}
  u_{\mathring{x}} = \dfrac{u_{i + 1} - u_{i - 1}}{2h} = \dfrac{1}{2}
  (u_{\overline{x}, i} + u_{x, i}) = u_i' + \dfrac{h^2}{6}u''' + O(h^3)
\end{equation}

Для оператора второй производной можно применить линейную комбинацию левой и правой
производной.
\begin{align}
  \label{23-2nd}
  &\nonumber Lu = u''\\
  &(u_{\overline{x}})_x = \dfrac{1}{h}(u_x - u_{\overline{x}}) = \dfrac{1}{h^2}
  (u_{i + 1} - 2u_i + u_{i - 1})
\end{align}

Погрешность аппроксимации оценивалась в отдельном $i$-ом узле. Для оценки на всей
сетке $\omega_h$ необходимо использовать сеточные нормы
\begin{equation}
  \begin{split}
    &\abs{\abs{\psi}}_{C,h} = \underset{x \in \omega_h}\max\abs{\psi(x)}\\
    &\abs{\abs{\psi}}_{2, h} = \left(\sum\limits_{x \in \omega_h}
    \psi^2(x)h\right)^{\frac{1}{2}}
  \end{split}
\end{equation}

\eqref{23-left} - \eqref{23-2nd} имеют одинаковый порядок аппроксимации. В общем
случае порядок аппроксимации может быть разным в различных сеточных нормах.

В качестве альтернативного подхода можно использовать определение производной как
решение интегралного уравнения и применения некоторой квадратурной формулы
\begin{align*}
  \dfrac{d^ku}{dx^k} = f(x)
\end{align*}
\begin{equation}
  \label{23-le}
  u(x) = \dfrac{1}{(k - 1)!}\int\limits_0^x(x - t)^{k - 1}f(t)dt
\end{equation}

Таким образом, можно определить производную как решение интегрального уравнения
\eqref{23-le} при известной функции $u(x)$.

\begin{example}
  $k = 1$.
  \begin{equation}
    \label{23-ex}
    u_{i + 1} - u_{i - 1} = \int\limits_{x_{i - 1}}^{x_{i + 1}}f(t)dt
  \end{equation}
  Если для вычисления \eqref{23-ex} использовать формулу центральных
  прямоугольников, то получим
  \begin{equation}
    \dfrac{u_{i + 1} - u_{i - 1}}{2h} = f_i + O(h^2)
  \end{equation}
\end{example}

Есть и другие варианты построения. Например, строить интерполяционный полином
и брать производную. Использование квадратур с многими внутренними узлами приводит
к так называемым \textbf{компактным разностным операторам}.

Если в формулы \eqref{23-ex} вычислять интеграл с использованием трёхточечной
формулы Симпсона, то получим
\begin{equation}
  \dfrac{u_{i + 1} - u_{i - 1}}{2h} = \dfrac{1}{6}(f_{i - 1} + 4f_i + f_{i + 1})
  + O(h^4).
\end{equation}

В этом случае без расширения шаблона достигается более высокий порядок аппроксимации,
но при этом вычисление связано с обращением трёхдиагональной матрицы.

\newpage
\begin{col-answer-preambule}
	\begin{plan}
    \item пока пусто
	\end{plan}
\end{col-answer-preambule}

\colquestion{Основные понятия теории разностных схем}

\begin{notes}
  \item пока пусто
\end{notes}

\newpage
\begin{col-answer-preambule}
	\begin{plan}
    \item пока пусто
	\end{plan}
\end{col-answer-preambule}

\colquestion{Интегро-интерполяционный метод}

\begin{notes}
  \item пока пусто
\end{notes}

\newpage
\begin{col-answer-preambule}
	\begin{plan}
    \item пока пусто
	\end{plan}
\end{col-answer-preambule}

\colquestion{Разностные схемы повышенного порядка аппроксимации}

Если коэффициенты и решение дифференциальной азадчи являются достаточно гладкими функциями,
то можно строить разностные схемы повышенного порядка аппроксимации
\begin{equation}
  Lu = -u''(x) = f(x)
\end{equation}

Первая возможность построения разностной схемы повышенного порядка связана с использванием
расширенного шаблона. Например, вместо минимального трёхточечного шаблона можно использовать
пятиточечный шаблон следующего вида
\begin{align}
  &\Lambda_u = -\dfrac{1}{12h^2}\left(-u_{i - 2} + 15u_{i - 1} - 30u_i + 16u_{i + 1}
  - u_{i + 2}\right)\\
  &\psi = Lu - \Lambda u = O(h^4)
\end{align}

Вторая возможность потсроения разностной схемы повышенного порядка аппроксимации: можно
получить без расширения шаблона засчёт использования компактных аппроксимаций
\begin{align}
  \label{26-taylor1}
  &u_{\overline{x}x} = u'' + \dfrac{h^2}{12}u^{(4)} + O(h^4)\\
  \label{26-taylor2}
  &u_{\overline{x}x} = \dfrac{1}{h^2}(u_{i + 1} - 2u_i + u_{i - 1}) = u'' + O(h^2)\\
  \label{26-taylor3}
  &\eqref{26-taylor1}, \eqref{26-taylor2} \Rightarrow
  u_{\overline{x}x} = u'' + \dfrac{h^2}{12}u''_{xx} + O(h^4)\\
  \label{26-diff}
  &\Delta u_i \equiv u_{i + 1} - 2u_i + u_{i - 1}\\
  \label{26-frac}
  &\eqref{26-taylor3}, \eqref{26-diff} \Rightarrow \dfrac{\Delta u_i}{h^2} =
  \left(1 + \dfrac{\Delta}{12}\right)u_i''\\
  \label{26-diff-approx}
  &u_i'' = \dfrac{1}{h^2}\left(1 + \dfrac{\Delta}{12}\right)^{-1}\Delta u_i
\end{align}

Формула \eqref{26-diff-approx} представляет собой неявную аппроксимацию второй
производной. Формулу \eqref{26-frac} можно записать в виде
\begin{equation}
  \Lambda u = -u_{\overline{x}x}, \phi = \dfrac{1}{12}\left(f(x_i - h) +
  10f(x_i) + f(x_i + h)\right)
\end{equation}

TODO: Способ повышения порядка аппроксимации на решении исходного ду.

\newpage
\begin{col-answer-preambule}
	\begin{plan}
    \item пока пусто
	\end{plan}
\end{col-answer-preambule}

\colquestion{Разностные схемы для уравнения Пуассона}

\begin{notes}
  \item пока пусто
\end{notes}

\newpage
\begin{col-answer-preambule}
	\begin{plan}
    \item пока пусто
	\end{plan}
\end{col-answer-preambule}

\colquestion{Аппроксимация краевых условий 2-го и 3-го рода}

\begin{notes}
  \item пока пусто
\end{notes}

\newpage
\begin{col-answer-preambule}
	\begin{plan}
    \item пока пусто
	\end{plan}
\end{col-answer-preambule}

\colquestion{Монотонные разностные схемы}

\begin{notes}
  \item пока пусто
\end{notes}

\newpage
\begin{col-answer-preambule}
	\begin{plan}
    \item пока пусто
	\end{plan}
\end{col-answer-preambule}

\colquestion{Явная левостороняя схема для уравнения переноса}

\begin{notes}
  \item пока пусто
\end{notes}

\newpage
\input{question31}
\newpage
\begin{col-answer-preambule}
	\begin{plan}
    \item пока пусто
	\end{plan}
\end{col-answer-preambule}

\colquestion{Начальная краевая задача для уравнения переноса}

\begin{notes}
  \item пока пусто
\end{notes}

\newpage
\begin{col-answer-preambule}
	\begin{plan}
    \item пока пусто
	\end{plan}
\end{col-answer-preambule}

\colquestion{Явная схема для уравнения теплопроводности}

\begin{notes}
  \item пока пусто
\end{notes}

\newpage
\input{question34}
\newpage
\end{document}
