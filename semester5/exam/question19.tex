\begin{col-answer-preambule}
	\begin{plan}
    \item пока пусто
	\end{plan}
\end{col-answer-preambule}

\colquestion{Методы Рунге-Кутта}

Исходное уравнение $u' = f(x, u)$. Интегрируем на отрезке $[x_n, x_n + h]$.

\begin{equation}
  \label{19-integral}
  u_{n + 1} = u_n + h\int\limits_0^1f(x_n + \alpha h, u(x_n + \alpha h))d\alpha
\end{equation}

Для вычисления интеграла предлагается использование следующего набора параметров

\begin{tabular}{c|c|c c c c}
  $A_0$ & & & & & \\
  $A_1$ & $\alpha_1$ & $\beta_{10}$ & & & \\
  $A_2$ & $\alpha_2$ & $\beta_{20}$ & $\beta_{21}$ & & \\
  $\vdots$ & $\vdots$ & $\vdots$ & $\vdots$ & $\ddots$ & \\
  $A_q$ & $\alpha_q$ & $\beta_{q0}$ & $\beta_{q1}$ & $\ldots$ & $\beta_{qq-1}$
\end{tabular}

При помощи параметров $\alpha$ и $\beta$ последовательно находим
\begin{equation}
  \label{19-alpha-beta}
  \begin{split}
    &\phi_0 = h f(x_n, y_n)\\
    &\phi_1 = h f(x_n + \alpha_1 h, y_n + \beta_{10}\phi_0)\\
    &\ldots\\
    &\phi_q = h f(x_n + \alpha_q h, y_n + \sum\limits_{j = 0}^{q - 1}
    \beta_{qj}\phi_j)
  \end{split}
\end{equation}

Параметры $\phi_0, \phi_1, \ldots \phi_q$ находится последовательно. После этого
интеграл в \eqref{19-integral} заменяется на $\sum\limits_{i = 0}^qA_i\phi_i$.
В результате получим формулу
\begin{equation}
  \label{19-int-approx}
  y_{n + 1} = y_n + \sum\limits_{i = 0}^qA_i\phi_i
\end{equation}

Неизвестные параметры $A, \alpha, \beta$ выбираются таким образом, чтобы при
заданном значении $q$ построить метод максимально высокого порядка точности.

\begin{definition}
  Формулы \eqref{19-alpha-beta}, \eqref{19-int-approx} - метод Рунге-Кутты.
\end{definition}

Локальная погрешность
\begin{equation}
  r_q(h) = u(x_n + h) - u(x_n) - \sum\limits_{i = 0}^qA_i\phi_i
\end{equation}

Считая функцию $f$ достаточно гладной, запишем разложение в ряд Тейлора
\begin{equation}
  r_q(h) = \sum\limits_{j = 0}^k\dfrac{h^j}{j!}r_q^{(j)}(0) + O(h^{k + 1})
\end{equation}

Если параметры $A, \alpha, \beta$ выбрать таким образом, чтобы производные
\begin{equation}
  \label{19-precision}
  r_q^{(j)} = 0, \forall j = \overline{0, k}
\end{equation}
то метод будет иметь $k$-ый порядок погрешности.

\begin{examples}
  \item $q = 0$.
  \begin{align*}
    &r_0(h) = u(x_n + h) - u(x_n) - hA_0f_n\\
    &r_0'(h) = u'(x_n + h) - A_0f_n\\
    &r_0''(h) = u''(x_n + h)
  \end{align*}
  Условие \eqref{19-precision} выполняется, когда $A_0 = 1, j = \overline{0, 1}$.
  \begin{align*}
    &y_{n + 1} = y_n + hf_n\\
    &u'(x_n) = f_n
  \end{align*}
  В итоге приходим к формуле явного метода Эйлера - метода первого порядка
  точности.
  \item $q = 1$.
  \begin{equation}
    \label{19-coef1}
    u_{n + 1} - u_n = hf_n + \dfrac{h^2}{2}(f_x + ff_u)_n +
    \dfrac{h^3}{6}(f_{xx} + 2ff_{xu} + f^2f_{uu} + f(f_x + ff_u))_n + O(h^4)
  \end{equation}
  \begin{equation}
    \label{19-coef2}
    \begin{split}
      &A_0\phi_0 + A_1\phi_1 = h(A_0f_n + A_1f(x_n + \alpha_1h, u + h\beta_{10}f_n)) =\\
      &= h(A_0 + A_1)f_n + h^2A_1(\alpha_1f_x + \beta_{10}ff_u)_n +
      \dfrac{h^3}{2}A_1(\alpha_1^2f_{xx} + 2\alpha_1\beta_{10}ff_{xu} + \beta_{10}^2
      f^2f_{uu}) + O(h^4)
    \end{split}
  \end{equation}
  \begin{equation}
    \label{19-q2}
    \begin{aligned}
      &hf: &A_0 + A_1 = 1\\
      &h^2f_x: &A_1\alpha_1 = \dfrac{1}{2}\\
      &h^2ff_u: &A_1\beta_{10} = \dfrac{1}{2}
    \end{aligned}
  \end{equation}
  При выполнении условий \eqref{19-q2} нельзя добиться совпадения коэффициентов
  \eqref{19-coef1} и \eqref{19-coef2} формулы при $h^3$. Поэтому при $q = 1$
  метод Рунге-Кутты имеет второй порядок точности
  \begin{equation}
    \label{19-coef3}
    \begin{cases}
      \alpha_1 = \beta_{10} = \dfrac{1}{2A_1}\\
      A_0 = 1 - A_1
    \end{cases}
  \end{equation}
  Формула \eqref{19-coef3} даёт семейство методов Рунге-Кутты второго порядка
  точности, где $A_1$ - свободный параметр. Возьмём $A_1 = \dfrac{1}{2}$.
  \begin{equation}
    \begin{cases}
      y_{n + 1} = y_n + \dfrac{\phi_0 + \phi_1}{2}\\
      \phi_1 = hf(x_n + h, y_n + \phi_0)\\
      \phi_0 = hf_n
    \end{cases}
  \end{equation}

  Если $A_1 = 1$.
  \begin{equation}
    \begin{cases}
      y_{n + 1} = y_n + \phi_1\\
      \phi_1 = hf\left(x_n + \dfrac{h}{2}, y_n + \dfrac{\phi_0}{2}\right)\\
      \phi_0 = hf_n
    \end{cases}
  \end{equation}
\end{examples}
