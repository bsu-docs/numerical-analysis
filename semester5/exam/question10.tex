\begin{col-answer-preambule}
	\begin{plan}
    \item пока пусто
	\end{plan}
\end{col-answer-preambule}

\colquestion{Интерполяционный сплайн второго порядка}

Поскольку интерпретация многочленами высокой степени не желательна, то в случае большого числа узлов сетки используют кусочно-полиномиальную интерполяцию. В этом случае исходный отрезок (интервал) разбивают на подынтервалы, затем на каждом подынтервале строится интерполяционный полином второй или третьей степени. Недостаток такого подхода состоит в том, что производная аппроксимирующей функции на стыках интервалах терпит разрыв.

Такая интерполяция называется кусочно-параболической (если аппроксимирующий полином - парабола ($P_2(x)$)) или кусочно-кубической (в случае $P_3(x)$).

Добиться необходимой гладкости можно с помощью сплайн-функции. Рассмотрим на $[a, b]$ сетку узлов $w = \{ a = x_0 < \ldots < x_m = b \}$ и введем опредление.

Функция $S(x) \in \mathbb{C}^{p-1} [a, b]$ - сплайн степени $p$, если на каждом отрезке $\Delta_i = [x_i, x_{i + 1}], i = \overline{0, m - 1}$, является полиномом степени $p \geqslant 2$. 

\begin{equation*}
  S(x) = S_i(x) = \sum\limits_{k = 0}^p a_k^{(i)} (x - x_i)^k, i = \overline{0, m - 1},
\end{equation*}
где $(i)$ означает, что $a_k$ принадлежит $\Delta_i$.

Условие непрерывности сплайн-функции означает непрерывность $S(x)$ и ее производной вплоть до порядка $p-1$ во всех внутренних узлах сетки. Т.о., это условие дает $p (m-1)$ уравнений относительно неизвестных коэффициентов $a_k^{(i)}$. 

В задачах интерполяции $S(x)$ должна совпадать со значениями функции $f(x)$ в узлах сетки: $S(x_i) = f(x_i), i = \overline{0, m}$. Получаем еще $(m + 1)$ уравнение (условие).

Не хватает еще $(p - 1)$ уравнения. Эти уравнения обычно находят из известных граничных условий на отрезке $[a, b]$.
