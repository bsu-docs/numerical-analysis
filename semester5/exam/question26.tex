\begin{col-answer-preambule}
	\begin{plan}
    \item пока пусто
	\end{plan}
\end{col-answer-preambule}

\colquestion{Разностные схемы повышенного порядка аппроксимации}

Если коэффициенты и решение дифференциальной азадчи являются достаточно гладкими функциями,
то можно строить разностные схемы повышенного порядка аппроксимации
\begin{equation}
  Lu = -u''(x) = f(x)
\end{equation}

Первая возможность построения разностной схемы повышенного порядка связана с использванием
расширенного шаблона. Например, вместо минимального трёхточечного шаблона можно использовать
пятиточечный шаблон следующего вида
\begin{align}
  &\Lambda_u = -\dfrac{1}{12h^2}\left(-u_{i - 2} + 15u_{i - 1} - 30u_i + 16u_{i + 1}
  - u_{i + 2}\right)\\
  &\psi = Lu - \Lambda u = O(h^4)
\end{align}

Вторая возможность потсроения разностной схемы повышенного порядка аппроксимации: можно
получить без расширения шаблона засчёт использования компактных аппроксимаций
\begin{align}
  \label{26-taylor1}
  &u_{\overline{x}x} = u'' + \dfrac{h^2}{12}u^{(4)} + O(h^4)\\
  \label{26-taylor2}
  &u_{\overline{x}x} = \dfrac{1}{h^2}(u_{i + 1} - 2u_i + u_{i - 1}) = u'' + O(h^2)\\
  \label{26-taylor3}
  &\eqref{26-taylor1}, \eqref{26-taylor2} \Rightarrow
  u_{\overline{x}x} = u'' + \dfrac{h^2}{12}u''_{xx} + O(h^4)\\
  \label{26-diff}
  &\Delta u_i \equiv u_{i + 1} - 2u_i + u_{i - 1}\\
  \label{26-frac}
  &\eqref{26-taylor3}, \eqref{26-diff} \Rightarrow \dfrac{\Delta u_i}{h^2} =
  \left(1 + \dfrac{\Delta}{12}\right)u_i''\\
  \label{26-diff-approx}
  &u_i'' = \dfrac{1}{h^2}\left(1 + \dfrac{\Delta}{12}\right)^{-1}\Delta u_i
\end{align}

Формула \eqref{26-diff-approx} представляет собой неявную аппроксимацию второй
производной. Формулу \eqref{26-frac} можно записать в виде
\begin{equation}
  \Lambda u = -u_{\overline{x}x}, \phi = \dfrac{1}{12}\left(f(x_i - h) +
  10f(x_i) + f(x_i + h)\right)
\end{equation}

TODO: Способ повышения порядка аппроксимации на решении исходного ду.
