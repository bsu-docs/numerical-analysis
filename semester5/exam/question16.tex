\begin{col-answer-preambule}
	\begin{plan}
    \item пока пусто
	\end{plan}
\end{col-answer-preambule}

\colquestion{Методы Эйлера, трапеций, средней точки}

\begin{equation}
  \label{16-problem}
  \begin{cases}
    u' = f(x, u)\\
    u(x_0) = u_0.
  \end{cases}
\end{equation}

Проинтегрируем \eqref{16-problem} на отрезке $[x_n, x_{n + 1}], x_{n + 1} - x_n = h$.

\begin{equation}
  \label{16-problem-integral}
  u_{n + 1} = u_n + \int\limits_{x_n}^{x_{n + 1}}f(t, u(t))dt, u_n \equiv u(x_n)
\end{equation}

Используем для вычисления интеграла в формуле \eqref{16-problem-integral}
правило левых прямоугольников
\begin{align*}
  \int\limits_A^Bfdx \approx f(A)(B - A)
\end{align*}

\begin{equation}
  \label{16-left-rects}
  u_{n + 1} = u_n + hf_n + R_2(f)
\end{equation}

Отбрасывая в \eqref{16-left-rects} $R_2(f)$, получаем
\textbf{явную формулу Эйлера}
\begin{equation}
  y_{n + 1} = y_n + hf(x_n, y_n) = y_n + hf_n
\end{equation}

Если для вычисления интеграла в формуле \eqref{16-problem-integral} применить
формулу правых прямоугольников
\begin{align*}
  \int\limits_A^Bfdx \approx f(B)(B - A)
\end{align*}

получим \textbf{неявную формулу Эйлера}
\begin{equation}
  y_{n + 1} = y_n + hf(x_{n + 1}, y_{n + 1}) = y_n + hf_{n + 1}
\end{equation}

В общем случае неявный метод Эйлера представляет собой неявное уравнение
относительно искомого значения $y_{n + 1}$. Для решения неявного уравнения можно
использовать итерационный метод (например, метод простой итерации).

\begin{equation}
  y_{n + 1}^{k + 1} = y_n + hf(x_{n + 1}, y_{n + 1}^k), k = 0, 1, \ldots;
  n = \overline{0, N - 1}.
\end{equation}

Чтобы итерационный метод сходился, достаточно потребовать, чтобы
\begin{equation}
  h\abs{\dfrac{\delta f}{\delta y_{n + 1}}} < 1, \forall n
\end{equation}

В качестве нулевого приближения возьмём:
\begin{enumerate}
  \item $y_{n + 1}^0 = y_n$
  \item $y_{n + 1}^0 = y_n + hf_n$
\end{enumerate}

Локальная погрешность явного и неявного метода Эйлера (это погрешность
нахождения $u(x + h)$ при известном значении $u(x)$) имеет порядок $O(h^2)$.

Рассмотрим для неявного метода Эйлера:
\begin{equation}
  \begin{split}
    r_{n + 1} &= u(x_n + h) - u(x_n) - hf(x_{n} + h, u(x_n + h)) =\\
    &=u_n + hu'_n + \dfrac{h^2}{2}u''_n + O(h^3) - u_n - h(u'_n + hu''_n + O(h^2))=\\
    &= -\dfrac{h^2}{2}u''_n + O(h^3) = O(h^2)
  \end{split}
\end{equation}

Неявный метод Эйлера сложнее в реализации, но имеет значительное преимущество
перед явным за счёт своей устойчивости.

Если интеграл в правой части вычислить по формуле трапеций, то получим:
\begin{equation}
  \begin{split}
    \begin{cases}
      y_{n + 1} = y_n + \dfrac{h}{2}(f_n + f_{n + 1})\\
      y_0 = u_0, n = 0, 1, \ldots
    \end{cases}
    r_{n + 1} = O(h^3)
  \end{split}
\end{equation}

Применим для вычисления \eqref{16-problem-integral} правило средних
прямоугольников (формулу средней точки)
\begin{equation}
  \label{16-med-point}
  y_{n + 1} = y_n + hf_{n + \frac{1}{2}}; y_0 = u_0, n = 0, 1, \ldots
\end{equation}

Чтобы вычислить $f_{n + \frac{1}{2}}$ необходимо знать значение
$y_{n + \frac{1}{2}}$. Способы вычисления:
\begin{equation}
  \label{16-half1}
  y_{n + \frac{1}{2}} \approx \frac{1}{2}(y_n + y_{n + 1})
\end{equation}
\begin{equation}
  \label{16-half2}
  y_{n + \frac{1}{2}} = y_n + \frac{h}{2}f_n
\end{equation}

Если применять \eqref{16-med-point} и \eqref{16-half1}, то получим
\begin{equation}
  y_{n + 1} = y_n + hf\left(x_n + \frac{h}{2}, \frac{1}{2}(y_n + y_{n + 1})
  \right)
\end{equation}

Если применять \eqref{16-med-point} и \eqref{16-half2}, то получим
\begin{equation}
  y_{n + 1} = y_n + hf\left(x_n + \frac{h}{2}, y_n + \frac{h}{2}f_n\right)
\end{equation}
