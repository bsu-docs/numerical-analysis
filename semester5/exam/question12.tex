\begin{col-answer-preambule}
	\begin{plan}
    \item пока пусто
	\end{plan}
\end{col-answer-preambule}

\colquestion{Наилучшее приближение в линейном векторном пространстве}

В ряде случаев функцию следует аппроксимировать (приближать) не путем интерполяции, а с помощью построения наилучшего приближения.

Пусть $H$ - линейное нормированное пространство.

Требуется найти наилучшие приближения элемента $f \in H$ с помощью ЛК $\sum\limits_{j=1}^n c_j g_{j}$, которые являются линейно-независимыми $g_j \in H, j = \overline{1, n}$.

Т.о., требуется найти элемент $\phi = \sum\limits_{j=1}^n \alpha_j g_j$ такой, что $\Delta = \left| \left| f - \phi \right| \right| = \inf\limits_{c_1, \ldots, c_n} \left| \left| f - \sum\limits_{j=1}^n c_j g_j \right| \right|$.

Если такой элемент существует $\phi \in H$ существует, то он называется элементом наилучшего приближения.

\begin{theorem}
  Элемент наилучшего приближения в линейном нормированном пространстве существует.
\end{theorem}

\textit{Замечание.}

Элемент наилучшего приближения может быть не единственным.

Пространство $H$ называется строго нормированным, если из условия $|| f + g|| = ||f|| + ||g||, ||f|| ||g|| \ne 0,$ следует $f = \alpha g, \alpha \ne 0$.

\begin{theorem}
  Если пространство $H$ строго нормированно, то элемент наилучшего приближения единственен.
\end{theorem}
