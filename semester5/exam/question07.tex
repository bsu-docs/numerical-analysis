\begin{col-answer-preambule}
	\begin{plan}
    \item пока пусто
	\end{plan}
\end{col-answer-preambule}

\colquestion{Многочлен Чебышева}

В классе ортогональных функций ортогональные многочлены имеют ряд свойств.

Пусть $f(x), g(x)$ - полиномы.

$(f, g) = 0$ и $\int\limits_a^b f(x) g(x) dx = 0$ - полиномы ортогональны.

\begin{characteristics_}
  \item Ортогональные многочлены удовлетворяют трехчленному рекуррентному соотношению.
  \item Их легко вычислить и превращать в степенной ряд.
  \item Их нули разделяют друг друга (чередование нулей).
\end{characteristics_}

Многочлены Чебышева обладают как свойствами рядов Фурье, так и ортогональных полиномов.

В сущности они являются функциями Фурье.

\begin{equation*}
  T_n(x) = \cos \Big(n \Theta(x)\Big), \Theta(x) = \arccos(x).
\end{equation*}

Покажем, что $T_n(x)$ является многочленом

\begin{equation}\label{eq:cheb_2}
  T_0(x) = 1, T_1(x) = x, x \in [-1, 1].
\end{equation}

\begin{equation*}
  \cos \Big((n+1) \Theta\Big) + \cos \Big((n-1) \Theta\Big) = 2 \cos (\Theta) \cos \Big(n \Theta\Big).
\end{equation*}

\begin{equation*}
  \Theta = \arccos x.
\end{equation*}

\begin{equation}\label{eq:cheb_4}
  T_{n+1}(x) = 2x T_n - T_{n-1}(x), x \in [-1, 1].
\end{equation}

Равенство выше называется трехчленной рекуррентной формулой.

Корни полинома Чебышева:
\begin{equation}\label{eq:set_5}
  x_k = \cos \left( \dfrac{\pi \left(k + \frac{1}{2}\right)}{n} \right), k = \overline{0, n - 1}.
\end{equation}

Покажем, что полиномы Чебышева $T_l (x), l = \overline{0, n - 1}$ на множестве $\{ x_k \}$ из \eqref{eq:set_5} являются ортогональными.

\begin{equation*}
  \left( T_l, T_m \right) = \sum\limits_{k = 0}^{n-1} T_l (x_k) T_m (x_k) = \sum\limits_{k = 0}^{n-1} \cos \left( \dfrac{\pi l}{n} \left(k + \frac{1}{2}\right) \right) \cos \left( \dfrac{\pi m}{n} \left(k + \frac{1}{2}\right) \right) =
\end{equation*}

\begin{equation*}
  = \sum\limits_{k = 0}^{n-1} \left( \dfrac{1}{2} \cos\left( \dfrac{\pi (l - m)}{n} \left(k + \frac{1}{2}\right) \right)
  + \dfrac{1}{2} \cos\left( \dfrac{\pi (l + m)}{n} \left(k + \frac{1}{2}\right) \right)  \right) = \delta_{lm} M_l,
\end{equation*}
где $\delta_{ij}$ - символ Кронекера, т.е. $ \delta_{ij} = \begin{cases} 1, i = j, \\ 0, i \ne j. \end{cases}$
\begin{equation*}
  M_l = (1 + \delta_{lm}) \dfrac{\pi}{2}.
\end{equation*}

\begin{equation*}
  (T_l, T_m) = 0 \; \forall \; l \ne m.
\end{equation*}

Если функция $f(x)$ задана на множестве узлов $\{ x_k \}$ по формуле \eqref{eq:set_5}, то можно построить интерполяционный полином.

\begin{equation}\label{eq:cheb_8}
  P_{n-1}(x) = \sum\limits_{l = 0}^{n-1} C_l T_l,
\end{equation}
где $C_l = \sum\limits_{k = 0}^{n-1} \dfrac{f_k T_l (x_k)}{M_l}$.

\begin{proof}
  \begin{equation*}
    \left( P_{n-1}(x), T_m \right) = \sum\limits_{k = 0}^{n-1} P_{n-1}(x_k) T_m (x_k) = \sum\limits_{k = 0}^{n-1} f_k T_m(x_k)
  \end{equation*}
  \begin{equation*}
    \left( P_{n-1}(x), T_m \right) =( \sum\limits_{l = 0}^{n-1} C_l T_l (x_l), T_m ) = \sum\limits_{l = 0}^{n-1} C_l (T_l, T_m) = C_m M_m.
  \end{equation*}
  \begin{equation*}
	C_m = \sum\limits_{k = 0}^{n-1} \dfrac{f_k T_m (x_k)}{M_m}
  \end{equation*}
\end{proof}

Из \eqref{eq:cheb_2}, \eqref{eq:cheb_4} следует, что
\begin{equation*}
  T_n (x) = 2^{n-1} x^n + \ldots (n \geqslant 1).
\end{equation*}

\begin{equation*}
  \left| \cos \Big( n \arccos x \Big) \right| = 1.
\end{equation*}

\begin{equation*}
  n \arccos x = \pi m, m = \overline{0, n}, x \in [-1, 1].
\end{equation*}

Отсюда находим $x_m = \cos \left( \dfrac{\pi m}{n} \right)$.

\begin{equation*}
  T_n (x_m) = \cos \left( n \arccos \dfrac{\pi m}{n} \right) = \cos \Big( \pi m \Big) = (-1)^m.
\end{equation*}
