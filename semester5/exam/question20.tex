\begin{col-answer-preambule}
	\begin{plan}
    \item пока пусто
	\end{plan}
\end{col-answer-preambule}

\colquestion{Экстраполяционные методы Адамса}

Данный метод относится к многошаговым, которые имеют следующий вид
\begin{equation}
  \label{20-method}
  y_{n + 1} = y_n + h\sum\limits_{i = 0}^qA_if_{n - i}
\end{equation}

Неизвестные параметры $A_i$ можно определить также, как и при построении методов
последовательного повышения порядка точности, т.е. из системы
\begin{equation}
  \label{20-system}
  \sum\limits_{i = 0}^qA_i(-i)^j = \dfrac{1}{j + 1}, j = \overline{0, q}
\end{equation}

Система \eqref{20-system} имеет единственное решение.

Погрешность явного метода Адамса можно также получить из формулы для метода
последовательного повышения порядка точности
\begin{equation}
  r_{n + 1} = h^{q + 2}u_n^{(q + 2)}\left(\dfrac{1}{(q + 2)!} -
  \dfrac{1}{(q + 1)!}\sum\limits_{i = 0}^qA_i(-i)^{q + 1}\right) + O(h^{q + 3})
  = O(h^{q + 2})
\end{equation}

Параметры $A_i$ можно построить, не решая систему \eqref{20-system}. Рассмотрим
интегральное уравнение (получено интегрированием исходного уравнения на $[0, 1]$)
\begin{equation}
  \label{20-int-eq}
  u_{n + 1} = u_n + h\int\limits_0^1f_{n + \alpha}d\alpha
\end{equation}

Подинтегральную функцию $f$ можно заменить интерполяционным полиномом,
построенным по точкам $0, -1, \ldots -q$. Поскольку узлы интерполирования
$x_n, \ldots x_{n - q}$ (или $\alpha = 0, -1, \ldots -q$) располагаются вне
отрезка $[x_n, x_{n + h}]$, то такая процедура интерполирования называется
\textbf{экстраполированием}.

Если интерполяционный полином взять в форме Лагранжа, то коэффициенты $A_i$
находятся по формуле
\begin{equation}
  A_i = \dfrac{(-1)^i}{i! (q - i)!}\int\limits_0^1
  \dfrac{\alpha(\alpha + 1)\ldots(\alpha + q)}{\alpha + i}d\alpha, i = \overline{0, q}
\end{equation}

Для организации счёта по формуле \eqref{20-method} одного известного условия
$y_0 = u_0$ не достаточно. Необходимо также задать значения $y_1, \ldots y_q$ с
той же точностью, которую имеет метод Адамса. Эти начальные значения находят
по одношаговым методам (например Рунге-Кутта). Эти одношаговые методы называются
стартовыми.

\begin{examples}
  \item $q = 0, A_0 = 1$.
  \begin{equation}
    y_{n + 1} = y_n + hf_n
  \end{equation}
  совпадает с явным методом Эйлера.
  \item $q = 1, A_0 + A_1 = 1, A_1 = \dfrac{1}{2} \Rightarrow A_0 = \dfrac{3}{2}$.
  Получаем метод Адамса второго порядка точности
  \begin{equation}
    \begin{split}
      &y_{n + 1} = y_n + \dfrac{h}{2}(3f_n - f_{n - 1})\\
      &y_0 = u_0
    \end{split}
  \end{equation}
  Получился явный двухшаговый метод. $y_1$ необходимо посчитать по формуле
  Рунге-Кутты.

  Локальная погрешность
  \begin{align*}
    r_{n + 1} = \dfrac{5}{12}h^3u_n^{(3)} + O(h^4) = O(h^3)
  \end{align*}

  \item $q = 2$.
  \begin{align*}
    \begin{cases}
      A_0 + A_1 + A_2 = 1\\
      A_1 + 2A_2 = -\dfrac{1}{2}\\
      A_1 + 4A_2 = \dfrac{1}{3}
    \end{cases}\\
    A_0 = \dfrac{23}{12}, A_1 = -\dfrac{4}{3}, A_2 = \dfrac{5}{12}
  \end{align*}
  \begin{equation}
    \begin{split}
      y_{n + 1} = y_n + \dfrac{h}{12}(23f_n - 16f_{n - 1} + 5f_{n - 2})\\
      y_0 = u_0
    \end{split}
  \end{equation}
  Получился трёхшаговый метод.
\end{examples}

Если подынтегральную функцию в \eqref{20-int-eq} аппроксимировать интерполяционным
полиномом в форме Ньютона, то экстраполяционный метод Адамса примет вид
\begin{equation}
  \begin{split}
    &y_{n + 1} = y_n + \phi_n + \dfrac{1}{2}\Delta \phi_{n - 1} +
    \dfrac{5}{12}\Delta^2\phi_{n - 2} + \ldots + C_q\Delta^q\phi_{n - q}\\
    &\phi_i = h f_i, C_q = \dfrac{1}{q!}\int\limits_0^1\alpha(\alpha + 1) \ldots
    (\alpha + q - 1)d\alpha\\
    &\Delta \text{ - оператор конечной разности}\\
    & \Delta \phi_{n - 1} = \phi_n - \phi_{n - 1}
  \end{split}
\end{equation}

Погрешность
\begin{equation}
  r_{n + 1} = h^{q + 2}u_n^{(q + 2)}C_{q + 1} + O(h^{q + 3}) = O(h^{q + 2})
\end{equation}
